\documentclass[11pt,openany]{book}
\usepackage[spanish]{babel}
\usepackage[utf8]{inputenc}
\usepackage[colorlinks=true, allcolors=black]{hyperref}
\usepackage[font=small]{caption}
\usepackage{amssymb,amsmath,amsthm,amsfonts,cancel}
\usepackage{graphicx}
\usepackage{multirow}
\usepackage{enumitem}
\usepackage{tikz-cd}
\usetikzlibrary{babel}
\usepackage{todonotes}
\usepackage{mathabx}
\usepackage[all]{xy}
%\usepackage{setspace}
%\usepackage{times}
%\usepackage{float}
%\usepackage{wrapfig}
%\usepackage[table,xcdraw]{xcolor}
%\usepackage{algorithm}
%\usepackage{algpseudocode}
%\usepackage[colorinlistoftodos]{todonotes}

\NeedsTeXFormat{LaTeX2e}
%\ProvidesPackage{quiver}[2021/01/11 quiver]
\usetikzlibrary{calc}
\usetikzlibrary{decorations.pathmorphing}

\tikzset{curve/.style={settings={#1},to path={(\tikztostart)
    .. controls ($(\tikztostart)!\pv{pos}!(\tikztotarget)!\pv{height}!270:(\mtikztotarget)$)
    and ($(\tikztostart)!1-\pv{pos}!(\tikztotarget)!\pv{height}!270:(\tikztotarget)$)
    .. (\tikztotarget)\tikztonodes}},
    settings/.code={\tikzset{quiver/.cd,#1}
        \def\pv##1{\pgfkeysvalueof{/tikz/quiver/##1}}},
    quiver/.cd,pos/.initial=0.35,height/.initial=0}

\tikzset{tail reversed/.code={\pgfsetarrowsstart{tikzcd to}}}
\tikzset{2tail/.code={\pgfsetarrowsstart{Implies[reversed]}}}
\tikzset{2tail reversed/.code={\pgfsetarrowsstart{Implies}}}
\tikzset{no body/.style={/tikz/dash pattern=on 0 off 1mm}}

\DeclareMathOperator\Img{Img}
\DeclareMathOperator\Ker{Ker}
\DeclareMathOperator\Coim{Coim}
\DeclareMathOperator\Coker{Coker}
\DeclareMathOperator\Ann{Ann}
\renewcommand\Im\Img
\newcommand\Coeq{\mathrm{Coeq}}  
\renewcommand\ker\Ker
\newcommand{\Mod}{\mathrm{M\acute{o}d}}
\newcommand{\Supp}{\mathrm{Supp}}
\newcommand{\Com}{\mathrm{Com}}
\newcommand{\Hom}{\mathrm{Hom}}
\newcommand{\Tor}{\mathrm{Tor}}
\newcommand{\tor}{\mathrm{tor}}
\newcommand{\Ext}{\mathrm{Ext}}
\newcommand{\SEC}{\mathrm{SEC}}
\newcommand{\Ab}{\mathrm{Ab}}
\newcommand{\Psh}{\mathrm{Psh}}
\newcommand\<\langle
\renewcommand\>\rangle
\newcommand\ol\overline

\newcommand{\SE}{\mathrm{SE}}
\newcommand{\id}{\mathrm{id}}
\newcommand{\U}{\mathcal{U}}
\newcommand{\F}{\mathcal{F}}
\newcommand{\G}{\mathcal{G}}
\newcommand{\bul}{{\scalebox{0.6}{$\bullet$}}}
\newcommand\C{\mathbb{C}}
\newcommand\R{\mathbb{R}}
\newcommand\Z{\mathbb{Z}}
\newcommand{\ds}{\text{-}}
\newcommand{\from}{\leftarrow}
\newcommand{\repi}{\twoheadrightarrow}
\newcommand{\rmono}{\hookrightarrow}
\newcommand{\xto}{\xrightarrow}
\newcommand{\xfrom}{\xleftarrow}
\renewcommand{\epsilon}{\varepsilon}
\let\cal\mathcal
\newcommand\colim\varinjlim
\newcommand\m{\mathfrak{m}}
\newcommand\maxSpec{\mathrm{maxSpec}}
\newcommand\homotopy[4]{\ar[from=#1,to=#2,"\rotatebox{#3}{\scalebox{#4}{$\sim$}}",phantom]}
\newcommand\cart[4]{\ar[from=#1,to=#2,"\rotatebox{#3}{\scalebox{#4}{$\ulcorner$}}" description,phantom,very near end]}

%\spanishdecimal{.}
\newtheorem{theorem}{Teorema}
\newtheorem{definition}{Definición}
\newtheorem{corollary}[theorem]{Corolario}
\newtheorem{proposition}[theorem]{Proposición}
\newtheorem{prop}[theorem]{Proposición}
\newtheorem{lemma}[theorem]{Lema}
\newtheorem{example}{Ejemplo} 
\newtheorem{remark}{Observación} 

\author{Lorena M. Noh Canul}
\date{Noviembre 2021}
\title{Introducción al álgebra homológica}
\textwidth=6.5in
\textheight=8.0in
\oddsidemargin=0in
\evensidemargin=0in
\begin{document}
\begin{titlepage}	
	\begin{center}
		\vspace*{-1in}	
		%\begin{LARGE}
		UNIVERSIDAD DE GUADALAJARA\\
		\vspace*{0.15in}
		CENTRO UNIVERSITARIO DE CIENCIAS EXACTAS E INGENIERÍAS\\
		\vspace*{0.15in}
		DIVISIÓN DE CIENCIAS BÁSICAS\\
		%\end{LARGE}
		\vspace*{0.5in}
		\begin{figure}[htb]
			\begin{center}
				\includegraphics[width=5cm]{logo_udg_negro.png}
			\end{center}
		\end{figure}
		\begin{Large}
			\textbf{Introducción al álgebra homológica}\\
			\vspace*{0.3in}
			\vspace*{0.15in}
			M. Noh Canul\\
			Alejandro Vázquez Aceves\\
			Alfredo Álvarez Contreras\\
			\vspace*{0.15in}
			\vspace*{0.15in}
		\end{Large}
		\vspace*{0.3in}
		%\rule{80mm}{0.1mm}\\
		\vspace*{0.15in}
		\begin{Large}
			Notas del curso de: \\
			Dr. Luis Ángel Zaldívar Corichi \\
		\end{Large}
	\end{center}
	\vspace*{0.3in}
	\begin{flushright}
		Guadalajara, Jalisco, Noviembre, 2021
	\end{flushright}
\end{titlepage} 
%\chapter*{Agradecimientos} 
%\addcontentsline{toc}{chapter}{Agradecimientos} % si queremos que aparezca en el índice
\listoftodos
\tableofcontents

\chapter{Complejos de cadenas en \texorpdfstring{$R$-$\Mod$}{RMod}}

\section{La categoría de complejos}

\begin{definition}[Complejo]
Un \textbf{complejo de cadenas} (o simplemente un complejo)
en $R$-$\Mod$ es una pareja $(C_\bul,d_\bul)$
donde $C_\bul=(C_n)_{n\in\mathbb{Z}}$ es una sucesión de objetos
en $R$-$\Mod$ y
$d_\bul=(d_{n}:C_n\rightarrow C_{n-1})_{n\in\mathbb{Z}}$
es una sucesión de morfismos, llamados diferenciales,
que satisfacen $d_nd_{n+1}=0$ para toda $n\in\mathbb{Z}$
\end{definition}

Notemos que la condición $d_nd_{n+1}=0$ es equivalente a
$$ \textup{Im}(d_{n+1})\subseteq \textup{Ker}(d_n). $$

\begin{definition}[Exactitud]
Decimos que un complejo $C_\bul$
\[
    \cdots\to C_{n+1}\xto{d_{n+1}}
    C_{n}\xto{d_n}
    C_{n-1}\xto{d_{n-1}}
    C_{n-2}\to\cdots
\]
es exacto en grado $n$ si $\Ker d_n = \Im d_{n+1}$ o,
equivalentemente, si $\Coim d_n = \Coker d_{n+1}$

También decimos que el complejo $C_\bul$ es exacto,
o bien, una sucesión exacta,
si es exacto en grado $n$ para todo $n\in\Z$.
\end{definition}

\begin{definition}[Morfismo de complejos]
Sean $(C_\bul,d_\bul)$, $(C_\bul',\delta_\bul)$ dos compejos.
Un morfismo
$$f_\bul: (C_\bul,d_\bul)\rightarrow (C_\bul',\delta_\bul) $$
es una sucesión de morfismos en $R$-$\Mod$
$$(f_n':C_n\rightarrow C_n')_{n\in\mathbb{Z}} $$
tal que el siguiente diagrama 
$$\begin{tikzcd}
	\cdots & {C_{n+1}} & {C_n} & {C_{n-1}} & {C_{n-2}} & \cdots \\
	\cdots & {C'_{n+1}} & {C_n'} & {C_{n-1}'} & {C_{n-2}} & \cdots
	\ar[from=1-1, to=1-2]
	\ar["{d_{n+1}}", from=1-2, to=1-3]
	\ar["{d_n}", from=1-3, to=1-4]
	\ar["{d_{n-1}}", from=1-4, to=1-5]
	\ar[from=1-5, to=1-6]
	\ar[from=1-2, to=2-2]
	\ar["{\delta_{n+1}}"', from=2-2, to=2-3]
	\ar["{\delta_{n}}"', from=2-3, to=2-4]
	\ar["{\delta_{n-1}}"', from=2-4, to=2-5]
	\ar[from=1-3, to=2-3]
	\ar[from=1-4, to=2-4]
	\ar[from=1-5, to=2-5]
	\ar[from=2-1, to=2-2]
	\ar[from=2-5, to=2-6]
\end{tikzcd}$$
conmuta en $R$-$\Mod$ (o en una categoría abstracta $\mu$).
\end{definition}

Así obtenemos la categoría de complejos sobre $\mu$, a la cual denotamos por $\Com(\mu)$.
Si $\mu=R$-$\Mod$, entonces $\-$ escribimos
$\Com(R)=\Com(R\text{-}\Mod)$.

\begin{example}[Módulos como complejos]
Fijemos $A\in R$-$\Mod$ y $k\in\mathbb{Z}$.
Definimos el complejo $\rho^k(A)$ cuyo $k$-ésimo t\'ermino es $A$
y las diferenciales son cero
$$\begin{tikzcd}
\cdots \ar[r] & 0 \ar[r] & A \ar[r] & 0 \ar[r] & \cdots
\end{tikzcd}$$
a este complejo se le llama el \textit{complejo $k$-centrado en $A$}. \\
Además, si $A,B$ son $R$-módulos, los morfismos de complejos
$\rho^k(A)\to\rho^k(B)$ están en correspondencia uno a uno con
los morfismos de $R$-módulos $A\to B$.
\end{example}

\begin{example}[Morfismos como complejos]
Sea $f:A\rightarrow B$. Definimos el complejo $\Sigma^k(f)$
cuyo $k$-ésimo diferencial es $f$ y todos los demás diferenciales son cero.
En particular $\Sigma^k(f)_k=A$ y $\Sigma^k(f)_{k-1}=B$.
$$
\begin{tikzcd}
\cdots \ar[r] & 0 \ar[r] & A \ar[r, "f"] & B \ar[r] & 0 \ar[r] & \cdots
\end{tikzcd}
$$
\end{example}

\begin{example}[Sucesiones exactas como complejos]
Recordemos que una sucesión de morfismos y módulos
\[ 0\to A\xto f B\xto g C\to 0 \]
es exacta cuando $f$ es inyectiva, $g$ es suprayectiva y $\Ker g=\Im g$.
En este caso, el diagrama
$$
\begin{tikzcd}
\cdots \ar[r]
& 0 \ar[r]
& 0 \ar[r]
& A \ar[r, "f"]
& B \ar[r, "g"]
& C \ar[r]
& 0 \ar[r]
& 0 \ar[r]
&\cdots
\end{tikzcd}
$$
es un complejo exacto.
\end{example}

\begin{definition}[Resoluciones]
Si $M$ es un módulo, una \textbf{resolución proyectiva} de $M$
es un complejo $P_M$
$$
\begin{tikzcd}
P_M : \cdots \ar[r] & P_3 \ar[r, "d_3"] & P_2 \ar[r, "d_2"] & P_1 \ar[r, "d_1"] & P_0 \ar[r] & 0
\end{tikzcd}
$$
donde cada $P_i$ es proyectivo,
junto con un morfismo $\epsilon:P_0\to M$, tales que la sucesión
$$
\begin{tikzcd}
\cdots \ar[r] & P_3 \ar[r, "d_3"] & P_2 \ar[r, "d_2"] & P_1 \ar[r, "d_1"] & P_0 \ar[r, "\varepsilon"] & M \ar[r] & 0
\end{tikzcd}
$$
es exacta. En particular, $\epsilon$ debe ser suprayectivo.

Si cada $P_i$ es libre, decimos que $(P_M,\epsilon)$
es una resolución libre de $M$.
Si cada $P_i$ es plano, decimos que
$(P_M,\epsilon)$ es una resolución plana de $M$.
Así,
$$
 \begin{tikzcd}
\textup{ Resolución libre} \ar[r, Rightarrow] & \textup{Resolución proyectiva} \ar[r, Rightarrow] & \textup{Resolución plana}.
\end{tikzcd}
$$
\end{definition}
Supongamos que $(P_M,\epsilon)$ es una resolución de un módulo $M$.
Si solo tenemos el complejo $P_M$, podemos recuperar el morfismo
$\epsilon:P_0\to M$ como el conúcleo de $d_1$, ya que la exactitud de
$P_1\xto{d_1} P_0\xto{\epsilon} M\to 0$ implica que existe un único isomorfismo
$\Coker(d_1)\to M$ que hace conmutar el diagrama
\[
\begin{tikzcd}
    \cdots \ar[r]
    & P_2 \ar[r, "d_2"]
    & P_1 \ar[r, "d_1"]
    & P_0 \ar[r] \ar[d, "\varepsilon"']
    & \Coker(d_1) \ar[r] \ar[dl,bend left=15,"\simeq"]
    & 0 \\
    &&& M \ar[d] \\
    &&& 0
\end{tikzcd}
\]
En particular, si $M\neq 0$, entonces el complejo $P_M$ no es exacto en $P_0$,
ya que $\Coim(P_0\to 0)=0$, mientras que $\Coker(P_1\xto{d_1}P_0)=M$.

\begin{proposition}
Todo módulo tiene una resolución proyectiva.
\end{proposition}
\begin{proof}
Si $M$ es proyectivo entonces se cumple trivialmente. Por tanto, supongamos que $M$ no es proyectivo, así existe un $F_0$ libre tal que
$$
\begin{tikzcd}
0 \ar[r] & K_1 \ar[r, "\iota_1"] & F_0 \ar[r, "\varepsilon"] & M \ar[r] & 0
\end{tikzcd}
$$
donde $K_1=\ker(\varepsilon)$. Recordemos que 
$$
 \begin{tikzcd}
\textup{Libre} \ar[r, Rightarrow] & \textup{Proyectivo} \ar[r, Rightarrow] & \textup{Plano}
\end{tikzcd}
 $$
 Así, si $K_1$ es libre el resultado queda demostrado. De lo contrario, lo cubrimos con un libre $F_1$, es decir, existe $\varepsilon_1:F_1\rightarrow K_1$ tal que 
 donde $K_2=\ker(\varepsilon_1)$.
 $$
 \begin{tikzcd}
0 \ar[r] & K_2 \ar[r, "\iota_2"] & F_1 \ar[r, "\varepsilon_1"] & K_1 \ar[r] & 0
\end{tikzcd}
 $$
 Repitiendo este proceso, para cada $n$ obtenemos una
 sucesión exacta corta
 $$
 \begin{tikzcd}
0 \ar[r] & K_{n+1} \ar[r, "\iota_{n+1}"] & F_n \ar[r, "\varepsilon_n"] & K_n \ar[r] & 0
\end{tikzcd}
 $$
 Así, uniendo las sucesiones tenemos
 $$
 \begin{tikzcd}[column sep=10]
& & 0 \ar[dr] & & 0 & & 0 \ar[dr] & & 0 & \\
& & & K_3 \ar[ur] \ar[dr] & & & & K_1 \ar[dr, "\iota_1"] \ar[ur] & & \\
\dots \ar[rr] & & F_3 \ar[ur] \ar[rr] & & F_2 \ar[rr] \ar[dr] && F_1 \ar[rr] \ar[ru] && F_0 \ar[rd, "\varepsilon"] \\
& K_3 \ar[ur] & & & & K_2 \ar[dr] \ar[ur] & & & & M \ar[rd] \\
0 \ar[ur] & & & & 0 \ar[ur] & & 0 & & & & 0
\end{tikzcd}
 $$
 Notemos que 
 $$ \Im(d_1)=K_1=\ker(\varepsilon),$$
 $$\ker(d_1)=\ker(\varepsilon_1)= K_2, $$
 lo que prueba que la sucesión
 \[
    \begin{tikzcd}
    \cdots \ar[r] & F_3 \ar[r] & F_2 \ar[r] & F_1 \ar[r] & F_0 \ar[r] & M \ar[r] & 0
    \end{tikzcd}
\]
 es exacta en el primer nivel y así tenemos la resolución proyectiva.
 \end{proof}
 
\begin{definition}
Una \textbf{resolución inyectiva} de un módulo $M$ es una sucesión exacta 
$$
E_M: \quad \begin{tikzcd}
0 \ar[r] & M \ar[r, "\delta"] & E^0  \ar[r, "d^0"] & E^1 \ar[r, "d^1"] & \cdots
\end{tikzcd}
$$
donde cada $E^i$ es inyectivo. 
\end{definition}
La resolución omitida 
$$
\begin{tikzcd}
0 \ar[r] & E^0  \ar[r, "d^0"] & E^1 \ar[r, "d^1"] & \cdots
\end{tikzcd}
$$
es tal que, ya que $\textup{Im}(d^1)=\ker(d^0)=M$, entonces no perdemos información, nuevamente
$$
\begin{tikzcd}
0 \ar[r] \ar[rd] & E^0 \ar[r, "d^0"]  & E^1  \ar[r, "d^1"] & E^2 \ar[r] & \cdots \\
                       & M=\ker(d^0) \ar[u] &                       &               &       
\end{tikzcd}
$$
y así tenemos el siguiente resultado: 

\begin{proposition}
Todo módulo tiene resolución inyectiva
\end{proposition}
\begin{proof}
Si $M$ es inyectivo se cumple trivialmente. De lo contrario, existe $\eta_0:M\rightarrow E$ tal que
$$
\begin{tikzcd}
0 \ar[r] & M \ar[r, "\eta_0"] & E^0 \ar[r, "\delta"] & \Coker(\eta_0) \ar[r] & 0
\end{tikzcd}
$$
de aquí
$$
\begin{tikzcd}[column sep = 10]
             &  &              &  &                           &                                     & 0 \ar[rd]              &                                     & 0              &  &        \\
             &  &              &  &                           &                                     &                           & \Coker(\eta_1) \ar[ru] \ar[rd] &                &  &        \\
0 \ar[rr] &  & M \ar[rr] &  & E^0 \ar[rr] \ar[rd] &                                     & E^1 \ar[ru] \ar[rr] &                                     & E^2 \ar[rr] &  & \cdots \\
             &  &              &  &                           & \Coker(\eta_0) \ar[ru, "\eta_1"'] &                           &                                     &                &  &        \\
             &  &              &  & 0 \ar[ru]              &                                     &                           &                                     &                &  &       
\end{tikzcd}
$$
lo que prueba que la sucesión es exacta. 
\end{proof}

Notemos que, dado un complejo
$$
C_\bul:\quad
\begin{tikzcd}
    \cdots \ar[r]
    & C_{n-1} \ar[r]
    & C_n \ar[r]
    & C_{n-1} \ar[r]
    & \cdots
\end{tikzcd}
$$
y un funtor aditivo covariante 
$$
F:R\textup{-}\Mod\rightarrow R\textup{-}\Mod
$$
la sucesión de módulos y morfismos
$$
FC_\bul:\quad
\begin{tikzcd}
    \cdots \ar[r]
    & FC_{n-1} \ar[r]
    & FC_n \ar[r]
    & FC_{n-1} \ar[r]
    & \cdots
\end{tikzcd}
$$
sigue siendo un complejo.
Si $P_M$ es una resolución proyectiva de un módulo $M$
tenemos que $F P_{M}$ es un complejo, pero no es necesariamente
exacto.
Por ejemplo, $\Hom (-,-)$ no preserva la exactitud.
En el caso de que $F$ sea contravariante entonces solo hacemos un cambio en los índices de la resolución y lo que obtenemos es similar. 

\begin{definition}
Un complejo $C_\bul$ es \textbf{positivo} si $C_n=0$ para toda $n<0$. 

Un complejo  $C_\bul$ es \textbf{negativo} si $C_n=0$ para toda $n>0$.
\end{definition}
Denotamos por $\Com_{\geq 0}(R)$ a los complejos positivos y por $\Com^{\leq 0}(R)$ a los complejos negativos.
\begin{definition}
Dado un complejo 
$$
\begin{tikzcd}
\cdots \ar[r] & C_{n+1} \ar[r, "d_{n+1}"] & C_n \ar[r, "d_n"] & C_{n-1} \ar[r] & \cdots
\end{tikzcd}
$$
definimos:
\begin{itemize}
    \item \textbf{$n$-cadenas:} $C_n$,
    \item \textbf{$n$-ciclos:} $Z_n(C_\bul)=\Ker{d_n}$,
    \item \textbf{$n$-fronteras:} $B_n(C_\bul)=\Img{d_{n+1}}$.
\end{itemize}
\end{definition}
Notemos que de la condición 
$$d_nd_{n+1}=0 $$
tenemos que 
$$B_n(C_\bul) \subseteq Z_{n}(C_\bul)$$
y el complejo es exacto en grado $n$ si esta contención es una igualdad.
Para medir la falta de exactitud en grado $n$, podemos formar el cociente.
Tenemos la siguiente definición.
\begin{definition}
Si $C_\bul$ es un complejo en $\Com(R)$, para $n\in \mathbb{Z}$ su  \textbf{$n$-ésima homología} es 
$$H_n(C_\bul)=\frac{Z_n(C_\bul)}{B_n(C_\bul)},$$
y a los elementos de $H_n(C_\bul)$ se les llama \textbf{clases de homología}, y las detonamos por $[z]$.
\end{definition}
Así, para cada $n$, un complejo
$C_\bul\in \Com(R)$
tiene asociadas dos sucesiones exactas cortas
$$
\begin{tikzcd}
0 \ar[r] & Z_n(C_\bul) \ar[r] & C_n \ar[r] & B_{n-1}(C_\bul) \ar[r] & 0
\end{tikzcd}
$$
$$
\begin{tikzcd}
0 \ar[r] & B_n(C_\bul) \ar[r] & Z_n(C_\bul) \ar[r] & H_n(C_\bul) \ar[r] & 0.
\end{tikzcd}
$$
Éstas encajan en el diagrama
\[
% https://tikzcd.yichuanshen.de/
\begin{tikzcd}[column sep = 10,row sep=25]
& & & & 0 \ar[rd] & 0 & & & & & \\
& & & 0 \ar[r] & B_{n} \ar[ru] \ar[r] & Z_n \ar[r] \ar[rd] & H_n \ar[r] & 0 & & & \rotatebox{80}{$\ddots$} \\
 \cdots \ar[rd] \ar[rrr] & & & C_{n+1} \ar[ru] \ar[rrr] & & & C_n \ar[rd] \ar[rrr] & & & C_{n-1} \ar[r] \ar[ru] & \cdots \\
0 \ar[r] & B_{n+1} \ar[r] \ar[rd] & Z_{n+1} \ar[ru] \ar[r] & H_{n+1} \ar[r] & 0 & & 0 \ar[r] & B_{n-1} \ar[rd] \ar[r] & Z_{n-1} \ar[ru] \ar[r] & H_{n-1} \ar[r] & 0 \\
& 0 \ar[ru] & 0 & & & & & 0 \ar[ru] & 0 & &       
\end{tikzcd}
\]
\begin{example}[La homología de un morfismo]
\label{exam:homologia-ker-coker}
Como todo morfismo define un complejo $\Sigma^n(f)$ entonces para $n=1$, tenemos
$$
\Sigma{}^1(f):\quad \begin{tikzcd}
\cdots \ar[r] & 0 \ar[r, " d_2"] & A \ar[r, "f"] & B \ar[r," d_0"] & 0 \ar[r] & \cdots
\end{tikzcd}
$$
de donde podemos notar que $\Img{d_2}=0$ y como $d_0=0$, entonces $\Ker{d_0}=B$, y de aquí 
\begin{align*}
H_1(\Sigma^1(f)) &= \Ker{f} \\
H_0(\Sigma^1(f)) &= B/\Img{ f} =\Coker(f).
\end{align*}
Por lo tanto, 
\begin{equation*}
H_n(\Sigma^1(f)) = \left\lbrace
\begin{array}{ll}
\Ker{f} & n=1\\
\Coker f & n=0\\
0 & \textup{en otro caso}
\end{array}
\right.
\end{equation*}
\end{example}
\section{La funtorialidad de la homología}
Para cada $n$, la construcción $H_n$ convierte
un complejo $C_\bul$ en un objeto $H_n(C_\bul)$ de $\cal A$.
Ahora veremos que cada $H_n$ es un funtor
$\Com(\cal A)\to\cal A$.
Es decir, dado un morfismo de complejos
$f_\bul:C_\bul\to C'_\bul$,
tenemos un morfismo $H_nf_\bul : H_n(C_\bul)\to H_n(C'_\bul)$
y esta asignación $f_\bul\mapsto H_nf_\bul$ respeta
composiciones e identidades.

Recordemos que un morfismo $f_\bul$ de complejos
$$f_\bul:(C_\bul,d_\bul) \rightarrow (C_\bul',d_\bul') $$ 
consiste en morfismos $(f_n)$ en cada nivel
$n\in \mathbb{Z}$,
tales que cada cuadrado del
siguiente diagrama es conmutativo
$$
\begin{tikzcd}
\cdots \ar[r] & C_{n+1} \ar[r, "d_{n+1}"] \ar[d, "f_{n+1}"'] & C_n \ar[r, "d_{n}"] \ar[d, "f_{n}"'] & C_{n-1} \ar[r] \ar[d, "f_{n-1}"'] & \cdots \\
\cdots \ar[r] & C_{n+1}' \ar[r, "d_{n+1}'"'] & C_n' \ar[r, "d_{n}'"'] & C_{n-1}' \ar[r] & \cdots \\
\end{tikzcd}
$$
Queremos producir un morfismo $H_n\to H'_n$,
para lo cual primero obtendremos un morfismo $Z_n\to H'_n$
y veremos que se factoriza a través de $Z_n\to H_n$.

Tomemos un $n$-ciclo $z\in Z_n\subseteq C_n$.
Así $f_nz\in C_n'$ y $d_n'f_nz\in C_{n-1}'$, pero como el diagrama conmuta, entonces
\[
    d_n'f_nz = f_{n-1}d_nz = 0,
\]
ya que $d_nz=0$.
En otras palabras, si $z\in Z_n$, entonces $f_nz\in Z'_n$.
Es decir, el morfismo compuesto $Z_n\rmono C_n\xto{f_n}C'_n$
se factoriza a través de la inclusión $Z'_n\rmono C'_n$.
Así, obtenemos el diagrama
\[
    \begin{tikzcd}
        C_n \ar[r,"f_n"] & C'_n \\
        Z_n \ar[u,hook] \ar[r]
            & Z'_n \ar[u,hook] \ar[r]
            & H'_n.
    \end{tikzcd}
\]
donde la flecha compuesta $Z_n\to Z'_n\to H'_n$
manda $z\mapsto [f_nz]$.
Par ver que esta flecha pasa al cociente $H_n$,
basta ver que su núcleo contiene a $B_n=B_n(C_\bul)$.
Dado $d_{n+1}x\in B_n$ (con $x\in C_{n+1}$),
al aplicar la flecha $Z_n\to H'_n$ obtenemos
\[
     [f_nd_{n+1}x] = [d'_{n+1}f_{n+1}x] = 0
\]
pues $d'_{n+1}f_{n+1}x\in B'_n$.
Así, en efecto tenemos un paso al cociente,
al cual llamamos $H_nf$.
\[
    \begin{tikzcd}
        C_n \ar[r,"f_n"] & C'_n \\
        Z_n \ar[u,hook] \ar[r] \ar[d,two heads]
            & Z'_n \ar[u,hook] \ar[r]
                & H'_n \\
        H_n \ar[urr,bend right=15,dotted,"H_nf"'] 
    \end{tikzcd}
\]

Por lo tanto, 
$H_n:\Com(R)\rightarrow R\textup{-}\Mod$
es un funtor, veamos: 
\begin{itemize}
    \item $H_n(1)=1$ es obvio.
    \item Tomemos $(C_\bul,d_\bul),\quad (C_\bul',d_\bul'), \quad (C_\bul'',d_\bul'')\in \Com(R)$ tal que
    \[
    \begin{tikzcd}
        {(C_\bul,d_{\cdot})} \ar[r, "f_\bul"] \ar[rr, "(fg)_\bul"', bend right] & {(C_\bul',d_{\cdot}')} \ar[r, "g_\bul"] & {(C_\bul'',d_{\cdot}'')}
        \end{tikzcd}
    \]
    \begin{eqnarray*}
        H_n(gf)[z] &= & [(gf)_n(z)]\\
        & = & [g_n(f_n(z))]\\
         & = & H_n(g)[f_n(z)]\\
         & = & H_n(g)H_n(f)[z]
    \end{eqnarray*}
Más aún, $H_n$ es un funtor aditivo, ya que
\begin{eqnarray*}
H_n(g+f)[z] &= & [(g_n+f_n)(z)]\\
& = & [f_n(z)+g_n(z)]\\
& = & [f_n(z)]+[g_n(z)]\\
 & = & H_n(f)[z]+H_n(g)[z]
\end{eqnarray*}
\end{itemize}
Denotamos por $f^*$ ó $f_*$ a $H_n(f)$, dependiendo de si estamos hablando de cohomología u homología.

\section{Sucesión exacta larga de Homología}

Arriba probamos que cada $H_n$ es un funtor
$\Com(\cal A)\to\cal A$,
para lo cual vimos la acción de $H_n$ en complejos y en morfismos
de complejos.
En particular, al aplicar cada funtor $H_n$
a una sucesión exacta corta en $\Com(\cal A)$
\[
    0\to C'_\bul\to C_\bul\to C''_\bul\to 0
\]
obtenemos sucesiones en $\cal A$
\[
\begin{tikzcd}
0 \ar[r] &
H_{n+1}(C'_\bul) \ar[r]
& H_{n+1}(C_\bul) \ar[r]
& H_{n+1}(C''_\bul) \ar[r] & 0
\\
0 \ar[r] &
H_{n}(C'_\bul) \ar[r]
& H_{n}(C_\bul) \ar[r]
& H_{n}(C''_\bul)  \ar[r] & 0
\\
0 \ar[r] &
H_{n-1}(C'_\bul) \ar[r]
& H_{n-1}(C_\bul) \ar[r]
& H_{n-1}(C''_\bul) \ar[r] & 0
\end{tikzcd}
\]
pero estas no son exactas, en general.
Sin embargo, nuestro objetivo ahora es ver que
todas estas encajan en una sola sucesión exacta larga.
\[
\begin{tikzcd}
& \cdots \ar[r]
& H_{n+2}(C''_\bul) \ar[dll]
\\
H_{n+1}(C'_\bul) \ar[r]
& H_{n+1}(C_\bul) \ar[r]
& H_{n+1}(C''_\bul) \ar[dll]
\\
H_{n}(C'_\bul) \ar[r]
& H_{n}(C_\bul) \ar[r]
& H_{n}(C''_\bul)  \ar[dll]
\\
H_{n-1}(C'_\bul) \ar[r]
& H_{n-1}(C_\bul) \ar[r]
& H_{n-1}(C''_\bul) \ar[dll]
\\
H_{n-2}(C'_\bul) \ar[r]
& \cdots
\end{tikzcd}
\]
Esta construcción le asigna, a cada sucesión exacta corta
en $\Com(\cal A)$, una sucesión exacta (larga) en $\cal A$.
Después veremos que esta asignación se extiende a un funtor
$\SEC(\Com\cal A)\to\SE(\cal A)$,
donde $\SEC(\cal A)$ y $\SE(\cal A)$
denotan las categorías de sucesiones
exactas cortas y sucesiones exactas en $\cal A$, respectivamente.

Lo primero que haremos será construir
los morfismos que aparecen
diagonalmente en el diagrama de arriba.
A cada uno de estos morfismos
se les llama ``morfismo de conexión''.

\begin{proposition}[Descripción del morfismo de conexión]
Sea
$$
\begin{tikzcd}
0 \ar[r] & C_\bul' \ar[r, "\iota"] & C_\bul \ar[r, "\rho"] & C_\bul'' \ar[r] & 0
\end{tikzcd}
$$
una sucesión exacta corta en $\Com(R)$.
Entonces para toda $n\in\mathbb{Z}$ tenemos un morfismo
\begin{eqnarray*}
\partial_n:H_n(C_\bul'') &\rightarrow & H_{n-1}(C_\bul')\\
\textup{$[z_n'']$} & \rightarrow & [\iota^{-1}_{n-1}d_n\rho_n^{-1}z_n'']
\end{eqnarray*}
\end{proposition}
\begin{proof}
Consideremos el siguiente diagrama
$$
\begin{tikzcd}
& {} \ar[d]
& {} \ar[d]
& {} \ar[d]
& \\
0 \ar[r]
& C_{n+1}' \ar[r, "\iota_{n+1}"] \ar[d, "d_{n+1}'"]
& C_{n+1} \ar[r, "\rho_{n+1}"] \ar[d, "d_{n+1}"]
& C_{n+1}'' \ar[r] \ar[d, "d_{n+1}''"]
& 0 \\
0 \ar[r]
& C_n' \ar[r, "\iota_{n}"] \ar[d, "d_n'"]
& C_n \ar[r, "\rho_{n}"] \ar[d, "d_n"]
& C_n'' \ar[r] \ar[d, "d_n''"]
& 0 \\
0 \ar[r] & C_{n-1}' \ar[r, "\iota_{n-1}"] \ar[d] 
& C_{n-1} \ar[r, "\rho_{n-1}"] \ar[d] 
& C_{n-1}'' \ar[r] \ar[d] 
& 0 \\
& {}
& {}
& {}
&  
\end{tikzcd}
$$
Tomemos $z \in Z_n(C_\bul'')\subseteq C_n''$ como $\rho_n$ es sobreyectiva, entonces existe $c\in C_n$ tal que $\rho_n(c)=z$. Aplicando el diferencial y por la conmutatividad del diagrama tenemos 
\begin{eqnarray*}
\rho_{n-1}d_n c & = & d_n''\rho_n c \\
& = & d_n''z\\
& = & 0
\end{eqnarray*}
por lo tanto, $d_n c\in \Ker{\rho_{n-1}}=\Img{\iota_{n-1}}$;
así existe $c'\in C_{n-1} '$ tal que 
$$\iota_{n-1}c' = d_n c $$
y $c'$ es único ya que $\iota_{n-1}$ es inyectiva. 

Así, 
$$c'= \iota_{n-1}^{-1}d_nc = \iota_{n-1}^{-1}d_n\rho_n^{-1}z.$$
Ahora probemos que $\partial_n$ está bien definida y es un morfismo. Llamemos $c\in C_n$ tal que $\rho_nc=z$ un \textit{levantamiento}. Y demostremos que es único. Tomemos $\check{c}\in C_n$ tal que $\rho_n\check{c}=z$ así 
$$\check{c}-c\in \Ker{\rho_n}=\Img{\iota_n} $$
por tanto, existe $u'\in C_n'$ tal que 
$$\iota_nu'=\check{c}-c $$
como los cuadros conmutan, entonces 
\begin{eqnarray*}
\iota_{n-1}d_n'u' & = & d_n\iota_nu' \\
& = & d_n(\check{c}-c)\\
& = & d_n\check{c}-d_nc.
\end{eqnarray*}
Por tanto, 
$$ \iota_{n-1}^{-1}(d_n\check{c}-d_nc)=d_n'u'\in B_{n-1}(C_\bul')=\Img{d_n}$$
entonces tomando el módulo en homología concluimos que determinan la misma clase y, por tanto, está bien definida. Así $\partial$ no depende del levantamiento.

Ahora, tomemos 
\begin{eqnarray*}
Z_n(C_\bul'')&\rightarrow& C_{n-1}'/B_{n-1}(C_\bul')\\
z_n'' & \mapsto & [c_{n-1}']
\end{eqnarray*}
que está bien definida por lo anterior. 
Más aún, sean $z'',z\in Z_n(C_\bul'')$ tales que 
$$\rho_nc'=z'' $$
$$\rho_nc_1 =z$$
nos fijamos en 
\begin{eqnarray*}
\rho_nc'+\rho_nc_1& = & z''+z\\
\rho_n(c'+c_1) & = & z''+z
\end{eqnarray*}
y como el levamiento es único, entonces 
$$z+z''\mapsto [c'+c_1]=[c']+[c_1].$$
Ahora, mostremos de $\iota_{n-1}c'=d_nc'$ que $c'$ es un ciclo único. Así, 
$$d_{n-1}\iota_{n-1}c'=\iota_{n-2}d_{n-1}'c' $$
por un lado tenemos que $d_{n-1}\iota_{n-1}c'=0$, por tanto, $d_{n-1}'c'=0$, ya que $\iota_{n-2}$ es inyectiva.
Por tanto $c'$ es un ciclo y 
$$B_n''\subseteq Z''_n\rightarrow Z_{n-1}'/B_{n-1}' $$
ahora tomemos $z''=d_{n+1}c''\in B_n''$ tal que $c''\in C_{n+1}''$ y $\rho_{n+1}u=c''$ con $u\in C_{n+1}$
de donde
\begin{eqnarray*}
\rho_nd_{n+1}u & = & d_{n+1}''\rho_{n+1}u \\
& = & d_{n+1}'' c'' \\
& = & z''
\end{eqnarray*}
como $\partial(z'')$ no depende del levantamiento elegimos $d_{n+1}u$ tal que 
$$\rho_nd_{n+1}u=z''\in B_{n}(C'') $$
entonces 
\begin{eqnarray*}
\partial[z] & = & [\iota_{n-1}\underline{d_nd_{n+1}}u]\\
& = & [0],
\end{eqnarray*}
es decir, tenemos un paso al cociente
$$
\begin{tikzcd}
Z_n'' \ar[r] \ar[d]                             & H_{n-1}(C_\bul) \\
Z_n''/B_n''=H_n(C_\bul'') \ar[ru, "\partial"', bend right] &            
\end{tikzcd}
$$
\end{proof}
\begin{definition}
La familia $(\partial_n:H_n(C_\bul'')\rightarrow H_{n-1}(C_\bul'))$ lleva por nombre \textbf{morfismo de conexión}.
\end{definition}

\begin{remark}
La caza de elementos a través de un diagrama tiene la desventaja
de que luego hay que revisar si las asignaciones obtenidas
son morfismos bien definidos.
A continuación, damos un ejemplo de una demostración del resultado
anterior sin cazar elementos. En lugar de esto, cazamos propiedades
universales, lo que nos garantiza que siempre tratamos con morfismos.
\end{remark}
\begin{proof}[Demostración con propiedades universales]
Consideremos la preimagen $I_n=\rho_n^{-1}(Z''_n)$.
Como las inclusiones $I_n\to C_n$ y $Z''_n\to C''_n$
son monomorfismos, tenemos que
$\Ker(I_n\to Z''_n)=\Ker(C_n\to C''_n)=C'_n$
con inclusión dada por la restricción
de $C'_n\xto{\iota_n}C_n$.
Además, como $C_n\xto{\rho_n}C''_n$ es epi,
entonces $I_n\to Z''_n$ también lo es.
Por lo tanto, tenemos un diagrama con renglones exactos
\[
\begin{tikzcd}
0 \ar[r]
    & C'_n \ar[r] \ar[d,equal]
    & I_n \ar[r] \ar[d,hook]
    & Z''_n \ar[r] \ar[d,hook]
    & 0 \\
0 \ar[r]
    & C'_n \ar[r,"\iota_n"] \ar[d,"d'_n"]
    & C_n \ar[r,"\rho_n"] \ar[d,"d_n"]
    & C''_n \ar[r] \ar[d,"d''_n"]
    & 0 \\
0 \ar[r]
    & C'_{n-1} \ar[r,"\iota_{n-1}"]
    & C_{n-1} \ar[r,"\rho_{n-1}"]
    & C''_{n-1} \ar[r]
    & 0
\end{tikzcd}
\]
aunque solo la tercera columna es exacta,
por lo cual la conmutatividad del diagrama
nos dice que el morfismo
\[
\begin{tikzcd}[color=gray]
0 \ar[r]
    & C'_n \ar[r] \ar[d,equal]
    & |[black]| I_n \ar[r] \ar[d,hook,color=black]
    & Z''_n \ar[r] \ar[d,hook]
    & 0 \\
0 \ar[r]
    & C'_n \ar[r] \ar[d]
    & |[black]|C_n \ar[r] \ar[d,color=black,"d_n"]
    & C''_n \ar[r] \ar[d,"d''_n"]
    & 0 \\
0 \ar[r]
    & C'_{n-1} \ar[r]
    & |[black]|C_{n-1} \ar[r,"\rho_{n-1}",color=black]
    & |[black]|C''_{n-1} \ar[r]
    & 0
\end{tikzcd}
\]
es cero, así que $C_{n-1}\from C_n\from I_n$ se factoriza
de manera única a través de $\iota_{n-1}$
(por la flecha intermitente en el siguiente diagrama).
De nuevo por conmutatividad, el morfismo
\[
\begin{tikzcd}[color=gray]
0 \ar[r]
& C'_n \ar[r] \ar[d,equal]
& |[black]| I_n \ar[r] \ar[d,hook]
& Z''_n \ar[r] \ar[d,hook]
& 0 \\
0 \ar[r]
& C'_n \ar[r] \ar[d]
& C_n \ar[r] \ar[d,"d_n"]
& C''_n \ar[r] \ar[d]
& 0 \\
0 \ar[r]
& |[black]|C'_{n-1}
\ar[r]
\ar[from=uur,crossing over,dashed,color=black]
\ar[d,color=black,"d'_{n-1}"]
& C_{n-1} \ar[r] \ar[d,"d_{n-1}"]
& C''_{n-1} \ar[r]
& 0 \\
0 \ar[r]
& |[black]| C'_{n-2} \ar[r,color=black,"\iota_{n-2}"]
& |[black]| C_{n-2}
\end{tikzcd}
\]
es cero.
Como $\iota_{n-2}$ es mono, esto implica que
$C'_{n-2}\xfrom{d'_{n-1}} C'_{n-1}\dashleftarrow I_n$ es cero.
Luego, $C'_{n-1}\dashleftarrow I_n$ se factoriza de manera única
a través de $Z'_{n-1}=\Ker(d'_{n-1})$.
Así, tenemos el siguiente diagrama conmutativo
con diagonales exactas.
Por conmutatividad, los caminos resaltados
son iguales:
\[
% https://tikzcd.yichuanshen.de
\begin{tikzcd}
[
row sep={4em,between origins},
column sep={5em,between origins},
color=gray
]
& 0 \ar[dr] \\
& 0 \arrow[rd]
& |[black]| C'_n
\ar[dr,color=black] \ar[d,equal,color=black]
&&& \\
0 \arrow[rd]
& |[black]| Z'_{n-1} \ar[d,hook,color=black]
%\ar[from=ur,crossing over,dashed]
& |[black]| C'_n
\arrow[rd,color=black]
\arrow[ld,color=black,"d'_n"',very near start]
& |[black]| I_n
\ar[ll,bend left,crossing over, squiggly,color=black]
\ar[dll,dashed,bend left,crossing over]
\arrow[d,hook] \arrow[rd]
&& \\
& |[black]| C'_{n-1} \arrow[rd,color=black,"\iota_{n-1}"']
&& C_{n} \arrow[rd] \arrow[ld]
& Z''_n \arrow[d,hook] \ar[dr]
& \\
&& |[color=black]| C_{n-1} \arrow[rd]
&& C''_{n} \arrow[rd] \arrow[ld]
& 0 \\
&&& C''_{n-1} \arrow[dr] && 0 \\
&&&& 0.
\end{tikzcd}
\]
Cancelando el mono $\iota_{n-1}$, vemos que
la composición $C'_n\to I_n\rightsquigarrow Z'_{n-1}$
es $C'_n\xto{d'_n}C'_{n-1}$.
Así, factorizando $d'_n$ a través de $B'_{n-1}$ y $Z'_{n-1}$,
vemos que los dos caminos marcados
en el siguiente diagrama son iguales.
\[
% https://tikzcd.yichuanshen.de
\begin{tikzcd}[color=gray]
0 \ar[r]
& |[black]| C'_n
\ar[r,color=black] \ar[d,two heads,color=black]
\ar[ddr,"d'_n"',controls={+(-2,-1.5) and +(-2.5,0)}]
& |[black]| I_n \ar[r] \ar[d,squiggly,color=black] 
& Z''_n \ar[r]
& 0
\\
0 \ar[r,crossing over]
& |[black]| B'_{n-1} \ar[r,color=black] 
& |[black]| Z'_{n-1} \ar[r] \ar[d,hook,color=black]
& H'_{n-1} \ar[r] 
& 0 \\
&& |[black]| C'_{n-1}.
\end{tikzcd}
\]
Cancelando el mono $Z'_{n-1}\rmono C'_{n-1}$,
obtenemos la conmutatividad del cuadrado y,
como los renglones son exactos, el diagrama
se puede completar de manera única
con un morfismo $Z''_n\to H'_{n-1}$.
Para obtener el morfismo deseado $H''_n\to H'_{n-1}$,
basta ver que el morfismo recién obtendo $Z''_n\to H'_{n-1}$
anula a $B''_n$ y aplicar un paso al cociente $H''_n=Z''_n/B''_n$.
Para esto, consideremos el grado $n+1$.
Factorizando el diferencial $d''_{n+1}:C''_{n+1}\to C''_n$
como $C''_{n+1}\to B''_n\to Z''_n\to C''_n$,
tenemos el siguiente diagrama conmuativo.
\[
\begin{tikzcd}
C_{n+1} \arrow[rr,"\rho_{n+1}"] \arrow[rdd, bend right,"d_{n+1}"']
&& C''_{n+1} \arrow[rd, two heads]
\\
& I_n \arrow[r, two heads] \arrow[d, hook]
\ar[dr,phantom,"\lrcorner" description,very near start]
& Z''_n \arrow[d, hook]
& B''_n \arrow[l, hook] \\
& C_n \arrow[r, two heads,"\rho_n"']
& C''_n
\end{tikzcd}
\]
Como $I_n=\rho_n^{-1}(Z''_n)$,
el cuadrado es un producto fibrado, así que el diagrama
se puede completar con un morfismo $C_{n+1}\to I_n$.
Ahora, por conmutatividad, la flecha
\[
\begin{tikzcd}[color=gray]
& 0 \ar[dr] \\
& 0 \arrow[rd]
& C'_n \ar[dr] \ar[d,equal]
& & & \\
0 \arrow[rd]
& |[black]| Z'_{n-1} \ar[d,hook,color=black]
& C'_n \arrow[rd] \arrow[ld]
& |[black]| I_n
\ar[ll,bend left,crossing over, squiggly,color=black]
\ar[dll,dashed,bend left,crossing over]
\arrow[d,hook]
& |[black]| C_{n+1} \ar[dr] \ar[dl] \ar[l,bend right=30,color=black]
& B''_{n} \ar[dl,crossing over]  \\
& |[black]| C'_{n-1} \arrow[rd,color=black]
& & C_{n} \arrow[rd] \arrow[ld]
& Z''_n \ar[from=ul,crossing over] \arrow[d,hook]
& C''_{n+1} \ar[dl] \ar[dr] \ar[u,two heads] \\
& & |[black]| C_{n-1} \arrow[rd]
& & C''_{n} \arrow[rd] \arrow[ld]
& 0 \ar[from=ul,crossing over] &[5mm] 0 \\
& & & C''_{n-1} \arrow[rd] & & 0 \\
& & & & 0 & \\
\end{tikzcd}
\]
es cero.
Cancelando los monos
$Z'_{n-1}\rmono C'_{n-1}\rmono C_{n-1}$,
se sigue que $(Z'_{n-1} \leftsquigarrow I_n\from C_{n+1}) = 0$.
Luego, por conmutatividad, la flecha
\[
% https://tikzcd.yichuanshen.de
\begin{tikzcd}[color=gray]
&& |[black]| C_{n+1} \ar[r,color=black] \ar[dd]
& |[black]| C''_{n+1} \ar[d,two heads,color=black] \ar[r]
& 0 \\
&&& |[black]| B''_{n} \ar[d,color=black] \\
0 \ar[r]
& C'_n \ar[r] \ar[d]
& I_n \ar[r] \ar[d,squiggly] 
& |[black]| Z''_n \ar[r] \ar[d,color=black]
& 0
\\
0 \ar[r]
& B'_{n-1} \ar[r] 
& Z'_{n-1} \ar[r] 
& |[black]| H'_{n-1} \ar[r] 
& 0
\end{tikzcd}
\]
es cero.
Cancelando los epis $C_{n+1}\repi C''_{n+1}\repi B''_n$
se sigue que $(B''_n\to Z''_n\to H'_{n-1})=0$,
que es lo que queríamos.
Por lo tanto, la flecha $Z''_n\to H'_{n-1}$ pasa al cociente $H''_n$:
\[
% https://tikzcd.yichuanshen.de
\begin{tikzcd}
B''_{n} \ar[d] \\
Z''_n \ar[r] \ar[d] & H'_{n-1} \\
H''_n \ar[ru,bend right,"\partial_n"]
\end{tikzcd}
\]
y esta es la flecha buscada.

También de esta descripción se puede observar
que $\partial_n$ se calcula como
$[z''_n]\mapsto [\iota_{n-1}^{-1}d_n\rho_n^{-1}z''_n]$
\end{proof}

Ahora veremos que el morfismo de conexión, en efecto,
conecta las sucesiones dadas por cada funtor $H_n$
para formar una sucesión exacta larga.

\begin{theorem}
\label{sucesionexactalarga}
Si 
$$
\begin{tikzcd}
0 \ar[r] & C_\bul' \ar[r, "\iota_\bul"] & C_\bul \ar[r, "\rho_\bul"] & C_\bul'' \ar[r] & 0
\end{tikzcd}
$$
es una sucesión exacta corta en $\Com(R)$, entonces
las sucesiones $H_n(C'_\bul)\to H_n(C_\bul)\to H_n(C''_\bul)$
junto con los morfismos de conexión
$\partial_n:H_n(C''_\bul)\to H_{n-1}(C'_\bul)$
forman una sucesión exacta larga en $R$-$\Mod$ (o $\mu$)
\[
\begin{tikzcd}
& \cdots \ar[r]
& H_{n+2}(C''_\bul) \ar[dll]
\\
H_{n+1}(C'_\bul) \ar[r]
& H_{n+1}(C_\bul) \ar[r]
& H_{n+1}(C''_\bul) \ar[dll]
\\
H_{n}(C'_\bul) \ar[r]
& H_{n}(C_\bul) \ar[r]
& H_{n}(C''_\bul)  \ar[dll]
\\
H_{n-1}(C'_\bul) \ar[r]
& H_{n-1}(C_\bul) \ar[r]
& H_{n-1}(C''_\bul) \ar[dll]
\\
H_{n-2}(C'_\bul) \ar[r]
& \cdots
\end{tikzcd}
\]
\end{theorem}
\begin{proof} Probemos la exactitud de la sucesi\'on en cada paso. 
Primero veamos que $\Im \iota_* = \Ker \rho_*$.  Sabemos que $$\iota_*\rho_*=(\iota\rho)_*=0,$$ por lo que $$\Im \iota_*\subseteq \Ker \rho_*.$$ 
Para probar la otra contenci\'on, notemos que si $\rho_*[z]=[\rho_n z]=0$ entonces 
\begin{equation}
    \rho_n z =d_{n+1}''c'' \qquad c''\in C_{n+1}'' 
\label{1}
\end{equation}
y as\'i, por sobreyectividad de $\rho$, existe  $c\in C_{n+1}$ tal que 
$$\rho_{n+1}c=c'', $$
de \eqref{1} y de la conmutatividad del diagrama tenemos que 
$$\rho_n z =d_{n+1}\rho_{n+1}c = \rho_nd_{n+1}c$$
entonces 
$$\rho_n(z-d_{n+1} c)=0 $$
por lo tanto, 
$$z-d_{n+1}c=0 \in\Ker\rho_n=\Im \iota_{n}.$$
Por exactitud existe $c'\in C_{n}'$ con 
$\iota_nc' =z-d_{n+1}c$. 
Como $c'$ es un ciclo para 
\begin{eqnarray*}
d_n\iota_n c' & = & \iota_{n-1}d_n'c'=0\\
d_n(z-d_{n+1}c) & = & d_nz-d_nd_{n+1}c\\ 
& =&d_nz
\end{eqnarray*}
Ya que $z$ es un ciclo y $\iota$ es inyectivo, entonces 
$$\iota_{n-1}d_nc'=0. $$
Por lo tanto, $d_n'c'=0$ y $c'$ es un $n$-ciclo. Por lo tanto,  
\begin{eqnarray*}
\iota_*[c'] & = & [\iota_n c']\\
 & = & [z-d_{n+1}c]\\
 & = & [z]-[d_{n+1}c]\\
 & = & [z] 
\end{eqnarray*}
Por lo tanto, $[z]\in \Im \iota_*$. 

Veamos ahora que $\Im \rho_* =\Ker \partial$. Sea $\rho_*[c]=[\rho_nc]\in \Im \rho_*$ entonces $$\partial\rho_*[c]=[c'],$$ donde 
$\iota_*c'=d_n\rho_n^{-1} \rho_n c $. Ya que se ha demostrado que esta f\'ormula no depende del levantamiento, entonces podemos elegir a $\rho_n^{-1}\rho_n c=c$ y as\'i $d_nc=0$ ya que $c$ es un ciclo, entonces 
$$\iota_nc'=d_nc=0 $$
por lo tanto, $c'=0$ y $\partial[\rho_n c]=0$. Luego, $\rho_*[c]\in \Ker \partial$


Tomemos $\partial[z'']=[0]$ con el levantamiento correspondiente 
$$z'=[\iota^{-1}d_n\rho_n^{-1}z'']\in B_n'(C_\bul') $$
entonces
$$z'= d_n'c' \qquad c'\in C_n' .$$
Pero 
\begin{equation}
    \iota z' =\iota d_n'c'=d_{n}\iota c'
    \label{1.2}
\end{equation}
y por otro lado, 
$$\iota z' = \iota \iota^{-1}d_n\rho_n^{-1}z''=d_n\rho_n^{-1}z'' $$
as\'i que sustituyendo en \eqref{1.2} tenemos 
\begin{eqnarray*}
d_n\rho_n^{-1}z''  & = & d_n\iota c'\\
d_n(\rho_n^{-1}z''-\iota c') & = & 0
\end{eqnarray*}
y 
$$\iota z' =\iota d_n' c' =d_n\iota c' = d_n\rho_n^{-1}z'' $$
por lo tanto,  $\iota c' -\rho_n^{-1}z''$ es un ciclo y por exactitud de la sucesi\'on original 
$$\rho_*[\rho_n^{-1}z'' -\iota c'] = \rho_*[\rho_n^{-1}z'']-\rho_*[\iota c']=[z'']. $$

Finalmente, probemos $\Im\partial = \Ker \iota_*$. Para ver que $\Im\partial \subseteq \Ker \iota_*$, tenemos que 
$$\iota_*\partial [z''] =[\iota_{n+1}z'] $$
pero $\iota_nz' =d_{n+1}\rho_n^{-1}z'' \in B_n(C_\bul)$; por lo tanto, 
$\iota_*\partial[z'']=0$.

Para la otra contenci\'on, si $\iota_*[z']=0=[\iota_n z']$, entonces $\iota_n z'=d_{n+1}c$ para alg\'un $c\in d_{n+1}$. Ya que es una cadena, 
\begin{eqnarray*}
d_{n+1}''\rho_{n+1}c & = & \rho_n d_{n+1}c \\
& = & \rho_n \iota_n z'' \\
&= & 0,
 \end{eqnarray*}
 por exactitud. Por lo tanto, $\rho_{n+1}c$ es un ciclo. Pero 
 \begin{eqnarray*}
\partial[\rho_* c] & = & [\iota^{-1}d_{n+1}\rho_n\rho_n^{-1}c]\\
& = & [\iota d_n c]\\
& = & [\iota_n^{-1}\iota_n z']\\
& = & [z'].
\end{eqnarray*} 
Lo que demuestra lo deseado.
\end{proof}

Lo anterior se suele expresar diciendo que el siguiente tri\'angulo es exacto: 
$$\begin{tikzcd}[row sep = 50,column sep = 10]
H_\bul(C_\bul') \ar[rr, "\iota_*"] &   & H_\bul (C_\bul) \ar[ld, "\rho_*"] \\ & H_\bul(C_\bul'') \ar[lu, "\partial_\bul"] &      
\end{tikzcd}
$$

Esto completa la construcción que, a cada sucesión
exacta corta en $\Com(\cal A)$, le asigna una sucesión
exacta en $\cal A$.
Antes de mostrar que esta asignación es funtorial,
veremos dos resultados que se obtienen inmediatamente.

\begin{corollary}[Lema de la serpiente]
Dado el siguiente diagrama en $R\text{-}\Mod$
$$ 
\begin{tikzcd}
0 \ar[r] & A' \ar[d, "f"] \ar[r] & A \ar[d, "g"] \ar[r] & A'' \ar[d, "h"] \ar[r] & 0 \\
0 \ar[r] & B' \ar[r]                 & B \ar[r]                & B'' \ar[r]                  & 0
\end{tikzcd}
$$
entonces existe una sucesi\'on exacta en $R$-$\Mod$
$$ 
\begin{tikzcd}
0 \ar[r] & \Ker f \ar[r]    & \Ker g \ar[r]   & \Ker h \ar[lld] &   \\
            & \Coker{f} \ar[r] & \Coker g \ar[r] & \Coker h \ar[r] & 0
\end{tikzcd}
$$
\end{corollary}
\begin{proof}
Recordemos que $f,g$ y $h$ tienen asociados sus complejos
correspondientes:
\begin{align*}
    \Sigma^1f &: \cdots\to 0\to A'\xrightarrow f B' \to 0\to \cdots \\
    \Sigma^1g &: \cdots\to 0\to A\xrightarrow g B \to 0\to \cdots \\
    \Sigma^1h &: \cdots\to 0\to A''\xrightarrow h B'' \to 0\to \cdots.
\end{align*}
Entonces notemos que el diagrama con renglones exactos
\[
\begin{tikzcd}
0 \ar[r] & A' \ar[d, "f"] \ar[r] & A \ar[d, "g"] \ar[r] & A'' \ar[d, "h"] \ar[r] & 0 \\
0 \ar[r] & B' \ar[r]                 & B \ar[r]                & B'' \ar[r]                  & 0
\end{tikzcd}
\]
es una sucesión exacta corta entre estos complejos:
\[
    0\to \Sigma^1f \to \Sigma^1 g \to\Sigma^1 h \to 0.
\]
Por lo tanto, el teorema anterior nos da una
sucesión exacta larga en homología:
\[
\begin{tikzcd}
    &\cdots \ar[r] & H_2(\Sigma^1h)\ar[dll] \\
    H_1(\Sigma^1 f) \ar[r]
    &H_1(\Sigma^1 g) \ar[r]
    &H_1(\Sigma^1 h) \ar[dll] \\
    H_0(\Sigma^1 f) \ar[r]
    &H_0(\Sigma^1 g) \ar[r]
    &H_0(\Sigma^1 h) \ar[dll] \\
    H_{-1}(\Sigma^1f) \ar[r] & \cdots,
\end{tikzcd}
\]
es decir (ver el ejemplo \ref{exam:homologia-ker-coker})
\[
\begin{tikzcd}
    &\cdots \ar[r] & 0 \ar[dll] \\
    \Ker f \ar[r]
    &\Ker g \ar[r]
    &\Ker h \ar[dll] \\
    \Coker f \ar[r]
    &\Coker g \ar[r]
    &\Coker h \ar[dll] \\
    0 \ar[r] & \cdots,
\end{tikzcd}
\]
que es la sucesión deseada.
\end{proof}

Podemos pensar en las sucesiones exactas cortas en $\cal B$
como objetos de una categoría, donde los morfismos son diagramas
\[
    \begin{tikzcd}
        0\ar[r]
            & A' \ar[r] \ar[d]
            & A \ar[r] \ar[d]
            & A''\ar[r] \ar[d]
        & 0
        \\
        0\ar[r]
            & B' \ar[r]
            & B \ar[r]
            & B''\ar[r]
        & 0.
    \end{tikzcd}
\]
En otras palabras,
ya que toda sucesión exacta corta es un complejo, la categoría
$\SEC(\cal B)$ de sucesiones exactas cortas en $\cal B$
es una subcategoría plena de $\Com(\cal B)$.
Del mismo modo, podemos pensar en las sucesiones exactas
(no necesariamente cortas) como otra subcategoría $\SE(\cal B)$
de $\Com(\cal B)$.
Así, tenemos inclusiones
$\SEC(\cal B)\rmono\SE(\cal B)\rmono\Com(\cal B)$.

Hasta ahora, dada una sucesión exacta corta en $\Com(\cal A)$,
los funtores $H_n$ y los morfismos de conexión nos dan una
sucesión exacta larga en $\cal A$.
Ahora veremos que esta asignación se puede extender
a un funtor $\SEC(\Com\cal A)\to\SE(\cal A)$.

\begin{theorem}[Naturalidad del morfismo de conexión. Funtorialidad de la sucesión exacta larga]
\label{naturalidad}
Dado un diagrama en $\Com(R)$
\[
\begin{tikzcd}
0 \ar[r]
    & A'_\bul \ar[d, "f_\bul"] \ar[r,"i_\bul"]
    & A_\bul \ar[d, "g_\bul"] \ar[r,"p_\bul"]
    & A''_\bul \ar[d, "h_\bul"] \ar[r] & 0 \\
0 \ar[r]
    & B_\bul' \ar[r,"j_\bul"]                
    & B_\bul \ar[r,"q_\bul"]               
    & B''_\bul \ar[r]                 
& 0
\end{tikzcd}
\]
con renglones exactos, entonces el cuadrado en $R\ds\Mod$
\[
\begin{tikzcd}
& H_n(A''_\bul) \ar[d, "h_*"] \ar[r,"\partial_n"]
& H_{n-1}(A'_\bul) \ar[d, "f_*"]
\\
& H_n(B''_\bul) \ar[r,"\partial_n'"]
& H_{n-1}(B_\bul').
\end{tikzcd}
\]
es conmutativo. Esto significa que 
los morfismos de conexión $\partial_n$
dan una transformación natural entre los funtores
$\SEC(\Com(\cal A))\to\cal A$
que mandan una sucesión exacta corta de complejos
$0\to A'_\bul\to A_\bul\to A''_\bul\to 0$
a $H_n(A''_\bul)$ y a $H_{n-1}(A'_\bul)$.

Por lo tanto,
el diagrama entre las sucesiones exactas largas en homología
\[
\begin{tikzcd}
\cdots \ar[r]
& H_n(A'_\bul) \ar[d, "f_*"] \ar[r,"i_*"]
& H_n(A_\bul) \ar[d, "g_*"] \ar[r,"p_*"]
& H_n(A''_\bul) \ar[d, "h_*"] \ar[r,"\partial"]
& H_{n-1}(A'_\bul) \ar[d, "f_*"] \ar[r]
& \cdots \\
\cdots \ar[r]
& H_n(B_\bul') \ar[r,"j_*"] 
& H_n(B_\bul) \ar[r,"q_*"]
& H_n(B''_\bul) \ar[r,"\partial'"]
& H_{n-1}(B_\bul') \ar[r]                
& \cdots
\end{tikzcd}
\]
es conmutativo, ya que el cuadrado izquierdo y el central conmutan
por la funtorialidad de $H_n$.
Esto significa que la construcción de la sucesión exacta larga en $R\ds\Mod$
a partir de una sucesión exacta corta en $\Com(R)$ es funtorial.
\end{theorem}
\begin{proof}
Consideremos el diagrama en grados $n$ y $n-1$:
\[
\begin{tikzcd}
  & 0 \ar[rr]
  && A'_n \ar[ld, "d'_n"'] \ar[dd, "f_n",near end] \ar[rr, "i_n"]
  && A_n \ar[ld, "d_n"'] \ar[dd, "g_n",near end] \ar[rr, "p_n"]
  && A''_n \ar[ld, "d''_n"'] \ar[dd, "h_n",near end] \ar[rr]
  && 0 \\
  0 \ar[rr]
  && A'_{n-1}
  && A_{n-1}
  && A''_{n-1}
  && 0 \\
  & 0 \ar[rr]
  && B'_n \ar[rr, "j_n"',near start] \ar[ld, "\delta'_n",near start]
  && B_n \ar[rr, "q_n"',near start] \ar[ld, "\delta_n",near start]
  && B''_n \ar[rr] \ar[ld, "\delta''_n",near start]
  && 0 \\
  0 \ar[rr]
  && B'_{n-1} \ar[rr, "j_{n-1}"']
  && B_{n-1} \ar[rr, "q_{n-1}"']
  && B''_{n-1} \ar[rr]
  && 0
 \ar[from=2-3,to=2-5,"i_{n-1}",crossing over,near end]
 \ar[from=2-5,to=2-7,"p_{n-1}",crossing over,near end]
 \ar[from=2-7,to=2-9,crossing over]
 \ar[from=2-3,to=4-3, "f_{n-1}",crossing over,near start]
 \ar[from=2-5,to=4-5, "g_{n-1}",crossing over,near start]
 \ar[from=2-7,to=4-7, "h_{n-1}",crossing over,near start]
\end{tikzcd}
\]
Dado $[z'']\in H_n(A''_\bul)$ (donde $z''\in Z_n(A''_\bul)\subseteq A''_n$),
debemos mostrar que $f_*\partial[z'']=\partial'h_*[z'']$.

Por un lado, recordemos que $\partial[z'']$
se define como $[i_{n-1}^{-1}d_na]$, donde $a\in A_n$ cumple $p_na=z''$.
Así, $f_*\partial[z'']=f_*[i_{n-1}^{-1}d_na]=[f_{n-1}i_{n-1}^{-1}d_na]$.

Similarmente, $\partial'h_*[z'']=\partial'[h_nz'']$ se define como
$[j_{n-1}^{-1}\delta_nb]$, donde $b\in B_n$ cumple $q_nb=h_nz''$.
Notando que $q_ng_na=h_np_na=h_nz''$, podemos tomar $b=g_na$.
La situación se muestra en el siguiente diagrama.
\[
\begin{tikzcd}
    &&
    & a \arrow[ld, maps to] \arrow[rr, maps to] \arrow[dd, maps to]
    && z'' \arrow[dd, maps to] \\
    i_{n-1}^{-1}d_na \arrow[rr, maps to] \arrow[dd, maps to]
    &
    & d_na \arrow[dd, maps to] \\
    &&
    & g_na \arrow[ld, maps to] \arrow[rr,maps to]
    && h_nz''
    \\
    j_{n-1}^{-1}\delta_ng_na \arrow[rr, maps to]
    && \delta_ng_na
\end{tikzcd}
\]
donde $j_{n-1}^{-1}\delta_ng_na=f_{n-1}i_{n-1}^{-1}d_na$ porque
$j_{n-1}f_{n-1}=g_{n-1}i_{n-1}$. Por lo tanto
\begin{align*}
    f_*\partial[z'']
    &= [f_{n-1}i_{n-1}^{-1}d_na] \\
    &= [j_{n-1}^{-1}\delta_ng_na] \\
    &= \partial'h_*[z''],
\end{align*}
finalizando la prueba.
\end{proof}

\section{Homotopía entre morfismos de complejos}

\begin{definition}
  Sean los complejos $C_\bul$ y $C_\bul'$, fijemos un entero $p\in\Z$ y sea una familia de morfismos  
  \[\begin{tikzcd}
    (C_n \ar[r, "s_n"] & C_{n+p}')_{n\in\Z}
  \end{tikzcd}\]
  entonces decimos que 
  \begin{tikzcd}
    C_\bul \ar[r, "s_\bul"] & C_\bul'
  \end{tikzcd}
  es un \emph{morfismo de grado $p$}.
\end{definition}

\begin{definition}
  Sean dos morfismos de complejos
  \begin{tikzcd}
    C_\bul \ar[r, "f_\bul", shift left] \ar[r, "g_\bul"', shift right] & C_\bul'
  \end{tikzcd}
  , decimos que son \emph{homotópicos} si existe un morfismo \begin{tikzcd}
    C_\bul \ar[r, "s_\bul"] & C_\bul'
  \end{tikzcd}
  de grado 1, tal que
  \[
    f_n-g_n=d'_{n+1}s_n+s_{n-1}d_n \text{ para todo } n\in\Z
  \]
  y escribimos $f_\bul\sim g_\bul$.
\end{definition}

Veamos un diagrama que ilustra de mejor manera los morfismos involucrados en la definición anterior. 
\[ 
    \begin{tikzcd}
... \ar[r, "..."] & C_{n+1} \ar[r, "d_{n+1}"] \ar[d, shift left] \ar[d, "..."', shift right] \ar[ld, "..."'] & C_n \ar[r, "d_n"] \ar[d, "f_n", shift left] \ar[d, "g_n"', shift right] \ar[ld, "s_n"'] & C_{n-1} \ar[r, "..."] \ar[d, "...", shift left] \ar[d, shift right] \ar[ld, "s_{n-1}"] & ... \ar[ld, "..."] \\
... \ar[r, "..."] & C_{n+1}' \ar[r, "d_{n+1}'"']                                                                      & C_n' \ar[r, "d_n'"']                                                                             & C_{n-1}' \ar[r, "..."]                                                                          & ...                  
    \end{tikzcd}
\]y recordemos que en general el diagrama no es conmutativo, solo se pide que satisfaga $$f_n-g_n=d'_{n+1}s_n+s_{n-1}d_n$$

\begin{definition}
  Decimos que un morfismo de complejos \begin{tikzcd}
C_\bul \ar[r, "f_\bul"] & C_\bul'
\end{tikzcd}
es \emph{nulhomotópico} si 
$$f_\bul\sim 0_\bul$$
\end{definition}

\begin{theorem}\label{teorema:morfismos-homotopos}
  Sean 
  \begin{tikzcd}
    C_\bul \ar[r, "f_\bul", shift left] \ar[r, "g_\bul"', shift right] & C_\bul'
  \end{tikzcd}
  dos morfismos de complejos. Si $f_\bul\sim g_\bul$, entonces los morfismos inducidos en homología son iguales, e.i.
  $$f_*=g_*:H_\bul(C_\bul)\to H_\bul(C_\bul')$$
\end{theorem}
\begin{proof}
  Por la funtorialidad de la homología, tenemos los morfimos $f_*,g_*:H_n(C_\bul)\to H_n(C_\bul')$ para cada $n\in\Z$, entonces sea $[z]\in H_n(C_\bul)$, entonces $z\in Z_n(C_\bul)\subseteq C_n$. Luego, como tenemos que $f_\bul\sim g_\bul$, exsite un morfismo \begin{tikzcd}
    C_\bul \ar[r, "s_\bul"] & C_\bul'
  \end{tikzcd}
  de grado 1, tal que $f_n-g_n=d'_{n+1}s_n+s_{n-1}d_n$, entonces tenemos que
  \begin{eqnarray*}
        f_nz-g_nz &=&  d'_{n+1}s_nz+s_{n-1}d_nz\\
        &=& d'_{n+1}s_nz \in B_n(C_\bul')
    \end{eqnarray*}
    lo que significa que $f_*[z]-g_*[z]=[f_nz]-[g_nz]=[f_nz-g_nz]=[0]$ y por lo tanto  $f_*=g_*$
\end{proof}

\begin{definition}
  Decimos que un complejo $C_\bul$ es \emph{contraíble} si el morfismo identidad es homotópico al morfismo cero, e.i. 
  $$1_{C_\bul}\sim 0$$
\end{definition}

Aunque nos gustaría que fuera cierto, el teorema
\ref{naturalidad} no es válido salvo homotopía, como lo muestra
el siguiente ejemplo.
\begin{example}
    Sea $k$ un campo y consideremos
    la sucesión exacta corta de complejos
    \[
      \begin{tikzcd}
        & 0 \ar[r] & A_\bul \ar[r] & B_\bul \ar[r] & C_\bul \ar[r] & 0 \\
        \text{grado} & & \vdots \ar[d] & \vdots \ar[d] & \vdots \ar[d] & \\
        3 & 0\ar[r] & 0\ar[d]\ar[r] & 0\ar[d]\ar[r] & 0\ar[d]\ar[r] & 0 \\
        2 & 0\ar[r] & 0\ar[d]\ar[r] & 0\ar[d]\ar[r] & 0\ar[d]\ar[r] & 0 \\
        1 & 0\ar[r] & 0\ar[d]\ar[r] & k\ar[d,"1"]\ar[r,"1"] & k\ar[d]\ar[r] & 0 \\
        0 & 0\ar[r] & k\ar[d]\ar[r,"1"] & k\ar[d]\ar[r] & 0\ar[d]\ar[r] & 0 \\
        -1 & 0\ar[r] & 0\ar[d]\ar[r] & 0\ar[d]\ar[r] & 0\ar[d]\ar[r] & 0 \\
        -2 & 0\ar[r] & 0\ar[d]\ar[r] & 0\ar[d]\ar[r] & 0\ar[d]\ar[r] & 0 \\
        & & \vdots & \vdots & \vdots
      \end{tikzcd}
    \]
    Entonces el primer cuadrado del diagrama
    \[
        \begin{tikzcd}
        0 \ar[r]
        & A_\bul \ar[r] \ar[d,"1"]
        & B_\bul \ar[r] \ar[d,"1"]
        & C_\bul \ar[r] \ar[d,"0"]
        & 0
        \\
        0 \ar[r]
        & A_\bul \ar[r]
        & B_\bul \ar[r]
        & C_\bul \ar[r]
        & 0 \homotopy{2-3}{1-4}{45}{1.5}
        \end{tikzcd}
    \]
    es conmutativo, pero el segundo solo es conmutativo salvo homotopia.
    Sin embargo, el diagrama en homología
    \[
    \begin{tikzcd}
        H_1(C_\bul) \ar[r,"\partial"] \ar[d,"0_*"'] & H_0(A_\bul) \ar[d,"1_*"] \\
        H_1(C_\bul) \ar[r,"\partial'"'] & H_0(A_\bul)
    \end{tikzcd}
    \hspace{10mm}
    es
    \hspace{10mm}
    \begin{tikzcd}
        k \ar[r,"1"] \ar[d,"0"'] & k \ar[d,"1"] \\
        k \ar[r,"1"'] & k,
    \end{tikzcd}
    \]
    el cual no conmuta.
\end{example}

\chapter{Funtores derivados}
\todo{SES 4}

\begin{theorem}[Lema de comparación] \label{lema:comparacion}
    Sea $\mathcal{A}$ una categoría abeliana, dado un morfismo  $A\overset{f}{\to}A'$, co consideremos el siguiente diagrama
    \[\begin{tikzcd}
    	{P_A:} \cdots& {P_2} & {P_1} & {P_0} & A & 0 \\
    	{P_{A'}:} \cdots& {P'_2} & {P'_1} & {P'_0} & {A'} & 0
    	\arrow[from=1-1, to=1-2]
    	\arrow[from=1-5, to=1-6]
    	\arrow["{d_2}", from=1-2, to=1-3]
    	\arrow["{d_1}", from=1-3, to=1-4]
    	\arrow["\varepsilon", from=1-4, to=1-5]
    	\arrow["f"', from=1-5, to=2-5]
    	\arrow["f_0"', from=1-4, to=2-4]
    	\arrow[from=2-1, to=2-2]
    	\arrow["{d'_2}"', from=2-2, to=2-3]
    	\arrow["{d'_1}"', from=2-3, to=2-4]
    	\arrow["{\varepsilon'}"', from=2-4, to=2-5]
    	\arrow[from=2-5, to=2-6]
    \end{tikzcd}\]
    Si cada $P_n$ es proyectivo y $P_{A'}$ es un complejo exacto, entonces existe un morfismo de complejos $\tilde{f}:P_A\to P_{A'}$ único salvo homotopía, tal que
\end{theorem}

Y el dual.

\section{Funtores derivados izquierdos}\label{fun-der-izq}
Sea $T:\mathcal{A}\rightarrow \mathcal{C}$ un funtor covariante aditivo entre categorías abelianas donde $\mathcal{A}$ tiene suficientes proyectivos. Construiremos sus \textit{funtores derivados izquierdos} $L_nT:\mathcal{A}\rightarrow \mathcal{C}$. 

Sea $A\in\mathcal{A}$, fijamos una resolución proyectiva;
$$P_A : 
\xymatrix{\ar[r] & P_2 \ar[r]^{d_2} &P_1 \ar[r]^{d_1} & P_0 \ar[r]^{\varepsilon} & A \ar[r] & 0 }
$$
recordemos que las resoluciones viven en la misma clase de homotopía por el lema de comparación. De aquí se deriva el complejo 
$$TP_A: 
\xymatrix{\ar[r] & TP_2 \ar[r] &TP_1 \ar[r] & TP_0 \ar[r] & TA \ar[r] & 0 }
$$
tomamos la homolog\'ia de $TP_A$ y as\'i 
$$(L_nT)A=H_n(TP_A). $$
Sea $f:A\rightarrow A'$ un morfismo y tomemos 
$$
\begin{tikzcd}
\cdots \arrow[r] & P_2 \arrow[r] \arrow[d] & P_1 \arrow[r] \arrow[d] & P_0 \arrow[r] \arrow[d] & A \arrow[r] \arrow[d, "f"] & 0 \\
\cdots \arrow[r] & TP_2 \arrow[r]          & T P_1 \arrow[r]         & TP_0 \arrow[r]          & A' \arrow[r]               & 0
\end{tikzcd}
$$
entonces por el Lema de Comparación, tenemos un morfismo $\tilde{f}:P_A\rightarrow P_A'$ sobre $f$. Por lo tanto, definimos 
$$(L_nT)f:(L_nT)A\rightarrow (L_nT)A' $$
mediante 
$$(L_nT)f=H_n(Tf)=(T\tilde{f})_{n*} ,$$
para cada $n$. Es decir, 
\begin{eqnarray*}
(L_nT)f:H_n(TP_A) & \rightarrow  & H_n(TP_A') \\
z+\Im{d_{n+1}} & \mapsto & (T\tilde{f})(z)+\Im Td_{n+1}'
\end{eqnarray*}
que es, 
$$(L_nT)(f)[z] =[T\tilde{f}_nz] $$
y as\'i tenemos una familia de funtores $(L_nT:\mathcal{A}\rightarrow \mathcal{C})_{n\in\mathbb{Z}}$. 

Veamos ahora que no depende de la homotop\'ia:
Para ello tomemos $h:P_A\rightarrow P_A'$ un morfismo sobre $f$ entonces $h\cong f$ por el Lema de Comparación, como se muestra en el siguiente diagrama
$$
\begin{tikzcd}
\cdots \arrow[r] & P_2 \arrow[r] \arrow[d, "f_2", bend left] \arrow[d, "h_2"', bend right] & P_1 \arrow[r] \arrow[d, "f_1", bend left] \arrow[d, "h_1"', bend right] & P_0 \arrow[r] \arrow[d, "f_0", bend left] \arrow[d, "h_1"', bend right] \arrow[ld, "s_0" description, dotted] & A \arrow[r] \arrow[d, "f", bend left] \arrow[d, "h"', bend right] \arrow[ld, "s_1" description, dotted] & 0 \\
\cdots \arrow[r] & P_2' \arrow[r]                                                          & P_1' \arrow[r]                                                          & P_0' \arrow[r]                                                                                                & A' \arrow[r]                                                                                            & 0
\end{tikzcd}
$$
ya que 
$$f_n-h_n=d_{n+1}'s_n+s_{n-1}d_n $$
y al aplicarle $T$, tenemos
$$Tf_n-Th_n=Td_{n+1}Ts_n+Ts_{n-1}Td_n $$
donde $\tilde{s}_n=Ts_n$ para toda $n$, por lo tanto, 
$$Th\cong T\tilde{f} \qquad \textup{y} \qquad L_nTh=L_nT\tilde{f}.$$
Por otro lado, si $\mathcal{A}=R\textup{-}\Mod$, tomemos $r\in Z(R)$ un elemento en el centro de $R$, entonces
$$\mu_r:M\rightarrow M \qquad \forall M\in R\textup{-}\Mod $$
es un morfismo (multiplicar por $r$). 
Un funtor aditivo, $T:R\textup{-}\Mod \rightarrow R\textup{-}\Mod $ preserva multiplicación si 
$$T(\mu_r):TM\rightarrow TM $$
es también multiplicar por $r$.
\begin{proposition}
Si $T:R\textup{-}\Mod\rightarrow R\textup{-}\Mod$ es un funtor aditivo y preserva multiplicaci\'on entonces 
$$L_nT:R\textup{-}\Mod\rightarrow R\textup{-}\Mod $$
también es aditivo y preserva multiplicación.
\end{proposition}
\begin{proof}
    Tomemos una resolución proyectiva en $A$
    $$P_A = 
\xymatrix{\ar[r] & P_2 \ar[r]^{d_2} &P_1 \ar[r]^{d_1} & P_0 \ar[r]^{\varepsilon} & A \ar[r] & 0 }
$$ 
así, por el Lema de Comparación $\mu_0$ induce los morfismos $\mu_i$ que también son multiplicar por $r$:
$$
\begin{tikzcd}
\cdots \arrow[r] & P_2 \arrow[r] \arrow[d, "\mu_2"'] & P_1 \arrow[r] \arrow[d, "\mu_1"'] & P_0 \arrow[r] \arrow[d, "\mu_0"'] & M \arrow[r] \arrow[d, "\mu_r"'] & 0 \\
\cdots \arrow[r] & P_2 \arrow[r]                     & P_1 \arrow[r]                     & P_0 \arrow[r]                     & M \arrow[r]                     & 0
\end{tikzcd}
$$
Esto quiere decir que 
$$\mu_r\varepsilon = \varepsilon \mu_0 $$
y as\'i $m_0(m)=r_m$ y es también multiplicar por $r$. Finalmente 
$$M\cong P_0/\Ker\varepsilon $$
$$\tilde{\mu}_r(m)= \tilde{\mu}_0(m) $$
y como $T$ preserva multiplicaciones entonces queda demostrado.
\end{proof}

\begin{definition}
Los funtores $(L_nT:\mathcal{A}\rightarrow \mathcal{C})_{n\in \mathbb{Z}}$ son llamados \textbf{funtores derivados izquierdos} de $T$. 
\end{definition}

\begin{proposition}
Sea $T$ un funtor aditivo entonces 
$$(L_nT)M =0 \qquad \forall n <0 \forall M .$$
\end{proposition}
\begin{proof}
    Ya que los $n$-\'esimos t\'erminos de $P_A$ son cero, entonces $(L_nT)A=0$ cuando $n$ es negativo.
\end{proof}
Es por esta raz\'on que los funtores $L_nT$ son llamados funcores derivados \textit{izquierdos}.

\begin{definition}
Decimos que una categoría abeliana $\cal A$
tiene suficientes proyectivos cuando, para todo $A\in\cal A$,
existe un objeto proyectivo $P\in\cal A$ junto con un epimorfismo $P\to A$.
\end{definition}

Si una categoría abeliana $\cal A$ tiene suficientes
proyectivos, entonces podemos escoger una resolución
proyectiva para cada objeto y definir el funtor derivado
de cualquier funtor aditivo con dominio $\cal A$.

Pero ¿qué pasaría si escogiéramos resoluciones proyectivas distintas?
¿cambiarían los valores del funtor derivado?
Ahora veremos que la respuesta es no:
los funtores derivados son naturalmente isomorfos.

\begin{prop}
Sea $\cal A$ una categoría abeliana con suficientes proyectivos.
Supongamos que, para cada objeto $A$ de $\cal C$,
tenemos dos resoluciones proyectivas $P_A$ y $\tilde P_A$.

Entonces para cualquier funtor aditivo $T:\cal A\to\cal B$,
las familias de morfismos de complejos
$(P_A\to\tilde P_A\mid A\in\cal A)$ y $(\tilde P_A\to P_A\mid A\in\cal A)$
dados por el lema de comparación inducen un isomorfismo natural entre
los funtores derivados $L_nT$ y $\tilde L_nT$ con respecto a
las familias de resoluciones proyectivas $P_A$ y $\tilde P_A$.
\end{prop}
\begin{proof}
    Recordemos, de la sección \ref{fun-der-izq},
    que dado un funtor aditivo $T:\cal A\to\cal B$,
    sus funtores derivados izquierdos
    $L_nT,\tilde L_nT:\cal A\to\cal B$
    están dados en objetos como
    \begin{align*}
        L_nT(A) &= H_n(TP_A)
        &
        \text{y}
        &
        &
        \tilde L_nT(A) &= H_n(T\tilde P_A).
    \end{align*}
    y en morfismos como
    \[
    \begin{tikzcd}
        L_nT(A) = H_n(TP_A) \ar[d,"(TF)_*"]
        & {} & {} & 
        \tilde L_nT(A) = H_n(T\tilde P_A) \ar[d,"(T\tilde F)_*"] \\
        L_nT(B) = H_n(TP_B)
        & {} & {} & 
        \tilde L_nT(B) = H_n(T\tilde P_B)
    \end{tikzcd}
    \]
    donde $F:P_A\to P_B$ y $\tilde F:\tilde P_A\to\tilde P_B$
    son los morfismos de complejos inducidos por $f$ gracias al
    lema de comparación (lema \ref{lema:comparacion}).
    En particular, $F$ y $\tilde F$ están sobre $f$. Es decir, los diagramas
    \[
    \begin{tikzcd}
        P_A \ar[d,"F"']\ar[r] & A \ar[d,"f"]
        &{}&
            & \tilde P_A \ar[d,"\tilde F"']\ar[r] & A\ar[d,"f"] \\
        \tilde P_B \ar[r] & B
        &{}&
            & P_B \ar[r] & B
    \end{tikzcd}
    \]
    son conmutativos.
    
    Para ver la equivalencia $L_nT\simeq\tilde L_nT$,
    primero construiremos una transformación natural
    $L_nT\to\tilde L_nT$.
    Usando de nuevo el lema de comparación, para cada objeto $A$
    de $\cal A$ tenemos un morfismo de complejos
    $\iota_A:P_A\to \tilde P_A$ sobre $\id_A$:
    \[
    \begin{tikzcd}
        P_A \ar[d,"\iota_A"']\ar[r] & A \ar[d,"\id_A"] \\
        \tilde P_A \ar[r] & A.
    \end{tikzcd}
    \]
    Ahora, acomodando estos cuadrados en un cubo, obtenemos el siguiente
    diagrama:
    \[
        % https://tikzcd.yichuanshen.de/
        \begin{tikzcd}
            & \tilde P_A \arrow[ld,"\tilde F"']
                \arrow[from=rr,"\iota_A"'] \arrow[dd]
                && P_A \arrow[dd] \arrow[ld,"F"] \\
            \tilde P_B \arrow[dd]
                \arrow[from=rr,crossing over,"\iota_B"',near start]
                && P_B \\
            & A \arrow[ld,"f"'] \arrow[from=rr,"\id_A",near end]
            & & A \arrow[ld,"f"] \\
            B \arrow[from=rr,"\id_B"] & &
            B \ar[from=uu,crossing over]
            \homotopy{1-2}{2-3}{-40}{2}
        \end{tikzcd}
    \]
    Nótese que la base y las caras laterales del cubo conmutan,
    pero la tapa del cubo no necesariamente conmuta.
    Sin embargo, dado que ambos morfismos de complejos
    $\tilde F\iota_A$ y $\iota_BF$
    están sobre $f\id_A=f=\id_Bf$,
    el lema de comparación nos da una homotopía
    $(\tilde F\iota_A)\sim(\iota_BF)$.
    Así, la tapa del cubo ``conmuta salvo homotopía'',
    lo cual está indicado en el diagrama
    de arriba con un $\sim$ dentro del cuadrado correspondiente.
    
    Ahora, al aplicar el funtor $T$, los cuadros conmutativos siguen
    conmutando y la tapa sigue conmutando salvo homotopía:
    \[
        % https://tikzcd.yichuanshen.de/
        \begin{tikzcd}
            & T\tilde P_A \arrow[ld,"T\tilde F"']
                \arrow[from=rr,"T\iota_A"'] \arrow[dd]
                && TP_A \arrow[dd] \arrow[ld,"TF"] \\
            T\tilde P_B \arrow[dd]
                \arrow[from=rr,crossing over,"T\iota_B"',near start]
                && TP_B \\
            & TA \arrow[ld,"Tf"'] \arrow[from=rr,"T\id_A",near end]
            & & TA \arrow[ld,"Tf"] \\
            TB \arrow[from=rr,"T\id_B"] & &
            TB \ar[from=uu,crossing over]
            \homotopy{1-2}{2-3}{-40}{2}
        \end{tikzcd}
    \]
    Como morfismos homótopos inducen el mismo morfismo en homología
    (teorema \ref{teorema:morfismos-homotopos}),
    al tomar homología en la tapa obtenemos un diagrama conmutativo
    \[
        \begin{tikzcd}
            H_n(T\tilde P_A) \ar[d,"(T\tilde F)_*"']
            & H_n(TP_A) \ar[l,"(T\iota_A)_*"'] \ar[d,"(TF)_*"] \\
            H_n(T\tilde P_B) & H_n(TP_B) \ar[l,"(T\iota_B)_*"],
        \end{tikzcd}
    \]
    es decir:
    \[
        \begin{tikzcd}
            \tilde L_nTA \ar[d,"L_nTf"']
            & L_nTA \ar[l,"(T\iota_A)_*"'] \ar[d,"L_nTf"] \\
            \tilde L_nTB & L_nTB \ar[l,"(T\iota_B)_*"]
        \end{tikzcd}
    \]
    lo cual muestra que los morfismos
    $\tau_A=(T\iota_A)_*:L_nTA\to\tilde L_nTA$
    son los componentes de una transformación natural
    $\tau:L_nT\to\tilde L_nT$.
    
    Ahora veamos que $\tau$ es un isomorfismo, para lo cual
    basta ver que cada componente $\tau_A$ es un isomorfismo.
    Recordemos que obtuvimos $\tau_A$ a partir de la familia
    de morfismos de complejos
    $(\iota_A:P_A\to\tilde P_A \mid A\in\cal A)$.
    Similarmente, el lema de comparación nos da otra familia
    de morfismos $(\kappa_A:\tilde P_A\to P_A\mid A\in\cal A)$,
    donde cada $\kappa_A$ está sobre la identidad $\id_A:A\to A$.
    En diagramas:
    \[
    \begin{tikzcd}
        P_A \ar[d,"\iota_A"']\ar[r] & A \ar[d,"\id_A"]
        &{}&
            & \tilde P_A \ar[d,"\kappa_A"']\ar[r] & A\ar[d,"\id_A"] \\
        \tilde P_A \ar[r] & A
        &{}&
            & P_A \ar[r] & A.
    \end{tikzcd}
    \]
    Pegando estos cuadrados verticalmente, obtenemos
    \[
    \begin{tikzcd}
        \tilde P_A \ar[d,"\kappa_A"']\ar[r] \ar[dd,"\iota_A\kappa_A"',bend right=50]
        & A \ar[d,"\id_A"] \ar[dd,"\id_A",bend left=50]
        &{}&
            & P_A \ar[d,"\iota_A"']\ar[r] \ar[dd,"\kappa_A\iota_A"',bend right=50]
            & A\ar[d,"\id_A"] \ar[dd,"\id_A",bend left=50] \\
        P_A \ar[d,"\iota_A"']\ar[r] & A \ar[d,"\id_A"]
        &{}&
            & \tilde P_A \ar[d,"\kappa_A"']\ar[r] & A\ar[d,"\id_A"] \\
        \tilde P_A \ar[r] & A
        &{}&
            & P_A \ar[r] & A.
    \end{tikzcd}
    \]
    Luego, el morfismo de complejos $\iota_A\kappa_A:\tilde P_A\to\tilde P_A$
    está sobre $\id_A$. Por el lema de comparación, obtenemos
    una homotopía $\iota_A\kappa_A\sim\id_{\tilde P_A}$.
    Similarmente, como el morfismo de complejos $\kappa_A\iota_A:P_A\to P_A$
    está sobre $\id_A$, tenemos una homotopía $\kappa_A\iota_A\sim\id_{PA}$.
    \[
    \begin{tikzcd}
        \tilde P_A \ar[dd,"\kappa_A\iota_A"'{name=KI},bend right]
        \ar[dd,"\id_{\tilde P_A}"{name=id1},bend left]
            &{}&
                & P_A \ar[dd,"\iota_A\kappa_A"'{name=IK},bend right]
                \ar[dd,"\id_{P_A}"{name=id2},bend left]
        \\
        {}
        \\
        \tilde P_A &{}& & P_A
        \\
        \ar[from=KI,to=id1,phantom,"\sim"]
        \ar[from=IK,to=id2,phantom,"\sim"]
    \end{tikzcd}
    \]
    Al aplicar $T$, los morfismos siguen siendo homótopos
    \[
    \begin{tikzcd}
        T\tilde P_A \ar[dd,"T(\kappa_A\iota_A)"'{name=KI},bend right]
        \ar[dd,"\id_{T\tilde P_A}"{name=id1},bend left]
            &{}&
                & TP_A \ar[dd,"T(\iota_A\kappa_A)"'{name=IK},bend right]
                \ar[dd,"\id_{P_A}"{name=id2},bend left]
        \\
        {}
        \\
        T\tilde P_A &{}& & TP_A
        \\
        \ar[from=KI,to=id1,phantom,"\sim"]
        \ar[from=IK,to=id2,phantom,"\sim"]
    \end{tikzcd}
    \]
    Finalmente, como morfismos homótopos inducen el mismo morfismo
    en homología, tenemos
    \[
    \begin{tikzcd}
        H_n(T\tilde P_A)
        \ar[dd,"(T(\kappa_A\iota_A))_*=\id_{H_n(TP_A)}"]
            &{}&
                & H_n(TP_A)
        \ar[dd,"(T(\iota_A\kappa_A))_*=\id_{H_n(T\tilde P_A)}"]
        \\
        {}
        \\
        H_n(T\tilde P_A) &{}& & H_n(TP_A).
    \end{tikzcd}
    \]
    Esto es:
    \[
    \begin{tikzcd}
        \tilde L_nTA
        \ar[dd,"(T\kappa_A)_*(T\iota_A)_*=\id_{\tilde L_nTA}"]
            &{}&
                & L_nTA
        \ar[dd,"(T\iota_A)_*(T\kappa_A)_*=\id_{L_nTA}"]
        \\
        {}
        \\
        \tilde L_nTA &{}& & L_nTA.
    \end{tikzcd}
    \]
    Luego, cada $\tau_A=(T\iota_A)_*$ es un isomorfismo, con inverso
    $(T\kappa_A)_*$.
    Se sigue que el funtor $\tau:L_nT\to\tilde L_nT$
    es un isomorfismo natural y, de hecho, su inverso
    $\sigma:\tilde L_nT\to L_nT$ tiene componentes $\sigma_A=(T\kappa_A)_*$.
\end{proof}

\begin{corollary}
    Si $T:\cal A\to\cal B$ es un funtor aditivo
    y $L_nT$ son sus funtores derivados, entonces
    \[
        L_nTP=0
    \]
    para $n\geq 1$ y cualquier proyectivo $P$,
    ya que el complejo $P_\bul$ con $P_0=P$
    y $P_k=0$ para $k\neq 0$
    es una resolución proyectiva de $P$.
\end{corollary}

\begin{definition}[El funtor Tor]
Si $B$ es un $R$-m\'odulo izquierdo, consideramos el funtor aditivo
$$T=- \otimes_R B:\Mod\ds R\to\Z\ds\Mod $$
y definimos 
$$\Tor_n^R(-,B)=L_nT.$$
\end{definition}
De esta manera, si tomamos una resolución proyectiva $(P_A,\epsilon)$
de un $R$-módulo derecho $A$
$$
P_A: 
\begin{tikzcd}
 \cdots \ar[r]  & P_2 \ar[r, "d_2"] & P_1 \ar[r, "d_1"] & P_0  \ar[r, "d_0"] & 0
\end{tikzcd}
$$
y aplicamos $-\otimes_RB$
$$
P_A\otimes_R B: 
\begin{tikzcd}
 \cdots \ar[r]  & P_2 \otimes_R B\ar[r, "d_2\otimes\id_B"] & P_1 \otimes_R B\ar[r, "d_1\otimes\id_B"] & P_0 \otimes_R B \ar[r,"d_0\otimes\id_B"] & 0,
\end{tikzcd}
$$
tenemos
$$\Tor_n^R(A,B)=H_n(P_A\otimes B)
=\frac{\Ker(d_n\otimes\id_B)}{\Im(d_{n+1}\otimes\id_B)}.$$
\begin{remark}
\begin{itemize}
    \item Dado que tomamos un $R$-módulo izquierdo $B$, derivamos el funtor
    $-\otimes_RB:\Mod\ds R\to\Z\ds\Mod$ y obtuvimos los funtores
    \[T
        \Tor^R_n(-,B):\Mod\ds R\to\Z\ds\Mod.
    \]
    \item Si $B$ es un bimódulo ${}_RB_S$, entonces
    $-\otimes_RB$ es un funtor $\Mod\ds R\to\Mod\ds S$.
    Así, sus funtores derivados son
    $$\Tor_n^R(-,B): \Mod\ds R\to\Mod\ds S.$$
    \item Si comenzamos con un $R$-módulo derecho $A$, podemos derivar
    el funtor $A\otimes_R - :R\ds\Mod\to\Z\ds\Mod$ para obtener funtores
    $$\tor_n^R(A,-):R\ds\Mod\to\Z\ds\Mod.$$
    Es decir, para cada $R$-módulo izquierdo $B$, $\tor_n^R(A,B)$ se calcula
    tomando una resolución proyectiva $(R_B,\epsilon)$ de $B$:
    $$
    \dots\to R_3\xto{d_3}R_2\xto{d_2}R_1\xto{d_1}R_0\xto{d_0}0,
    $$
    aplicando $A\otimes_R-$
    $$
    \dots\to
    A\otimes_RR_3 \xto{\id_A\otimes d_3}
    A\otimes_RR_2\xto{\id_A\otimes d_2}
    A\otimes_RR_1\xto{\id_A\otimes d_1}
    A\otimes_RR_0\xto{\id_A\otimes d_0}0,
    $$
    y tomando homología
    \begin{eqnarray*}
    \tor_n^R(A,B) & = & H_n(A\otimes_R R_B) \\
    & = & \frac{\Ker(\id_A\otimes d_n)}{\Im(\id_A\otimes d_{n+1})}
    \end{eqnarray*}
    \end{itemize}
\end{remark}

Por los resultados anteriores,
los módulos $\Tor_n^R(A,B)$ son independientes
de la elección de resolución proyectiva de $A_R$, mientras
que los módulos $\tor_n^R(A,B)$ son independientes de
la elección de resolución proyectiva de ${}_RB$.
Además, $\Tor_n(P,B)=0$ y $\tor(A,Q)=0$
para cualesquiera módulos proyectivos $P_R$, ${}_RQ$.

\begin{corollary}
    Sea $T:\cal A\to\cal B$ un funtor aditivo,
    y sea $(P_A,\epsilon)$ una resolución proyectiva
    de $A\in\cal A$.
    Tomando $K_n=\Ker(P_n\xrightarrow{d_n}P_{n-1})$ para $n\geq 1$
    y $K_0=\Ker(P_0\xrightarrow\epsilon A)$,
    tenemos
    \begin{align*}
        L_{n+1}TA
        &\simeq L_nTK_0 \\
        &\simeq L_{n-1}TK_1 \\
        &\simeq L_{n-2}TK_2 \\
        &\vdots \\
        &\simeq L_1TK_{n-1} \\
        &\simeq L_0TK_n.
    \end{align*}
    En particular, si $\cal A=\Mod\ds R$,
    entonces para cualquier módulo ${}_RB$ se tiene
    \begin{align*}
        \Tor_{n+1}^R(A,B)
        &\simeq \Tor_n^R(K_0,B) \\
        &\simeq \Tor_{n-1}^R(K_1,B) \\
        &\simeq \Tor_{n-2}^R(K_2,B) \\
        &\vdots \\
        &\simeq \Tor_1^R(K_{n-1},B) \\
        &\simeq \Tor_0^R(K_n,B).
    \end{align*}
    o, si $\cal A=R\ds\Mod$, entonces para cualquier
    módulo $B_R$ se tiene
    \begin{align*}
        \tor_{n+1}^R(B,A)
        &\simeq \tor_n^R(B,K_0) \\
        &\simeq \tor_{n-1}^R(B,K_1) \\
        &\simeq \tor_{n-2}^R(B,K_2) \\
        &\vdots \\
        &\simeq \tor_1^R(B,K_{n-1}) \\
        &\simeq \tor_0^R(B,K_n).
    \end{align*}
\end{corollary}
\begin{proof}
Sea $r\geq 0$. Como la sucesión
\[
  \cdots\to P_3\xto{d_3}P_2\xto{d_2}P_1\xto{d_1}P_0\xto\epsilon A\to 0
\]
es exacta, entonces
\[
  \cdots\to P_{r+3}\xto{d_{r+3}}
  P_{r+2}\xto{d_{r+2}}
  P_{r+1}\xto{d_{r+1}}
  K_r \to 0
\]
es exacta, ya que $\Im(d_{r+1})=\Ker(d_r)=K_r$.
Por lo tanto, reetiquetando como $Q_s=P_{r+s+1}$,
$\delta_r=d_{r+s+1}$ y poniendo $\eta=d_{r+1}$,
la sucesión exacta anterior
\[
  \cdots\to Q_2\xto{\delta_{2}}
  Q_1\xto{d_{1}}
  Q_0\xto{\eta}
  K_n \to 0
\]
nos da una resolución proyectiva $(Q_\bul,\eta)$ de $K_r$.
Luego, para cada $s\geq 0$ tenemos
\begin{align*}
    L_sTK_r
    &\simeq H_s(TQ_\bul) \\
    &= \frac{\Ker(\delta_s)}
            {\Im(\delta_{s+1})} \\
    &= \frac{\Ker(d_{r+s+1})}
            {\Im(d_{r+s+2})} \\
    &= H_{r+s+1}(TP_A) \\
    &= L_{r+s+1}TA.
\end{align*}
En particular, dado $n\geq 0$, tenemos
    \begin{align*}
        L_{n+1}TA
        &\simeq L_nTK_0 && \text{con }s=n, r=0 \\
        &\simeq L_{n-1}TK_1 && \text{con }s=n-1, r=1 \\
        &\simeq L_{n-2}TK_2 && \text{con }s=n-2, r=2 \\
        &\vdots \\
        &\simeq L_1TK_{n-1} && \text{con }s=1, r=n-1 \\
        &\simeq L_0TK_n && \text{con }s=0, r=n.
    \end{align*}
\end{proof}

\begin{lemma}
    Dado un diagrama
    \[
      \begin{tikzcd}
            & P'\ar[d,"f'",two heads]
            &
            & P''\ar[d,"f''",two heads] \\
        0 \ar[r]
            & A' \ar[r,"i"]
            & A \ar[r,"q"]
            & A'' \ar[r]
        & 0
      \end{tikzcd}
    \]
    donde las dos flechas verticales son epimorfismos,
    entonces existe un epimorfismo $f:P'\oplus P''\repi A$
    tal que el siguiente diagrama conmuta y tiene renglones exactos.
    \[
      \begin{tikzcd}
        0 \ar[r]
            & P'\ar[d,"f'",two heads] \ar[r]
            & P'\oplus P'' \ar[d,"f",two heads] \ar[r]
            & P''\ar[d,"f''",two heads] \ar[r]
            & 0 \\
        0 \ar[r]
            & A' \ar[r,"i"]
            & A \ar[r,"q"]
            & A'' \ar[r]
        & 0.
      \end{tikzcd}
    \]
\end{lemma}
\begin{proof}
    Sea $P=P'\oplus P''$
    y consideremos los morfismos del coproducto
    $P'\xto{i'} P\xfrom{i''} P''$.
    Como $q$ es epi y $P''$ es proyectivo, $f''$
    se factoriza a través de $q$, digamos $f''=q\sigma$.
    Por la propiedad del coproducto, existe un único morfismo
    $f:P\to A$
    (explícitamente dado como $f(x,y)=if'x+\sigma y$)
    tal que el diagrama siguiente es conmutativo:
    \[
      \begin{tikzcd}
            & P' \ar[r,"i'"]\ar[d,"f'",two heads] \ar[dr]
            & P \ar[d,"f",dotted]
            & P'' \ar[d,"f''",two heads]
            \ar[dl,"\sigma"] \ar[l,"i''"'] \\
        0 \ar[r]
            & A' \ar[r,"i"]
            & A \ar[r,"q"]
            & A'' \ar[r]
        & 0
      \end{tikzcd}
    \]
    Recordemos que las proyecciones $P'\xfrom{q'}P\xto{q''}P''$ y
    las inclusiones $P'\xto{i'}P\xfrom{i''}P''$
    cumplen la ecuación $\id_P=i'q'+i''q''$.
    De la conmutatividad del diagrama anterior, tenemos $qfi'=0$
    y $qfi''=f''$.
    Luego, $0=qfi'q'$ y $f''q''=qfi''q''$.
    Sumando, tenemos
    \begin{align*}
        f''q''
        &= qfi'q'+qfi''q'' \\
        &= qf(i'q'+i''q'') \\
        &= qf,
    \end{align*}
    por lo cual el diagrama
    \[
      \begin{tikzcd}
        0 \ar[r]
            & P' \ar[r,"i'"] \ar[d,"f'",two heads]
            & P \ar[r,"q''"] \ar[d,"f"]
            & P'' \ar[d,"f''",two heads] \ar[r]
            & 0\\
        0 \ar[r]
            & A' \ar[r,"i"]
            & A \ar[r,"q"]
            & A'' \ar[r]
        & 0
      \end{tikzcd}
    \]
    es conmutativo.
    La exactitud del primer renglón se verifica directamente, así que,
    del lema de la serpiente, se sigue que $\epsilon$ es epi.
\end{proof}
\begin{lemma}[Lema de la herradura]
Dada una sucesión exacta corta
\[
    0\to A'\xto i A\xto q A''\to 0
\]
y resoluciones proyectivas $(P'_{A'},\epsilon')$ de $A'$ y
$(P''_{A''},\epsilon'')$ de $A''$,
existe una resolución proyectiva $(P_A,\epsilon)$
y morfismos de complejos tales que
\[
  \begin{tikzcd}
    0 \ar[r]
        & P'_{A'} \ar[r]\ar[d,"\epsilon'"]
        & P_A \ar[r]\ar[d,"\epsilon"]
        & P''_{A''} \ar[r]\ar[d,"\epsilon''"]
    & 0 \\
    0 \ar[r]
        & A' \ar[r,"i"]
        & A \ar[r,"q"]
        & A'' \ar[r]
    & 0
  \end{tikzcd}
\]
es conmutativo y tiene renglones exactos.
\end{lemma}
\begin{proof}
    Por hipótesis, tenemos el diagrama
    \[
      \begin{tikzcd}
            & P'_0 \ar[d,"\epsilon'",two heads]
            &
            & P''_0 \ar[d,"\epsilon''",two heads] \\
        0 \ar[r]
            & A' \ar[r,"i"]
            & A \ar[r,"q"]
            & A'' \ar[r]
        & 0.
      \end{tikzcd}
    \]
    Poniendo $P_0=P'_0\oplus P''_0$ y aplicando el lema anterior, tenemos un epi
    $\epsilon:P_0\to A$ tal que el siguiente diagrama conmuta y tiene renglones
    exactos.
    \[
      \begin{tikzcd}
        0 \ar[r]
            & P'_0 \ar[d,"\epsilon'",two heads] \ar[r]
            & P_0 \ar[d,"\epsilon",two heads] \ar[r]
            & P''_0 \ar[d,"\epsilon''",two heads] \ar[r]
            & 0 \\
        0 \ar[r]
            & A' \ar[r,"i"]
            & A \ar[r,"q"]
            & A'' \ar[r]
        & 0.
      \end{tikzcd}
    \]
    Además, $P_0=P'_0\oplus P''_0$ es proyectivo porque $P'_0$ y $P''_0$ lo son.
    Del lema de la serpiente, tenemos una sucesión exacta corta
    $0\to\ker\epsilon'\to\ker\epsilon\to\ker\epsilon''\to 0$.
    Este es el caso $n=0$ de un argumento por inducción.
    
    Ahora hagamos inducción sobre $n\geq 0$.
    Supongamos que tenemos la sucesión central
    construida hasta el grado $n$ y que, definiendo
    \begin{align*}
        K'_n &=
        \begin{cases}
            \ker \epsilon' & n=0 \\
            \ker d'_n & n\geq 1
        \end{cases}
        &
        K_n &=
        \begin{cases}
            \ker \epsilon & n=0 \\
            \ker d_n & n\geq 1
        \end{cases}
        &
        K''_n &=
        \begin{cases}
            \ker \epsilon'' & n=0 \\
            \ker d''_n & n\geq 1
        \end{cases}
    \end{align*}
    tenemos una sucesión exacta $0\to K'_n\to K_n\to K''_n\to 0$.
    Por exactitud, $d'_{n+1}$ y $d''_{n+1}$ se factorizan a través de sus
    respectivas imágenes $K'_n$ y $K''_n$.
    Así, obtenemos el diagrama con renglones exactos
    \[
      \begin{tikzcd}[column sep=5,row sep=40]
        &&& P'_{n+1} \arrow[ld, two heads]
        &&
        && P''_{n+1} \arrow[ld, two heads]
        \\
        0 \arrow[rr]
        && K'_n \arrow[rr] \arrow[rd, hook]
        && K_n \arrow[rr] \arrow[rd, hook]
        && K''_n \arrow[rr] \arrow[rd, hook]
        && 0 \\
        & 0 \arrow[rr]
        && P'_n \arrow[rr]
        && P_n \arrow[rr]
        && P''_n \arrow[rr]
        && 0.
        \ar[from=1-4,to=3-4,crossing over,"d'_{n+1}",near start]
        \ar[from=1-8,to=3-8,crossing over,"d''_{n+1}",near start]
        \end{tikzcd}
    \]
    Poniendo $P_{n+1}=P'_{n+1}\oplus P''_{n+1}$ y aplicando el lema anterior,
    obtenemos un epimorfismo $P_{n+1}\to\ker d_n$
    que hace que el siguiente diagrama sea conmutativo
    y que el primer renglón también sea exacto:
    \[
      \begin{tikzcd}[column sep=5,row sep=40]
        & 0 \arrow[rr]
        && P'_{n+1} \arrow[rr] \arrow[ld, two heads]
        && P_{n+1} \arrow[rr] \arrow[ld, two heads,dotted]
        && P''_{n+1} \arrow[rr] \arrow[ld, two heads]
        && 0 \\
        0 \arrow[rr]
        && K'_n \arrow[rr] \arrow[rd, hook]
        && K_n \arrow[rr] \arrow[rd, hook]
        && K''_n \arrow[rr] \arrow[rd, hook]
        && 0 \\
        & 0 \arrow[rr]
        && P'_n \arrow[rr]
        && P_n \arrow[rr]
        && P''_n \arrow[rr]
        && 0.
        \ar[from=1-4,to=3-4,crossing over,"d'_{n+1}",near start]
        \ar[from=1-6,to=3-6,crossing over,dashed]
        \ar[from=1-8,to=3-8,crossing over,"d''_{n+1}",near start]
        \end{tikzcd}
    \]
    Definiendo $d_{n+1}$ como la flecha intermitente $P_{n+1}\to P_n$,
    tenemos que $\Im(d_{n+1})=\Ker(d_n)$, así que la sucesión central
    es exacta en grado $n$.
    Ahora definamos
    \begin{align*}
        K'_{n+1}&=\ker d'_{n+1} = \ker(P'_{n+1}\repi K'_n) \\
        K_{n+1}&=\ker d_{n+1} =  \ker(P_{n+1}\repi K_n) \\
        K''_{n+1}&=\ker d''_{n+1} =  \ker(P''_{n+1}\repi K''_n).
    \end{align*}
    Aplicando el lema del 9 al diagrama
    \[
        \begin{tikzcd}
            & 0 \ar[d] & 0 \ar[d] & 0 \ar[d] \\
            0 \ar[r]
            & K'_{n+1} \ar[r]\ar[d]
            & K_{n+1} \ar[r]\ar[d]
            & K''_{n+1} \ar[r]\ar[d]
            & 0 \\
            0 \ar[r]
            & P'_{n+1} \ar[r]\ar[d]
            & P_{n+1} \ar[r]\ar[d]
            & P''_{n+1} \ar[r]\ar[d]
            & 0 \\
            0 \ar[r]
            & K'_{n} \ar[r]\ar[d]
            & K_{n} \ar[r]\ar[d]
            & K''_{n} \ar[r]\ar[d]
            & 0 \\
            & 0 & 0 & 0,
        \end{tikzcd}
    \]
    obtenemos la exactitud del primer renglón, lo cual conluye la inducción.
\end{proof}

\begin{remark}
    En el teorema anterior, para cada $n\geq 0$,
    la sucesión exacta
    \[
        0\to P'_n\to P_n\to P''_n\to 0
    \]
    se escinde como $P_n=P'_n\oplus P''_n$.
    Esto puede parecer especial, pero en realidad no lo es, ya que cualquier
    sucesión exacta $P\to P''\to 0$ con $P''$ proyectivo se escinde.
\end{remark}

\begin{corollary}
    Si en una sucesión exacta corta de módulos
    \[
        0\to A'\to A\to A''\to 0
    \]
    tanto $A'$ como $A''$ son finitamente presentados,
    entonces $A$ también lo es.
\end{corollary}
\begin{proof}
    Por hipótesis, tenemos un diagrama con columnas exactas
    \[
      \begin{tikzcd}
        & \vdots \ar[d] && \vdots \ar[d] \\
        & 0 \ar[d]      && 0 \ar[d] \\
        & R^n\ar[d]     && R^k\ar[d]
        \\
        & R^m\ar[d]     && R^l\ar[d]
        \\
        0 \ar[r]
            & A' \ar[r] \ar[d]
            & A \ar[r]
            & A'' \ar[r] \ar[d] & 0 \\
        & 0 && 0
      \end{tikzcd}
    \]
    Por el lema de la herradura, tenemos un diagrama
    con columnas y renglones exactos
    \[
      \begin{tikzcd}
        & \vdots \ar[d]
          & \vdots \ar[d]
            & \vdots \ar[d] \\
        0 \ar[r]
        & 0 \ar[r] \ar[d]
        & 0 \ar[r] \ar[d]
        & 0 \ar[r] \ar[d]
        & 0 \\
        0 \ar[r]
            & R^n \ar[r]\ar[d]
            & R^{n+k} \ar[r]\ar[d]
            & R^k \ar[r]\ar[d]
        & 0 \\
        0 \ar[r]
            & R^m \ar[r]\ar[d]
            & R^{m+l} \ar[r]\ar[d]
            & R^l \ar[r]\ar[d]
        & 0 \\
        0 \ar[r]
            & A' \ar[r] \ar[d]
            & A \ar[r] \ar[d]
            & A'' \ar[r] \ar[d]
        & 0 \\
        & 0 & 0 & 0
      \end{tikzcd}
    \]
    lo cual muestra que $A$ es finitamente presentado, como se quería.
\end{proof}

\iffalse
\begin{theorem}
    Consideremos un diagrama conmutativo con renglones exactos
    en una categoría abeliana $\cal A$ con suficientes proyectivos.
    \[
    \begin{tikzcd}
        0 \ar[r]
            & A' \ar[r]\ar[d]
            & A \ar[r]\ar[d]
            & A'' \ar[r]\ar[d]
        & 0 \\
        0 \ar[r]
            & C' \ar[r]
            & C \ar[r]
            & C'' \ar[r]
        & 0
    \end{tikzcd}
    \]
    Entonces, para cualquier funtor aditivo $T:\cal A\to\cal C$,
    tenemos un diagrama entre la sucesiones exactas largas de $L_nT$:
    \[
    \begin{tikzcd}[column sep=10]
        \cdots \ar[r] &
        L_{n+1}T(A'') \ar[r]\ar[d]
            & L_nT(A') \ar[r]\ar[d]
            & L_nT(A) \ar[r]\ar[d]
            & L_nT(A'') \ar[r]\ar[d]
        & L_{n-1}T(A')\ar[d]
        \ar[r] & \cdots
        \\
        \cdots \ar[r] &
        L_{n+1}T(C'') \ar[r]
            & L_nT(C') \ar[r]
            & L_nT(C) \ar[r]
            & L_nT(C'') \ar[r]
        & L_{n-1}T(C')
        \ar[r] & \cdots
    \end{tikzcd}
    \]
    En particular, si $\cal A=\Mod\ds R$ y $T=(-\otimes_RB)$,
    tenemos un diagrama entre sucesiones exactas:
    \[
    \begin{tikzcd}[column sep=10]
        \cdots \ar[r] &
        \Tor_{n+1}^R(A'',B) \ar[r]\ar[d]
            & \Tor_n^R(A',B) \ar[r]\ar[d]
            & \Tor_n^R(A,B) \ar[r]\ar[d]
            & \Tor_n^R(A'',B) \ar[r]\ar[d]
        & \Tor_{n-1}^R(A',B)\ar[d]
        \ar[r] & \cdots
        \\
        \cdots \ar[r] &
        \Tor_{n+1}^R(A'',B) \ar[r]
            & \Tor_n^R(C',B) \ar[r]
            & \Tor_n^R(C,B) \ar[r]
            & \Tor_n^R(C'',B) \ar[r]
        & \Tor_{n-1}^R(C',B)
        \ar[r] & \cdots
    \end{tikzcd}
    \]
\end{theorem}
\fi
\begin{theorem}\label{naturalidad-sucesiones-Tor}
    Dado un diagrama de $R$-módulos derechos con renglones exactos
    \[
    \begin{tikzcd}
        0 \ar[r]
            & A' \ar[r]\ar[d]
            & A \ar[r]\ar[d]
            & A'' \ar[r]\ar[d]
        & 0 \\
        0 \ar[r]
            & C' \ar[r]
            & C \ar[r]
            & C'' \ar[r]
        & 0,
    \end{tikzcd}
    \]
    para cada $R$-módulo izquierdo $B$ hay un diagrama conmutativo
    entre las sucesiones exactas:
    \[
    \begin{tikzcd}[column sep=10]
        \cdots \ar[r] &
        \Tor_{n+1}^R(A'',B) \ar[r]\ar[d]
            & \Tor_n^R(A',B) \ar[r]\ar[d]
            & \Tor_n^R(A,B) \ar[r]\ar[d]
            & \Tor_n^R(A'',B) \ar[r]\ar[d]
        & \Tor_{n-1}^R(A',B)\ar[d]
        \ar[r] & \cdots
        \\
        \cdots \ar[r] &
        \Tor_{n+1}^R(A'',B) \ar[r]
            & \Tor_n^R(C',B) \ar[r]
            & \Tor_n^R(C,B) \ar[r]
            & \Tor_n^R(C'',B) \ar[r]
        & \Tor_{n-1}^R(C',B)
        \ar[r] & \cdots
    \end{tikzcd}
    \]
\end{theorem}
\begin{proof}
    Tomemos resoluciones proyectivas $P_{A'}$ y $P_{A''}$.
    Por el lema de la herradura, tenemos una sucesión exacta de resoluciones
    \[
    \begin{tikzcd}
        0 \ar[r]
            & P_{A'} \ar[r]
            & P_A \ar[r]
            & P_{A''} \ar[r]
        & 0
    \end{tikzcd}
    \]
    al aplicar $(-\otimes_RB)$, la sucesión 
    \[
    \begin{tikzcd}
        0 \ar[r]
            & TP_{A'} \ar[r]
            & TP_A \ar[r]
            & TP_{A''} \ar[r]
        & 0
    \end{tikzcd}
    \]
    sigue siendo exacta, pues cada renglón
    \[
        0\to P'_n\to P_n\to P''_n\to 0
    \]
    se escinde por la proyectividad de $P''_n$ y los funtores aditivos
    mandan sucesiones exactas que se escinden en
    sucesiones exactas que se escinden.
    Dice Rotman que por el teorema \ref{naturalidad}, obtenemos el diagrama deseado.
    pero yo (Alfredo) no entiendo cómo puede aplicar ese teorema.
    \todo{help}.
    
    Me sale un teorema parecido tomando una resolución proyectiva de $B$,
    pero eso es con el $\tor$ en vez del $\Tor$.
    Para que me salga con el $\Tor$ también se puede uno tomar
    una resolución de $B$, pero el diagrama tiene que ser
    de módulos izquierdos y $B$ tiene que ser derecho.
\end{proof}

PENDIENTE: SESIÓN 7
\todo{SES 7}

\subsubsection{Axiomas para el Tor.}
En esta secci\'on presentamos los axiomas que caracterizan la sucesi\'on de funtores $\Tor_n^R(-,M)$.
\begin{theorem}[Axiomas para Tor] Sea $(T_n:\Mod\ds R\to\Z\ds\Mod)$ una familia de funtores covariantes aditivos. Si, \begin{enumerate}
    \item para cada sucesi\'on exacta corta $\xymatrix{ 0\ar[r] & A\ar[r] & B \ar[r] & C \ar[r] & 0 }$ de $R$-m\'odulos derechos, existe una sucesi\'on exacta larga con homomorfismo de conexión natural 
    $$
    \xymatrix{\ar[r] & T_{n+1}(C) \ar[r]^{\Delta_{n+1}} & T_n(A)\ar[r] & T_n(B) \ar[r] & T_n(C) \ar[r]^{\Delta_n} & T_{n-1}(A)\ar[r] & },
    $$
    \item $T_0\cong -\otimes_R M$ para alg\'un $R$-m\'odulo $M$,
    \item $T_n(P)=\{0\}$ para todo proyectivo $P$ y para toda $n\geq 1$, 
\end{enumerate}
entonces 
    $$T_n\cong \Tor_n^R(-,M) \qquad \forall n\geq 0.$$
    \label{PropTor}
\end{theorem}
\begin{proof}
    Procedemos por inducci\'on, 
    \begin{itemize}
        \item $n=0$ se sigue por el axioma (ii). 
        \item $n=1$, tenemos lo siguiente, sea $A\in\textup{m\'od}$-$R$, lo cubrimos con un proyectivo, 
        $$
        \xymatrix{0\ar[r] & K\ar[r] & P \ar[r] & A \ar[r] & 0}
        $$
        con $P$ proyectivo, y as\'i por el axioma (i), tenemos el siguiente diagrama con renglones exactos
        $$
        \begin{tikzcd}
{} \arrow[r] & T_1(P) \arrow[r]          & T_1(A) \arrow[r, "\Delta_1"] \arrow[d, "\tau_1A", dashed] & T_0(K) \arrow[r] \arrow[d, "\tau_0K"] \arrow[d] & T_0(P) \arrow[d, "\tau_0P"] \\
{} \arrow[r] & {\Tor_1^R(P,M)} \arrow[r] & {\Tor_1^R(A,M)} \arrow[r, "\delta_1"']                    & {\Tor_0^R(K,M)} \arrow[r]                       & {\Tor_0^R(P,M)}            
\end{tikzcd}
        $$
        donde $\tau_0K$ y $\tau_0P$ son los isomorfismos naturales dados por el axioma (ii). Mientras que por el axioma (iii) tenemos 
        $T_1(P)=\{0\}=\Tor_1^R(P,M)$, as\'i $\Delta_1$ y $\delta_1$ son inyectivos, por lo tanto $T_1(A)\cong \Tor_1^R(A,M)$, con isomorfismo $\tau_1A$, que hace conmutar el diagrama. 
        \item Para $n\geq 1$. Tomemos el siguiente diagrama
        $$
        \begin{tikzcd}
{} \arrow[r] & T_{n+1}(P) \arrow[r]          & T_{n+1}(A) \arrow[r, "\Delta_{n+1}"] \arrow[d, "\tau_nA", dashed] & T_n(K) \arrow[r] \arrow[d, "\tau_n K"] & T_n(P) \arrow[d] \\
{} \arrow[r] & {\Tor_{n+1}^R(P,M)} \arrow[r] & {\Tor_{n+1}^R(A,M)} \arrow[r, "\delta_{n+1}"']                    & {\Tor_n^R(K,M)} \arrow[r]              & {\Tor_n^R(P,M)} 
\end{tikzcd}
        $$
    \end{itemize}
    Por hip\'otesis, 
    $$T_n(K)\cong \Tor_1^R(K,M) \qquad \textup{y} \qquad T_n(P)\cong \Tor_1^R(P,M) $$
    $$ T_{n+1}(P)=\{0\}=\Tor_{n+1}(P) \qquad \textup{y} \qquad T_{n}(P)=\{0\}=\Tor_{n}(P) $$
    As\'i, 
    $$T_{n+1}(A) \cong T_n(K) \qquad \textup{y} \qquad \Tor_n^R(A,M)\cong \Tor_n^R(K,M),$$
    y definimos  
    $$\tau_{n+1}A= \delta_{n+1}^{-1}\tau_nK\Delta_{n+1} $$
    que es un isomorfismo natural.
\end{proof}
Este teorema puede ser generalizado como sigue: 
\begin{corollary}
    Sean $(T_n)_{n\geq 0}$, $(T_n')_{n\geq 0}$ sucesiones de funtores aditivos covariantes entre categorías abelianas $\mathcal{A}$ y $\mathcal{C}$, donde $\mathcal{A}$ tiene suficientes proyectivos. Si 
    \begin{enumerate}
        \item para cada sucesi\'on exacta corta $\xymatrix{ 0\ar[r] & A\ar[r] & B \ar[r] & C \ar[r] & 0 }$ en $\mathcal{A}$, existe una sucesión exacta larga con homomorfismos de conexión naturales,
        \item $T_0\cong T_0'$,
        \item $T_n(P)=\{0\}=T_n'(P)$ para todo $P$ proyectivo y toda $n\geq 1$, 
    \end{enumerate}
    entonces $T_n\cong T_n'$ para toda $n\geq 0$.
\end{corollary}
La demostraci\'on se sigue de manera similar al Teorema \ref{PropTor}. Notemos que en este Corolariom no asumimos que  $(T_n)_{n\geq 0}$ y $(T_n')_{n\geq 0}$ sean sucesiones de funtores derivados. 


\subsubsection{\texorpdfstring{$\partial$}{d}-Funtores homológicos}
\begin{definition}
Si $\mathcal{A}$ y $\mathcal{C}$ son categorías abelianas entonces diremos que una sucesión de funtores aditivos covariantes $(T_n:\mathcal{A}\rightarrow\mathcal{C})_{n\geq 0}$ es un \textbf{$\partial$-funtor homol\'ogico} si para cada sucesi\'on exacta corta
$$\xymatrix{ 0\ar[r] & A\ar[r] & B \ar[r] & C \ar[r] & 0 }$$ en $\mathcal{A}$
se tiene una sucesión exacta larga
$$
\begin{tikzcd}
{} \arrow[r] & T_n(A) \arrow[r]     & T_n(B) \arrow[r]     & T_n(C) \arrow[lld]   &    \\
             & T_{n-1}(A) \arrow[r] & T_{n-1}(B) \arrow[r] & T_{n-1}(C) \arrow[r] & {}
\end{tikzcd}
$$
y adem\'as termina en 
$$\xymatrix{T_0(A)\ar[r] & T_0(B)\ar[r] & T_0(C)\ar[r] &} $$
con morfismos de conexi\'on naturales
$$\partial_n:T_n(C)\rightarrow T_{n-1}(A), $$
es decir, si 
$$\xymatrix{ 0\ar[r] & A'\ar[r] & B '\ar[r] & C' \ar[r] & 0 }$$
entonces el siguiente diagrama conmuta
$$
\begin{tikzcd}
T_n(C) \arrow[d] \arrow[r] & T_{n-1}(A) \arrow[d] \\
T_n(C') \arrow[r]          & T_{n-1}(A')         
\end{tikzcd}
$$
\end{definition}
\begin{definition}
Un \textbf{morfismo de $\partial$-funtores homol\'ogicos } es una familia 
$$\tau:(T_n)_{n\geq 0}\rightarrow (H_n)_{n\geq 0} $$
de transformaciones naturales
$$\tau_n: T_n\rightarrow H_n $$
tales que los siguientes cuadrados son s\'olidos
$$
\begin{tikzcd}
T_n(C) \arrow[d," \tau_{n,C}"] \arrow[r, "\partial"] & T_{n-1}(A) \arrow[d," \tau_{n,A}"] \\
H_n(C) \arrow[r,"\partial"]          & H_{n-1}(A)         
\end{tikzcd}
$$
para cada sucesi\'on exacta corta $(\xymatrix{ 0\ar[r] & A\ar[r] & B \ar[r] & C \ar[r] & 0 })$ en $\mathcal{A}$.
\end{definition}
Finalmente, 
\begin{definition}
Si $F:\mathcal{A} \rightarrow\mathcal{B}$ es un funtor aditivo, un $\partial$-funtor homol\'ogico $(T_n:\mathcal{A}\rightarrow \mathcal{B})_{n\geq 0}$ es una \textbf{extensi\'on homol\'ogica} de $F$ si existe un isomorfismo natural 
$$\tau_0:F\rightarrow T_0 $$
\end{definition}
A manera de ejemplo, el funtor $(\Tor_n(A,-))_{n\geq 0}$ es una extensi\'on homol\'ogica de $A\otimes_R -$.

\begin{theorem}
    Sea $\mathcal{A}$ una categoría abeliana con suficientes proyectivos. 
    \begin{enumerate}
        \item Si $(T_n)$ y $(H_n)$ son $\partial$-funtores homol\'ogicos $\mathcal{A}\rightarrow \mathbb{Z}$-m\'od con $H_n(P)=\{0\}$ para todo $P$ proyectivo para toda $n\geq 1$, y si $\tau_0:T_0\rightarrow H_0$ es una transformaci\'on natural, entonces existe un \'unico morfismo $\tau:(T_n)\rightarrow (H_n)$. 
    
    M\'as a\'un, si $\tau_0$ es un isomorfismo natural, entonces $\tau_n$ es un isomorfismo natural para toda $n\geq 0$.
     \item Si $F: \mathcal{A}\rightarrow \mathbb{Z}$-m\'od  es un funtor exacto izquiero aditivo covariante entonces existe una extensi\'on homol\'ogica  $(H_n)_{n\geq 0}$ de $F$ con $H_n(P)=\{P\}$ para todo $P$ proyectivo y toda $n\geq 1$. 
    \end{enumerate}
\end{theorem}
\begin{proof}
    \begin{enumerate}
        \item Si $$\xymatrix{ 0\ar[r] & A\ar[r] & B \ar[r] & C \ar[r] & 0 } $$ es exacta, construimos $\tau_n:T_n\rightarrow H_n$ por inducci\'on en $n\geq 0$. 
        \begin{itemize}
            \item Para $n=0$, se asume por hip\'otesis.
            \item Para $n\geq 1$ supones que hemos construido 
            $$\tau_{n-1}:T_{n-1}\rightarrow H_{n-1}. $$
            Como hay suficientes proyectivos, entonces para $C$ tenemos la siguiente sucesi\'on exacta corta
             $$\begin{tikzcd}
            &             &             & 0 \arrow[d]           &   \\
            &             &             & K \arrow[d]           &   \\
            &             &             & K \arrow[d]           &   \\
0 \arrow[r] & A \arrow[r] & B \arrow[r] & C \arrow[r] \arrow[d] & 0 \\
            &             &             & 0                     &  
\end{tikzcd} $$
             con $P$ proyectivo. Ahora, obtenemos el siguiente diagrama conmutativo
             $$
             \begin{tikzcd}
                 & T_{n}(C) \arrow[r, "\partial"] \arrow[d, dotted] & T_{n-1}(K) \arrow[r] \arrow[d, "{\tau_{n,K}}"] & T_{n-1}(P) \arrow[d, "{\tau_{n,P}}"] \\
H_n(P) \arrow[r] & H_n(C) \arrow[r, "\rho"']                        & H_{n-1}(K) \arrow[r]                           & H_{n-1}(P)                          
\end{tikzcd}
             $$
             como 
             $$H_n(P)=\{0\}=H_{n-1}(P) \qquad \forall n\geq 1 $$
             entonces 
             $\rho$ es un isomorfismo y podemos definir la composici\'on 
             $$\tau_{n,C}:T_n(C)\rightarrow H_n(C) $$
             como $$\tau_{n,C}= \rho^{-1}\tau_{n-1,K}\partial.$$
             Afirmamos que 
             $$(\tau_{n,C}):T_n(C)\rightarrow H_n(C))_{C\in\mathcal{A}} $$ es natural y est\'a bien definida. Para ello notemos lo siguiente: dado el siguiente diagrama
             $$ 
             \begin{tikzcd}
            &              &             & C \arrow[d, "f"] &   \\
0 \arrow[r] & A' \arrow[r] & A \arrow[r] & A'' \arrow[r]    & 0
\end{tikzcd}
             $$ donde el rengl\'on es exacto, existe un diagrama conmutativo $$
             \begin{tikzcd}
T_n(C) \arrow[r, "T_nf"] \arrow[d, "{\tau_{n,C}}"'] & T_n(A'') \arrow[r, "\partial"]  & T_{n-1}(A') \arrow[d, "{\tau_{n-1,A}}"] \\
H_n(C) \arrow[r, "H_nf"']                           & H_n(A'') \arrow[r, "\partial"'] & H_{n-1}(A')                            
\end{tikzcd}
             $$
             Si nos tomamos una cubierta proyectiva para $C$, tenemos el siguiente diagrama conmutativo
             $$
             \begin{tikzcd}
0 \arrow[r] & K \arrow[r] \arrow[d, "g"'] & P \arrow[r] \arrow[d] & C \arrow[r] \arrow[d, "f"] & 0 \\
0 \arrow[r] & A' \arrow[r]                & A \arrow[r]           & A'' \arrow[r]              & 0
\end{tikzcd}
             $$
             con $P$ proyectivo. Considere el siguiente diagrama 
             $$
             \begin{tikzcd}
T_n(C) \arrow[rrr, "\partial"] \arrow[ddd, "T_nf"'] \arrow[rd, "{\tau_{n,C}}"] &                                                 &                                  & T_{n-1}(K) \arrow[ddd, "{T_{n-1,K}}"] \arrow[ld, "{\tau_{n-1,K}}"] \\
                                                                               & H_n(C) \arrow[r, "\partial"] \arrow[d, "H_nf"'] & H_{n-1}(K) \arrow[d, "H_{n-1}g"] &                                                                    \\
                                                                               & H_n(A'') \arrow[r, "\partial"']                 & H_{n-1}(A')                      &                                                                    \\
T_n(A'') \arrow[rrr, "\partial"']                                              &                                                 &                                  & T_{n-1}(A') \arrow[lu, "{\tau_{n-1,A}}"]                          
\end{tikzcd}
             $$
             notemos que los trapecios de arriba y el lado derecho conmutan por naturalidad de $\tau_n$ y $\tau_{n-1}$, as\'i como el cuadrado del centro conmuta ya que $(H_n)$ es un $\partial$-funtor homol\'ogico. El rect\'angulo restante conmuta por el diagrama anterior. Se sigue que el cuadrado completo conmuta. 
             
             Con esto tenemos la familia $(\tau_n):(T_n)\rightarrow (H_n)$ para toda $n\geq 0$. 
             
             Para ver que es natural, tomemos $X_1,X_2\in\mathcal{A}$ y tomemos  la siguiente sucesi\'on exacta
             $$
             \begin{tikzcd}
0 \arrow[r] & K_i \arrow[r] & P_i \arrow[r] & X_i \arrow[r] & 0
\end{tikzcd}
             $$
            para cada $i=1,2$ y $P_i$ proyectivo. Sea $f:X_1\rightarrow X_2$, definamos
            $$\tau_{n,X_i}:T_n(X_i)\rightarrow H_n(X_i),$$
            para ello tomemos 
            $$
            \begin{tikzcd}
            &               &               & X_1 \arrow[d, "f"] &   \\
0 \arrow[r] & K_2 \arrow[r] & P_2 \arrow[r] & X_2 \arrow[r]      & 0
\end{tikzcd}
            $$
             coomo antes, as\'i por lo ya visto, tenemos el siguiente diagrama conmutativo
             $$ \begin{tikzcd}
T_n(X_1) \arrow[r, "T_nf"] \arrow[d, "{\tau_{n,X_1}}"'] & T_n(X_2) \arrow[r, "\partial"] \arrow[d, "{\tau_{n,X_2}}"] & T_{n-1}(X_2) \arrow[d, "{\tau_{n-1,K_2}}"] \\
H_{n}(X_1) \arrow[r, "H_nf"']                           & H_{n}(X_2) \arrow[r, "\partial"']                          & H_{n-1}(K_2)                              
\end{tikzcd}$$
             lo que nos da la naturalidad. 
             
             Ahora, veamos que no dependen de la resoluci\'on proyectiva. Tomemos $X_1=X_2$ y $f=id_{X_1}$, as\'i tenemos 
             $$ 
             \begin{tikzcd}
0 \arrow[r] & K_1' \arrow[r] \arrow[d] & P_1' \arrow[r] \arrow[d] & X_1 \arrow[r] \arrow[d, "id_{X_1}"] & 0 \\
0 \arrow[r] & K_1 \arrow[r]            & P_1 \arrow[r]            & X_1 \arrow[r]                       & 0
\end{tikzcd}
             $$
             y hacemos lo mismo para 
             $$
             \begin{tikzcd}
            &             &             & C \arrow[d, "id_C"] &   \\
0 \arrow[r] & A \arrow[r] & B \arrow[r] & C \arrow[r]         & 0
\end{tikzcd}
             $$
        \end{itemize}
          \item Se sigue de Proposici\'on \ref{sucesionexactalarga}, del teorema \ref{naturalidad-sucesiones-Tor} y CITAR SESION 7.
    \end{enumerate}
\end{proof}

\section{Funtores derivados derechos (covariantes)}
Como se ha visto anteriormente, los funtores derivados izquierdos satisfacen que si $T$ es exacto derecho, entonces $L_0T$ es naturalmente isomorfo a $T$. Es por esta raz\'on que es natural tomar los funtores derivados izquierdos. 
En esta secci\'on definimos los \textit{funtores derivados derechos} $R^nT$, donde $T:\mathcal{A}\rightarrow \mathcal{C}$ es un funtor covariante aditivo entre categor\'ias abelianas. De esta manera, tenemos el an\'alogo al Corolario REFERENCIA SESION 7, que dice que $R^0\cong T$ cuando $T$ es exacto. Esta construcci\'on es propia para los funtores $Hom$. 
Sea $T:\mathcal{A}\rightarrow \mathcal{C}$ un funtor covariante aditivo. Derivamos a $T$ por la derecha, para eso fijamos $B\in\mathcal{A}$ con suficientes inyectivos y tomamos la resolución inyectiva 
$$
E = \xymatrix{0\ar[r] & B \ar[r]^{\eta} &E^0\ar[r]^{d^0} & E^1 \ar[r]^{d^1} & E^2 \ar[r]^{d^2} & E^3 \ar[r] & }
$$
y le aplicamos $T$ para obtener el complejo $TE^B$, donde $E^B$ es la resoluci\'on inyectiva omitida y tomamos la homolog\'ia: 
$$(R^nT)B=H^{n}(TE^B)=\frac{\Ker Td^n}{\Im Td^{n-1}} .$$
donde $E^n=E_{-n}$ y $d^n=d_{-n}$, de esta manera, lo anterior puede escribirse como
$$(R^nT)B=H_{-n}(TE^B)=\frac{\Ker Td_{-n}}{\Im Td_{-n+1}} .$$
Usando el dual al lema de Comparación, si $f:B\rightarrow B'$ es un homomorfismo, entonces existe un único morfismo, salvo homotopía, $\tilde{f}:E^B\rightarrow E'^B$ sobre $f$, y as\'i en homolog\'ia tenemos
$$(T \tilde{f})_{n*}= (R^nT)f:H^n(TE^B)\rightarrow H^n(TE'^{B} ) $$

\begin{theorem}
    Si $\mathcal{A}$ y $ \mathcal{C}$  son categorías abelianas con $\mathcal{A}$ con suficientes inyectivos. Si $T:\mathcal{A}\rightarrow \mathcal{C}$ es un funtor aditivo covariante, entonces 
    $$R^nT:\mathcal{A} \rightarrow \mathcal{C} $$ también son aditivos para toda $n\in \mathbb{Z}$.
\end{theorem}
La demostración de este y los siguientes resultados son esencialmente duales a  las demostraciones en los funtores derivados izquierdos. Por tanto, serán omitidas. 

\begin{definition}
A los funtores $(R^nT)_{n\in\mathbb{Z}}$ construidos se les llama \textbf{funtores derivados derechos} de $T$.  
\end{definition}

De manera similar a los funtores derivados izquierdos, para los derechos tenemos:
\begin{itemize}
    \item Si $T$ preserva multiplicación, entonces $R^nT$ también.
    \item  $(R^nT)B=0$ para toda $n$ negativa y para toda $B\in\mathcal{A}$.
\end{itemize}
\begin{definition}
Si $T=\Hom_R(A,-)$ entonces definimos $\Ext_r^n(A,-)=R^nT$. Si la resolución proyectiva de $B$ es 
$$
E=\xymatrix{0\ar[r] & B \ar[r]^{\eta} &E^0\ar[r]^{d^0} & E^1 \ar[r]^{d^1} & E^2 \ar[r]^{d^2} & E^3 \ar[r] & },
$$
entonces 
\begin{eqnarray*}
\Ext_R^n(A,B) & = & H^n(\Hom_R(A,E^B))\\
& = & \frac{\Ker d^n_*}{\Im d_*^{n-1}},
\end{eqnarray*}
donde $d_*^n:\Hom_R(A,E^n) \rightarrow \Hom_R(A,E^{n+1})$ se define como $d_*^n:f\mapsto d^nf$.
\end{definition}

\begin{corollary}
    Sea $T:\mathcal{A}\rightarrow \mathcal{C}$ un funtor aditivo covariante, si $E$ es inyectivo, entonces 
    $$(R^nT)E=\{0\} \qquad \forall n\geq 1.$$
    En particular, si $E\in$ mód-$R$ es inyectivo, 
    $$\Ext_R^n(A,E)=\{0\}\qquad  \forall n\geq 1, \forall A\in R-\textup{mód} $$
\end{corollary}

\begin{corollary}
    Sea $\mathcal{A}$ una categoría abeliana con suficientes inyectivos, sea $B\in \mathcal{A}$ y $\mathbf{E}$ una resolución inyectiva para $B$ 
    $$
    \mathbf{E} = \xymatrix{0\ar[r] & B\ar[r]^{\eta} & E^0 \ar[r]^{d^0} & E^1\ar[r]^{d^1} & E^2 \ar[r]^{d^2} & E^3 \ar[r] &  }
    $$
    Definimos,  
    $$V_0=\Im \eta \qquad \textup{y}\qquad V_n=\Im d^{n-1}  $$
    para toda $n\geq 1$. Entonces, 
    $$(R^{n+1}T)B \cong (R^nT)V_0 \cong (R^{n-1}T)V_1\cong \cdots \cong (R^1T)V_{n-1}.$$
    En particular, 
    $$\Ext_R^{n+1}(A,B)\cong\Ext_R^{n}(A,V_0)\cong \cdots \Ext_R^1(A,V_{n-1}). $$
\end{corollary}

\begin{theorem}
    Si  
    $$\xymatrix{0 \ar[r] & B' \ar[r]^{i} & B \ar[r]^{p} & B'' \ar[r] & 0 } $$
    es una sucesión exacta en una categoría abeliana $\mathcal{A}$ con suficientes inyectivos y $T:\mathcal{A}\rightarrow \mathcal{C}$ un funtor covariante aditivo, con $\mathcal{C}$ abeliana, entonces existe una sucesión exacta larga
    $$ 
    \begin{tikzcd}
{} \arrow[r] & (R^nT)B' \arrow[r, "(R^nT)i"]          & (R^nT)B \arrow[r, "(R^nT)p"]          & (R^nT)B'' \arrow[lld, "\partial_n"'] &    \\
             & (R^{n+1}T)B' \arrow[r, "(R^{n+1}T)i"'] & (R^{n+1}T)B \arrow[r, "(R^{n+1}T)p"'] & (R^{n+1}T)B'' \arrow[r]              & {}
\end{tikzcd}
    $$
    que empieza en 
    $$
    \begin{tikzcd}
0 \arrow[r] & (R^0T)B' \arrow[r, "(R^0T)i"] & (R^0T)B \arrow[r, "(R^0T)p"] & (R^0T)B'' \arrow[r] & {}
\end{tikzcd}.
    $$
\end{theorem}

\begin{corollary}
    Si  $T:\mathcal{A}\rightarrow \mathcal{C}$ es un funtor covariante aditivo donde $\mathcal{A}$ tiene suficientes inyectivos, entonces el funtor $R^0T$ es exacto izquierdo.
\end{corollary}

\begin{theorem}\label{teorema:ext0-hom}
    \begin{enumerate}
        \item Si un funtor covariante aditivo  $T:\mathcal{A}\rightarrow \mathcal{C}$ es exacto izquierdo, donde $\mathcal{A}$ y $\mathcal{C}$ son categorías abelianas con $\mathcal{A}$ con suficientes inyectivos, entonces $T\cong R^0T$.
        \item Si $A\in R$-mód, entonces el funtor $\Hom_R(A,-)\cong \Ext_R^0(A,-)$. Así, para todo $B\in R$-mód , existe un isomorfismo 
        $$\Hom_R(A,B)\cong \Ext_R^0(A,B).$$    
        \end{enumerate}
\end{theorem}

\begin{theorem}
  Supongamos que $\cal A$ es una categoría abeliana con
  suficientes inyectivos. Si
  \[
    0\to B'\to B\to B''\to 0
  \]
  es una sucesión exacta en $\cal A$ y $T:\cal A\to\cal C$ es un
  funtor aditivo, entonces hay una sucesión exacta larga
  
  \[
    \begin{tikzcd}
    0 \ar[r]&
    R^0T(B') \ar[r]
    & R^0T(B) \ar[r]
    & R^0T(B'') \ar[dll]
    \\
    & R^1T(B') \ar[r] &\cdots
    \\
    &&\cdots \ar[r]
    &R^{n-1}T(B'')\ar[dll]
    \\
    &R^{n}T(B') \ar[r]
    & R^{n}T(B) \ar[r]
    & R^{n}T(B'')  \ar[dll]
    \\
    &R^{n+1}T(B') \ar[r]
    & R^{n+1}T(B) \ar[r]
    & R^{n+1}T(B'') \ar[dll]
    \\
    &R^{n+2}T(B') \ar[r]
    & \cdots
    \end{tikzcd}
  \]
\end{theorem}

\begin{corollary}
  Si $T:\cal A\to\cal C$ es un funtor aditivo entre categorías
  abelianas y $\cal A$ tiene suficientes inyectivos, entonces
  $R^0T:\cal A\to\cal C$ es exacto izquierdo.
\end{corollary}

\begin{theorem}
  Sean $\cal A,\cal C$ categorías abelianas y supongamos que
  $\cal A$ tiene suficientes inyectivos.
  Si $T:\cal A\to \cal C$ es exacto izquierdo, entonces
  \[
    T\simeq R^0T
  .\]
  En particular, si $A$ es un $R$-módulo, entonces
  \[
    \Hom(A,-)\simeq\Ext^0_R(A,-)
  .\]
\end{theorem}

\begin{corollary}
  Si $A$ es un $R$-módulo y
  \[
    0\to B'\to B\to B''\to 0
  \]
  es una sucesión exacta corta de $R$-módulos, entonces hay una
  sucesión exacta
  \[
    \begin{tikzcd}
    0 \ar[r]&
    \Hom(A,B') \ar[r]
    & \Hom(A,B) \ar[r]
    & \Hom(A,B'') \ar[dll]
    \\
    & \Ext^1(A,B') \ar[r] &\cdots
    \\
    &&\cdots \ar[r]
    &\Ext^{n-1}(A,B'')\ar[dll]
    \\
    &\Ext^{n}(A,B') \ar[r]
    & \Ext^{n}(A,B) \ar[r]
    & \Ext^{n}(A,B'')  \ar[dll]
    \\
    &\Ext^{n+1}(A,B') \ar[r]
    & \Ext^{n+1}(A,B) \ar[r]
    & \Ext^{n+1}(A,B'') \ar[dll]
    \\
    &\Ext^{n+2}(A,B') \ar[r]
    & \cdots
    \end{tikzcd}
  \]
\end{corollary}

\begin{theorem}[Axiomas para el Ext covariante]
  Sea $(F_n:R\ds\Mod\to\Z\ds\Mod)_{n\geq 0}$ una sucesión de
  funtores aditivos covariantes. Si
  \begin{enumerate}[label=(\roman*)]
    \item
    para toda sucesión exacta corta de $R$-módulos
    \[
      0\to B'\to B\to B''\to 0
    \]
    existe una sucesión exacta larga
    \[
      \begin{tikzcd}
      0 \ar[r]&
      F^0(B') \ar[r]
      & F^0(B) \ar[r]
      & F^0(B'') \ar[dll]
      \\
      & F^1T(B') \ar[r] &\cdots
      \\
      &&\cdots \ar[r]
      &F^{n-1}(B'')\ar[dll]
      \\
      &F^{n}(B') \ar[r]
      & F^{n}(B) \ar[r]
      & F^{n}(B'')  \ar[dll]
      \\
      &F^{n+1}(B') \ar[r]
      & F^{n+1}(B) \ar[r]
      & F^{n+1}(B'') \ar[dll]
      \\
      &F^{n+2}(B') \ar[r]
      & \cdots
      \end{tikzcd}
    \]
      
    \item
    existe un $R$-módulo $M$ con
    \[
      F^0 \simeq \Hom(M,-)
    \]
    y
    
    \item
    \[
      F^n(E)=0
    \]
    para todo $R$-módulo inyectivo $E$
    y para cualquier $n\geq 1$
  \end{enumerate}
  entonces
  \[
    F^n = \Ext^n(M,-)
  \]
  para todo $n\geq 0$.
\end{theorem}

\begin{corollary}
  Si
  $(F^n)_{n\geq 0}$ y $(F'{}^n)_{n\geq 0}$ son sucesiones de funtores
  que cumplen las condiciones (i) y (iii) de arriba y
  $F^0\simeq F'{}^0$, entonces $F^n\simeq F{}^n$
  para todo $n\geq 0$.
\end{corollary}

\begin{definition}
  Sean $\cal A$ y $\cal B$ categorías abelianas donde $\cal A$
  tiene suficientes inyectivos.
  Decimos que una sucesión $(T^n:\cal A\to\cal B)_{n\geq 0}$ de
  funtores aditivos es un $\partial$-funtor cohomológico si
  para toda sucesión exacta en $\cal A$
  \[
    0\to A\to B\to C\to 0
  \]
  existe una sucesión exacta larga
  \[
    \begin{tikzcd}
    0 \ar[r]&
    T^0(A) \ar[r]
    & T^0(B) \ar[r]
    & T^0(C) \ar[dll]
    \\
    & T^1T(A) \ar[r] &\cdots
    \\
    &&\cdots \ar[r]
    &T^{n-1}(C)\ar[dll]
    \\
    &T^{n}(A) \ar[r]
    & T^{n}(B) \ar[r]
    & T^{n}(C)  \ar[dll]
    \\
    &T^{n+1}(A) \ar[r]
    & T^{n+1}(B) \ar[r]
    & T^{n+1}(C) \ar[dll]
    \\
    &T^{n+2}(A) \ar[r]
    & \cdots
    \end{tikzcd}
  \]
  con morfismos de conexión $\partial_n:T^n(C)\to T^{n+1}(A)$
  naturales en la sucesión. Es decir, dado un diagrama con
  renglones exactos
  \[
    \begin{tikzcd}
      0 \ar[r]
      & A \ar[r]\ar[d] & B \ar[r]\ar[d] & C \ar[r]\ar[d] & 0 \\
      0 \ar[r] & A' \ar[r] & B' \ar[r] & C' \ar[r] & 0
    \end{tikzcd}
  \]
  el diagrama
  \[
    \begin{tikzcd}
      T^n(C) \ar[r,"\partial_n"] \ar[d] & T^{n+1}(A) \ar[d] \\
      T^n(C') \ar[r,"\partial_n"'] & T^{n+1}(A')
    \end{tikzcd}
  \]
  es conmutativo.
\end{definition}

\begin{definition}
  Si $(T^n)_{n\geq 0}$ y $(F^n)_{n\geq 0}$ son
  $\partial$-funtores cohomológicos, un morfismo de
  $\partial$-funtores cohomológicos
  \[
    \tau:(T^n)_{n\geq 0} \to (F^n)_{n\geq 0}
  \]
  es una sucesión de transformaciones naturales
  \[
    (\tau^n:T^n\to F^n)_{n\geq 0}
  \]
  tales que, para toda sucesión exacta
  \[
    0\to A\to B\to C\to 0
  \]
  los diagramas
  \[
    \begin{tikzcd}
      T^n(C) \ar[r,"\partial_n"] \ar[d,"\tau_C"']
        & T^{n+1}(A) \ar[d,"\tau_A"] \\
      F^n(C) \ar[r,"\partial_n"'] & F^{n+1}(A)
    \end{tikzcd}
  \]
  sean conmutativos para todo $n\geq 0$.
\end{definition}
Dadas dos categorías abelianas $\cal A,\cal B$, si $\cal A$ tiene
suficientes inyectivos, los $\partial$-funtores cohomológicos,
junto con sus morfismos, forman una categoría.

\begin{definition}
  Una extensión cohomológica de un funtor $F:\cal A\to\cal B$ es
  un $\partial$-funtor cohomológico $(T^n:\cal A\to\cal B)_{n\geq
  0}$ junto con un isomorfismo natural
  \[
    F \xto\simeq T^0
  .\]
\end{definition}
\begin{example}
  La sucesión $(\Ext^n(A,-):R\ds\Mod\to Z\ds\Mod)_{n\geq 0}$ es
  un $\partial$-funtor cohomológico equipado con un isomorfismo
  natural
  \[
    \Hom(A,-)\xto\simeq\Ext^0(A,-)
  .\]
  Así, $(\Ext^n(A,-))_{n\geq 0}$ es una extensión cohomológica de
  $\Hom(A,-)$.
\end{example}

\begin{theorem}
  Sean $\cal A$ y $\cal B$ son categorías abelianas, donde $\cal A$
  tiene suficientes inyectivos.
  Si $(T^n:\cal A\to\cal B)_{n\geq 0}$,
  $(F^n:\cal A\to\cal B)_{n\geq 0}$ son $\partial$-funtores
  cohomológicos tales que
  \[
    T^nE=0=F^nE
  \]
  para todo inyectivo $E\in\cal A$, entonces cada transformación
  natural
  \[
    \tau^0:T^0\to F^0
  \]
  induce un morfismo de $\partial$-funtores homológicos
  \[
    \tau:(T^n)_{n\geq 0}\to (F^n)_{n\geq 0}
  .\]
  Más aún, si $\tau^0$ es un isomorfismo, entonces $\tau$ también
  lo es.
\end{theorem}
\begin{theorem}
  Si $F:\cal A\to\cal B$ es un funtor aditivo exacto izquierdo,
  donde $\cal A$ tiene suficientes inyectivos,
  entonces existe una única extension cohomológica
  $(F^n:\cal A\to\cal B)_{n\geq 0}$ de $F$ tal que
  \[
    F^nE=0
  \]
  para todo inyectivo $E\in\cal A$ y toda $n\geq 1$.
\end{theorem}

PENDIENTE SESIÓN 10
\todo{SES 10}

\chapter{Los funtores Tor y Ext}

\section{Propiedades del Tor}

\subsubsection{Resoluciones planas y Tor}
\begin{theorem}
    Los funtores $\Tor_n^R(A,-)$ y $\Tor_n^R(-,B)$ pueden calcularse usando resoluciones planas de cualquiera de las variables; más precisamente, para toda resolución plana $F_A$ y $G_B$ de $A$ y $B$ respectivamente, y para toda $n\geq 0$,
    $$H_n(F_A\otimes_R B)\cong \Tor_n^R(A,B)\cong H_n(A\otimes_R G_B) .$$
\end{theorem}
\begin{proof}
    Sea $F_A\rightarrow A$ una resolución libre para $A$ y $P_B\rightarrow B$ una resolución plana entonces basta probar que 
    \begin{eqnarray*}
    H_n(F_A\otimes_R B) & \cong & \Tor_n^R(A,B)\\
    & \cong & Tor_n{R^{op}}(B,A)\\
    & \cong & H_n(P_B \otimes A)
    \end{eqnarray*}
Entonces, procedemos por inducción sobre $n\geq 0$, tomemos  
$$ 
\xymatrix{ \ar[r] & F_2\ar[r]^{d_2} & F_1\ar[r]^{d_1} & F_0 \ar[r] & 0  }
$$
una resolución plana de $A$ que al tensorizar por $B$ resulta 
$$ 
\xymatrix{ \ar[r] & F_2\otimes B\ar[r]^{d_2\otimes id_B } & F_1\otimes B\ar[r]^{d_1\otimes id_B} & F_0 \otimes B\ar[r] & 0  }
$$
de donde 
\begin{eqnarray*}
    H_n(F_A\otimes B) & = & F_0/ \Im(d_1 \otimes id_B)\\
    & = & \Coker(d_1 \otimes id_B)\\
    & \cong & F_0\otimes B / \Ker(d_0\otimes id_B)\\
    & \cong & A\otimes B
    \end{eqnarray*}
Para $n=1$, tomemos 
$$ 
\xymatrix{ \ar[r] & F_2\ar[r]^{d_2} & F_1\ar[r]^{d_1} & F_0 \ar[r] & 0  }
$$
como antes,  así, podemos factorizar de la siguiente manera, 
$$\begin{tikzcd}
F_2 \arrow[rr, "d_2"] &  & F_1 \arrow[rr, "d_1"] \arrow[rd] &                         & F_0 \\
                      &  &                                  & F_1/\ker d_1 \arrow[ru] &    
\end{tikzcd}$$
y aplicando el producto tensorial tenemos
$$\begin{tikzcd}
F_2\otimes B \arrow[rr, "d_2 \otimes id_B"] &  & F_1 \otimes B\arrow[rr, "d_1\otimes id_B"] \arrow[rd] &                         & F_0\otimes B \\
                      &  &                                  & F_1/\ker d_1 \otimes B \arrow[ru, "i\otimes id_B "] &    
\end{tikzcd}$$
Notemos que, 
$$\Im(d_1\otimes id)=\Im(i\otimes id) $$
de hecho, 
$$\frac{F_1}{\Ker d_1}\otimes B \cong \Im d_1\otimes id $$
ahora, $d_1\otimes id$ es suprayectivo por lo que tenemos el siguiente diagrama
$$
\begin{tikzcd}
F_2\otimes B \arrow[r, "d_1\otimes id_B"] & F_1\otimes B \arrow[r, "d_2\otimes id_B"] \arrow[d, "\alpha"] & F_0\otimes B \\
                                          & F_1\otimes B/ \Im(d_1\otimes id_B)) \arrow[d, "\beta"]        &              \\
                                          & F_1\otimes B / \ker(d_1\otimes id_B)                          &             
\end{tikzcd}
$$
ya que $\Im(d_1\otimes id)\subseteq \Ker(d_1\otimes id)$, $\beta$ es sobreyectivo y así 
$$ 
\begin{tikzcd}
F_1\otimes B / \ker(d_1\otimes id_B)) \arrow[r, "\gamma"] & F_0\otimes B
\end{tikzcd}
$$
tal que $\gamma$ es isomorfa en su imagen; en particular, es inyectiva y definimos $\eta=\gamma\beta$. 
De esta manera podemos observar que 
$$\ker\eta=\frac{\ker(d_1\otimes id)}{\Im(d_2\otimes id)}=H_1(F_A\otimes_R B) $$
y 
$\Im \eta =\Im \gamma$, pero $d_1\otimes id=\gamma\beta\alpha$, entonces
$$\Im{d_1\otimes id}=\Im \gamma$$
ya que $\alpha$ y $\beta$ son sobreyectivos. Por lo tanto, 
$$H_1(F_A\otimes B)\cong \Im\gamma = \Im{d_1\otimes id}=\Im{i\otimes id}. $$
Ahora consideremos,
$$
\begin{tikzcd}
{Tor_1^R(F_0,B)} \arrow[r] & {Tor_1^R(A,B)} \arrow[r] & Y\otimes B \arrow[r] & F_0\otimes B
\end{tikzcd}
$$
como $F$ es plano, entonces 
$$Tor_1^R(F_0,B) =\{0\}, $$
sustituyendo en la sucesión anterior tenemos, 
$$
\begin{tikzcd}
{0} \arrow[r] & {Tor_1^R(A,B)} \arrow[r] & Y\otimes B \arrow[r] & F_0\otimes B
\end{tikzcd}
$$
como la sucesión es exacta, entonces
\begin{eqnarray*}
\Tor_1^R(A,B) & \cong & \Ker(i\otimes id) \\
& \cong & H_1(F_A\otimes B)\\
& \cong & \Ker(\frac{F_1\otimes B}{\Im{d_1\otimes id}}\rightarrow F_0\otimes B)\\
& \cong & H_1(F_A \otimes B)
\end{eqnarray*}
y de aquí llegamos a que 
$$\Tor_1^R(A,B) \cong H_1(F_A\otimes B).$$
Para $n\geq 1$, tomemos
$$
\begin{tikzcd}
{Tor_{n+1}^R(F_0,B)} \arrow[r] & {Tor_{n+1}^R(A,B)} \arrow[r] & {\Tor_{n+1}(Y,B)} \arrow[r] & {Tor_0(F_0,B)} \arrow[r] & {}
\end{tikzcd}
$$
Por hipótesis de inducción, tenemos 
\begin{eqnarray*}
\Tor_n^R(F_0,B) & = & 0, \\
\Tor_{n+1}^R(F_0,B)& = & 0,
\end{eqnarray*}
por lo tanto, 
$$\Tor_{n+1}(A,B) \cong \Tor_{n+1}(Y,B).$$
Si tomamos una resolución plana $F'$ para $Y$, entonces por hipótesis $H_n(F'\otimes B)\cong \Tor_n(Y,B)$, pero $H_n(F'\otimes B)=H_{n+1}(F_A\otimes B)$.
\end{proof}

\subsubsection{Sumas directas y Tor}

\begin{proposition}
Si $(B_k)_{k\in K}$ es una familia de $R$-módulos, entonces existe un isomorfismo natural 
$$\Tor_n^R(A, \oplus_{k\in K} B_k)\cong \oplus_{k\in K}\Tor_n^R (A,B_k) \qquad \forall n\geq 0.$$
\end{proposition}
\begin{proof}
    Para $n=0$ sabemos que $\Tor_0^R(A,-)\cong A\otimes_R -$. 
    Ahora para el paso inductivo, para cada $i$ tenemos 
    $$ 
    \xymatrix{0\ar[r] & N_i \ar[r] & P_i \ar[r] & B_i \ar[r] & 0}
    $$
    con $\{B_i\}_{i\in \mathbb{N}}\subseteq R$-mód. Es decir, tenemos proyectivos que cubren a cada $B_i$. Además, tenemos la siguiente sucesión exacta 
     $$ 
    \xymatrix{0\ar[r] &  \oplus_i N_i \ar[r] & \oplus_iP_i \ar[r] & \oplus_iB_i \ar[r] & 0}
    $$
    y dado que la suma de proyectivos es proyectiva, entonces
     $$ 
    \xymatrix{\Tor_1(A,\oplus_iP_i) \ar[r] &  \Tor_1(A,\oplus_iB_i) \ar[r] & A\otimes \oplus_iN_i \ar[r] &A \otimes \oplus_iP_i}
    $$
    Por otro lado, tomemos,
    $$ 
    \xymatrix{\oplus_i \Tor_1(A,P_i) \ar[r] &  \oplus_i\Tor_1(A,B_i) \ar[r] &\oplus_i( A\otimes N_i) \ar[r] & \oplus_i(A \otimes P_i)}
    $$
    de donde podemos notar que
    \begin{eqnarray*}
     A\otimes \oplus_iN_i & \cong & \oplus_i( A\otimes N_i) \\
      A\otimes \oplus_iP_i & \cong & \oplus_i( A\otimes P_i)
    \end{eqnarray*}
    Sin embargo,  $\Tor_1(A,\oplus_i P_i)=0$ por ser $\oplus P_i$ proyectivo, de la misma manera $\oplus\Tor_1(A,P_i)=0 $. 
    Por lo tanto, del lema del 5to tenemos, 
    $$\Tor_1(A,\oplus B_i)\cong \oplus \Tor_1(A,B_i). $$
\end{proof}
\begin{example}
Ahora tomemos $B\in \mathbb{Z}$-mód, $n\in\mathbb{Z}$ y 
$$
\xymatrix{0\ar[r]& \mathbb{Z} \ar[r]^{\mu_n} & \mathbb{Z} \ar[r] & \mathbb{Z}/n\mathbb{Z} \ar[r] & 0 }
$$
donde $\mu_n$ es multiplicar por $n$. Aplicando el producto tensorial por $B$, tenemos
$$
\xymatrix{0\ar[r]& \mathbb{Z}\otimes B \ar[r]^{\mu_n\otimes id_B} & \mathbb{Z}\otimes B \ar[r] & \mathbb{Z}/n\mathbb{Z}\otimes B \ar[r] & 0 }
$$
del cual resulta la siguiente sucesión,
$$
\xymatrix{0\ar[r]& \Tor_1(\mathbb{Z},B) \ar[r] & \Tor_1(\mathbb{Z}/n\mathbb{Z},B) \ar[r] & \mathbb{Z}\otimes B \ar[r]^{\mu_n\otimes id_B} & \mathbb{Z}\otimes B}
$$
así $\Tor_1(\mathbb{Z},B)=0$ y dado que son grupos abelianos $\mathbb{Z}\otimes B=B$, por lo tanto $\mu_n\otimes id_B=id_B$.

Definimos 
$$B[n]=\{b\in B: nb=0 \}, $$
así de lo anterior resulta la siguiente sucesión 
$$ 
\xymatrix{0\ar[r]& B[n] \ar[r] & B \ar[r] & B}
$$
y 
$$ 
\xymatrix{0\ar[r] & \Tor_1(\mathbb{Z}/n \Z), B \ar[r] & \Z \otimes B \ar[r] & \Z \otimes B}
$$
por lo tanto, 
$$B[n]\cong \Tor_1(\Z / n\Z , B) $$.
\end{example}
Ahora, si $A$ y $B$ son $\Z$-módulos finitamente generados, entonces 
\begin{eqnarray*}
A & \cong & C_i \\
B & \cong & C_j'
\end{eqnarray*}
para algunos $C_i$ y $C_j'$. Por lo tanto, 
\begin{eqnarray*}
\tor_1(A,B) & \cong & \Tor(\oplus_i C_i,\oplus_j C_j') \\
& \cong & \oplus_{i,j} \Tor_1(C_i,C_j')
\end{eqnarray*}
ya que $C_i$ y $C_j'$ son cíclicos, entonces son cícilos finitos y aplicando el ejemplo anterior, tenemos $\Tor_1(\Z/ n\Z, \Z/m\Z)$ para alguna componente, de esta manera ,
\begin{eqnarray*}
\Tor_1(\Z/ n\Z, \Z/m\Z) & = & \{x\in \Z / m\Z: nx=0\} \\
& = & \{\overline{x}\in \Z / n\Z: nx\in m\Z\}\\
& = & [n:m\Z] \\
& = & \Z / m \Z[d] 
\end{eqnarray*}
donde $d=mcd(m,n)$.

\subsubsection{Límites dirigidos y Tor}

Tomemos un $R$-módulo derecho $A$.
Si $(B_i,\phi_j^i)$ es un sistema dirigido de $R$-módulos izquierdos,
entonces los morfismos $B_i\to\colim_iB_i$ inducen morfismos
$A\otimes B_i\to A\otimes\colim_iB_i$.
Así, tenemos un único morfismo
\[
  A\otimes_R\colim_iB_i \to \colim_i(A\otimes_RB_i)
\]
que hace conmutar el diagrama
\[
  \begin{tikzcd}
    A\otimes B_i \ar[dr] \ar[dd] \\
    & A\otimes\colim_iB_i \ar[dl] \\
     \colim_iA\otimes B_i
  \end{tikzcd}
\]
para todo $i$ (dado en los generadores como
$a\otimes [b]\mapsto [a\otimes b]$).
Este morfismo es, de hecho, un isomorfismo.
Expresamos esto diciendo que el funtor $A\otimes_R-$ preserva
colímites dirigidos.
Ahora probaremos que sus funtores derivados $\Tor_n^R(A,-)$ tienen la
misma propiedad.
\begin{theorem}
  Sean $A$ un $R$-módulo derecho y $(I,\leq)$ un conjunto dirigido.
  Entonces las transformaciones naturales
  \[
    t_n:\Tor_n(A,\colim-)\to\colim\Tor_n(A,-)
  \]
  entre los funtores $(\Mod_R)^{(I,\leq)}\to\Mod_\Z$ son, de hecho,
  isomorfismos naturales.
  Los componentes de estas transformaciones naturales están dadas,
  para cualquier sistema dirigido $(B_i,\phi_j^i)$
  de $R$-módulos izquierdos, como el único morfismo
  \[
    \Tor_{n}^{R}(A,\colim_iB_i) \to \colim_i\Tor_{n}^{R}(A,B_i)
  \]
  que hace conmutar el diagrama
  \[
    \begin{tikzcd}
      \Tor_{n}^{R}(A,B_i) \ar[dr] \ar[dd] \\
      & \Tor_{n}^{R}(A,\colim_iB_i) \ar[dl] \\
       \colim_i\Tor_{n}^{R}(A,B_i).
    \end{tikzcd}
  \]
\end{theorem}
\begin{proof}
  Procederemos por inducción sobre $n\geq 0$.
  El caso base $n=0$ es el del producto tensorial,
  el cual se trató antes de enunciar el teorema.

  Ahora tomemos $n\geq 1$ y supongamos que
  \[
    t_{n-1}:\Tor_{n-1}^{R}(A,\colim-)\to\colim\Tor_{n-1}^{R}(A,-)
  \]
  es un isomorfismo.

  Para cualquier sistema dirigido $(B_i,\phi_i^i)$, 
  consideremos un nuevo sistema dirigido $(P_i,\psi_{i}^{j})$,
  donde $P_i$ son los módulos libres
  $P_i=R^{(B_i)}$ para cada $i\in I$, mientras que los
  $\psi_{i}^{j}:P_i\to P_j$ son los morfismos inducidos por las
  funciones $\phi_{i}^{j}:B_i\to B_j$. Así, siempre que
  $i\leq j\leq k$, tenemos $\phi_i^j\phi_k^i=\phi_i^k$,
  por lo cual $\psi_i^j\psi_k^j=\psi_i^k$.
  Además, para cada $i$ tenemos un morfismo suprayectivo
  natural $P_i\to B_i$, lo cual nos da un morfismo de sistemas
  dirigidos $(P_i)\to(B_i)$. Los núcleos $K_i=\ker(P_i\to
  B_i)$ forman otro sistema dirigido con un morfismo a $(P_i)$ de tal
  modo que obtenemos una sucesión exacta de sistemas dirigidos
  indicados por $I$:
  \[
    0\to(K_i)\to(P_i)\to(B_i)\to 0
  .\]
  Dado que el funtor $\colim$ es exacto, obtenemos una
  sucesión exacta
  \[
    0\to\colim_iK_i\to\colim_iP_i\to\colim_iB_i\to 0
  .\]
  Aplicando $A\otimes_R-$, obtenemos una sucesión exacta larga
  \[
    \begin{tikzcd}
    &\cdots \ar[r]
    & \Tor_3(A,\colim_iB_i) \ar[dll]
    \\
    \Tor_2(A,\colim_iK_i) \ar[r]
    & \Tor_2(A,\colim_iP_i) \ar[r]
    & \Tor_2(A,\colim_iB_i) \ar[dll]
    \\
    \Tor_1(A,\colim_iK_i) \ar[r]
    & \Tor_1(A,\colim_iP_i) \ar[r]
    & \Tor_1(A,\colim_iB_i) \ar[dll]
    \\
    A\otimes\colim_iK_i \ar[r]
    & A\otimes\colim_iP_i \ar[r]
    & A\otimes\colim_iB_i \ar[r]
    & 0.
    \end{tikzcd}
  \]
  Por otro lado, al aplicar $A\otimes-$ a cada sucesión
  \[
    0\to K_i\to P_i\to B_i\to 0
  \]
  obtenemos una sucesión exacta larga
  \[
    \begin{tikzcd}
    &\cdots \ar[r]
    & \Tor_3(A,B_i) \ar[dll]
    \\
    \Tor_2(A,K_i) \ar[r]
    & \Tor_2(A,P_i) \ar[r]
    & \Tor_2(A,B_i) \ar[dll]
    \\
    \Tor_1(A,K_i) \ar[r]
    & \Tor_1(A,P_i) \ar[r]
    & \Tor_1(A,B_i) \ar[dll]
    \\
    A\otimes K_i \ar[r]
    & A\otimes P_i \ar[r]
    & A\otimes B_i \ar[r]
    & 0
    \end{tikzcd}
  \]
  y, por la funtorialidad de $\Tor_n(A,-)$, obtenemos
  morfismos que convierten a estas sucesiones exactas largas en
  una sucesión exacta larga de sistemas dirigidos indicados por
  $I$. De nuevo por la exactitud de $\colim$, tenemos 
  \[
    \begin{tikzcd}
    &\cdots \ar[r]
    & \colim_i\Tor_3(A,B_i) \ar[dll]
    \\
    \colim_i\Tor_2(A,K_i) \ar[r]
    & \colim_i\Tor_2(A,P_i) \ar[r]
    & \colim_i\Tor_2(A,B_i) \ar[dll]
    \\
    \colim_i\Tor_1(A,K_i) \ar[r]
    & \colim_i\Tor_1(A,P_i) \ar[r]
    & \colim_i\Tor_1(A,B_i) \ar[dll]
    \\
    \colim_i A\otimes K_i \ar[r]
    & \colim_i A\otimes P_i \ar[r]
    & \colim_i A\otimes B_i \ar[r]
    & 0.
    \end{tikzcd}
  \]
  En este punto es donde entran los morfismos
  $\Tor_{n}^{R}(A,\colim_iB_i)\to\colim_i\Tor_{n}^{R}(A,B_i)$
  que aparecen en el enunciado del teorema.
  Éstos nos dan un morfismo entre estas dos sucesiones exactas
  \[
    \begin{tikzcd}[column sep=10]
      \dots \ar[r]
      & \Tor_1(A,\colim_iP_i) \ar[r] \ar[d,"t_{1,P}"]
      & \Tor_1(A,\colim_iB_i) \ar[r] \ar[d,"t_{1,B}"]
      &A\otimes\colim_iK_i \ar[r] \ar[d,"\simeq"]
      & A\otimes\colim_iP_i \ar[r] \ar[d,"\simeq"]
      & A\otimes\colim_iB_i \ar[r] \ar[d,"\simeq"]
      & 0
      \\
      \dots \ar[r]
      & \colim_i\Tor_1(A,P_i) \ar[r]
      & \colim_i\Tor_1(A,B_i) \ar[r]
      & \colim_i A\otimes K_i \ar[r]
      & \colim_i A\otimes P_i \ar[r]
      & \colim_i A\otimes B_i \ar[r]
      & 0.
    \end{tikzcd}
  \]
  Dado que cada $P_i$ es plano (porque es libre) entonces
  $\colim_iP_i$ es plano (porque el límite dirigido de módulos
  planos es plano). Así, tenemos
  \[
    \forall n\geq 1,\hspace{5mm} \Tor_n(-,\colim_iP_i) = \colim_i\Tor_n(-,P_i) = 0
  .\]
  Además, por hipótesis, $t_{n,K},t_{n,P}$ son
  isomorfismos. Luego, la exactitud del diagrama
  \[
    \begin{tikzcd}[column sep=10]
      \dots \ar[r]
      & \cancelto{0}{\Tor_{n}(A,\colim_iP_i)} \ar[r] \ar[d,"t_{n,P}"]
      & \Tor_{n}(A,\colim_iB_i) \ar[r] \ar[d,"t_{n,B}"]
      & \Tor_{n-1}(A,\colim_iK_i) \ar[r] \ar[d,"t_{n-1,K}","\simeq"']
      & \Tor_{n-1}(A,\colim_iP_i) \ar[d,"t_{n-1,K}","\simeq"']
      \\
      \dots \ar[r]
      & \cancelto{0}{\colim_i\Tor_{n}(A,P_i)} \ar[r]
      & \colim_i\Tor_{n}(A,B_i) \ar[r]
      & \colim_i\Tor_{n-1}(A,K_i) \ar[r]
      & \colim_i\Tor_{n-1}(A,P_i).
    \end{tikzcd}
  \]
  implica que
  $t_{n,B}:\Tor_1(A,\colim_iB_i)\to\colim_i\Tor_1(A,B_i)$
  es un isomorfismo. Como el sistema dirigido $(B_i,\phi_{i}^{j})$ era
  arbitrario, tenemos que
  \[
    t_n:\Tor_{1}^{R}(A,\colim-)\to\colim\Tor_{n}^{R}(A,-)
  \]
  es un isomorfismo. Esto concluye el paso de inducción.
\end{proof}

\section{El teorema de Chase}

Ahora queremos demostrar el teorema de Chase, que relaciona
algunas propiedades de un anillo $R$ con propiedades de
su categoría de módulos.

Antes, necesitamos dos lemas.

Sean $M$ un $R$-módulo derecho y $(L_i\mid i\in X)$ una familia
de $R$-módulos izquierdos.
Notemos que, para cada $i\in X$, la función
$M\times\prod_i L_i\to M\otimes L_i$ dada por $(m,l)\mapsto
m\otimes l_i$ es bilineal, por lo cual induce un morfismo de
$\Z$-módulos $\phi_i:M\otimes\prod_iL_i\to M\otimes L_i$
dado en $x\otimes l$ como $\phi(x\otimes l)=x\otimes l_i$.
Por la propiedad universal del producto, los $\phi_i$ inducen un morfismo
$\phi:M\otimes\prod_iL_i\to\prod_i(M\otimes L_i)$
dado en los generadores como $\phi(x\otimes l)=(x\otimes l_i \mid i\in X)$.
Notemos que $\phi$ es un iso en el caso $M=R^n$,
pues tenemos un diagrama conmutativo
\[
  \begin{tikzcd}
    R^n\otimes\prod_iL_i\ar[r,"\phi"] \ar[d,"\simeq"']
      &\prod_i(R^n\otimes L_i) \ar[d,"\simeq"] \\
    \left(\prod_iL_i\right)^n \ar[r,"\simeq"] & \prod_iL_i^n
  \end{tikzcd}
\]
en el cual la flecha inferior está dada por
$(q_1,\dots,q_n)\mapsto((q_{1i},\dots,q_{ni})\mid i\in X)$,
donde $q_{ji}$ son los componentes de $q_j$; es decir:
$q_j=(q_{ji}\mid i\in X)\in\prod_iL_i$ para $j=1,\dots,n$.

Sin embargo, esto no siempre ocurre.
Los dos lemas siguientes relacionan propiedades de finitud
de $M$ con propiedades de $\phi$.

\begin{lemma}
Para un $R$-módulo derecho $M$ son equivalentes
  \begin{enumerate}
    \item
    $M$ es finitamente generado.
    \item
    Para toda familia $(L_i\mid i\in X)$ de $R$-módulos izquierdos,
    el morfismo
    \[
        \phi:M\otimes\prod_iL_i\to\prod_i(M\otimes L_i)
    \]
    es suprayectivo.
    \item
    Para todo conjunto $X$, el morfismo
    \[
        M\otimes R^X\to M^X
    \]
    es suprayectivo.
  \end{enumerate}
\end{lemma}
\begin{proof}
\begin{itemize}
  \item 1$\implies$2.
  Por hipótesis, tenemos una sucesión exacta $R^m\to M\to 0$.
  Luego, obtenemos un cuadrado
  \[
    \begin{tikzcd}
      R^n\otimes\prod_iL_i\ar[r,"\simeq"] \ar[d]
        &\prod_i(R^n\otimes L_i) \ar[d] \\
      M\otimes\prod_iL_i \ar[r,"\phi"] \ar[d]
      & \prod_i(M\otimes L_i) \ar[d] \\
      0 & 0
    \end{tikzcd}
  \]
  donde las columnas son exactas porque el producto tensorial
  y el producto directo son exactos derechos.
  Como la flecha superior es iso,
  concluimos que la flecha inferior también es epi.
  
  \item
  2$\implies$3. Tómese $L_i=R$ para todo $i\in X$.
  
  \item 
  3$\implies$1.
  Tómese $X=M$, el conjunto subyacente de $M$.
  Por hipótesis, el morfismo de $\Z$-módulos
  \[
    \psi:M\otimes R^M \to M^M
  \]
  dado en los generadores como $\psi(m\otimes t)=(mt_m \mid m\in M)$,
  es suprayectivo, así que para el elemento $u=(m \mid m\in M)\in M^M$
  existe una suma finita
  $\sum_{i=1}^n m_i\otimes t_i\in M\otimes R^M$
  tal que $\psi(\sum_{i=1}^n m_i\otimes t_i)=u$.
  Esto es,
  \[
    \left(\sum_{i=1}^n m_it_{im}\mid m\in M\right)
    = (m \mid m\in M)
  ,\]
  de modo que, para cada $m\in M$, tenemos
  \[
    m = \sum_{i=1}^n m_it_{im}
  .\]
  Luego, los $m_1,\dots,m_n\in M$ generan a $M$.
\end{itemize}
\end{proof}

\begin{lemma}
Para un $R$-módulo derecho $M$ son equivalentes
  \begin{enumerate}
    \item
    $M$ es finitamente presentado.
    \item
    Para toda familia $(L_i\mid i\in X)$ de $R$-módulos izquierdos,
    el morfismo
    \[
        \phi:M\otimes\prod_iL_i\to\prod_i(M\otimes L_i)
    \]
    es un iso.
    \item
    Para todo conjunto $X$, el morfismo
    \[
        M\otimes R^X\to M^X
    \]
    es un iso.
  \end{enumerate}
\end{lemma}
\begin{proof}
  \begin{itemize}
    \item
    1$\implies$2.
    Por hipótesis, $M$ es finitamente presentado,
    así que tenemos una sucesión exacta
    \[
      R^n\to R^m\to M\to 0
    .\]
    Como el producto tensorial y el producto directo son exactos,
    el diagrama
    \[
    \begin{tikzcd}
      R^n\otimes\prod_iL_i\ar[r,"\simeq"] \ar[d]
        &\prod_i(R^n\otimes L_i) \ar[d] \\
      R^m\otimes\prod_iL_i\ar[r,"\simeq"] \ar[d]
        &\prod_i(R^m\otimes L_i) \ar[d] \\
      M\otimes\prod_iL_i \ar[r,"\phi"] \ar[d]
      & \prod_i(M\otimes L_i) \ar[d] \\
      0 & 0
    \end{tikzcd}
    \]
    tiene columnas exactas. Como los dos morfismos superiores son
    isos, $\phi$ también lo es.
    
    \item
    2$\implies$3.
    Tómese $L_i=R$ para todo $i\in X$.
    
    \item
    3$\implies$1.
    Por el lema anterior, la hipótesis implica que
    $M$ es finitamente generado, así que
    tenemos una sucesión exacta
    \[
      0\to K\to R^m\to M\to 0
    .\]
    Veamos que $K$ es finitamente generado,
    para lo cual usaremos el lema anterior.
    Sea $X$ un conjunto.
    Como el producto directo es exacto
    y el producto tensorial es exacto derecho, el diagrama
    \[
    \begin{tikzcd}
      & 0 \ar[d] \\
      K\otimes R^X \ar[r] \ar[d]
        & K^X \ar[d] \\
      R^m\otimes R^X\ar[r,"\simeq"] \ar[d]
        &(R^m)^X \ar[d] \\
      M\otimes R^X \ar[r,"\simeq"] \ar[d]
      & M^X \ar[d] \\
      0 & 0
    \end{tikzcd}
    \]
    tiene columnas exactas. Además, el renglón inferior es iso
    por hipótesis y el renglón de en medio es iso por la observación
    antes del lema anterior.
    Cazando un elemento $a\in K^X$, se muestra que
    $K\otimes R^X\to K^X$ es suprayectiva.
    Como $X$ es arbitrario, se sigue que $K$ es finitamente generado.
    Así, $M$ es finitamente presentado.
  \end{itemize}
\end{proof}

Finalmente, antes de probar el teorema que queremos, haremos una
definición más.

\begin{definition}[Módulos coherentes]
  Un módulo es coherente si es finitamente presentado y todos sus
  módulos finitamente generados son finitamente presentados.
\end{definition}

Ahora sí, el teorema que queremos es el siguiente.

\begin{theorem}[Chase]
  Sea $R$ un anillo. Son equivalentes:
  \begin{enumerate}[label=(\alph*)]
    \item Todo producto directo de módulos planos es plano.
    \item $R^I$ es plano para cualquier conjunto $I$.
    \item Todo módulo derecho finitamente presentado es
    coherente.
    \item Para todo $a\in R$, el anulador derecho
    $\Ann(a)\in\Lambda(R_R)$ es finitamente generado y la
    intersección de ideales derechos finitamente generados de $R$
    es un ideal derecho finitamente generado.
    \item $R_R$ es coherente.
  \end{enumerate}
\end{theorem}
\begin{proof}
  \begin{itemize}
    \item
    (a)$\implies$(b). Es inmediato, pues $R$ es plano.

    \item
    (b)$\implies$(c). Supongamos que $M_R$ es finitamente
    presentado. Mostraremos que $M$ es coherente. Para esto,
    tomemos un submódulo $L\leq M$ finitamente generado.
    Para cualquier conjunto $I$, consideremos los lados del
    cuadrado conmutativo el diagrama
    \[
      \begin{tikzcd}
        L\otimes R^I \ar[r]\ar[d] & M\otimes R^I \ar[d] \\
        L^I \ar[r] & M^I.
      \end{tikzcd}
    \]
    Como $L$ es finitamente generado, y $M$ es finitamente
    presentado, el lado izquierdo del cuadrado es epi y el lado
    derecho es iso, por los lemas anteriores.
    Por hipótesis, $R^I$ es plano, así que la flecha superior es
    mono.
    Se sigue que $L\otimes R^I\to L^I\to M^I$ es un monomorfismo,
    pues es composición de dos monomorfismos.
    Luego, $L\otimes R^I\to L^I$ es mono, así que es iso.
    Como $I$ es un conjunto arbitrario, se sigue que $L$ es
    finitamente presentado.
    
    \item
    (c)$\implies$(d). Sea $a\in R$.
    Notemos que el anulador derecho $\Ann(a)$ es el núcleo del
    morfismo $R\to aR$. Es decir, tenemos la sucesión exacta
    \[
      0\to\Ann(a)\to R\to aR\to 0
    .\]
    Como $aR$ es un submódulo finitamente generado de $R_R$,
    la hipótesis (c) nos dice que es finitamente presentado.
    Por lo tanto, existe una sucesión exacta
    $R^n\to R^m\to aR\to 0$.
    Como $R^m$ es libre, también es proyectivo, así que obtenemos
    una flecha $R^m\to R$ que hace conmutar el cuadrado derecho
    del siguiente diagrama.
    \[
      \begin{tikzcd}
        & R^n \ar[r]\ar[d,dotted]
          & R^m \ar[r]\ar[d]
          & aR \ar[r]\ar[d,"\id"]
          & 0 \\
        0 \ar[r]
          & \Ann(a) \ar[r]
          & R \ar[r]
          & aR \ar[r]
          & 0
      \end{tikzcd}
    .\]
    Esto produce la flecha punteada $R^n\to R$ que hace conmutar
    el cuadrado de la izquierda.
    Por el lema de la serpiente tenemos una sucesión exacta
    \[
      0\to\Coker(R^n\to\Ann(a))\to\Coker(R^m\to R)\to 0
    ,\]
    así que $\Coker(R^n\to\Ann(a))\simeq\Coker(R^m\to R)$.
    En particular, $\Coker(R^m\to\Ann(a))$ es finitamente generado,
    pues es un cociente de $R^m$. Luego, $\Ann(a)$ es finitamente
    generado, pues tenemos la sucesión exacta
    \[
      0\to\Im(R^n\to\Ann(a))\to\Ann(a)\to\Coker(R^n\to\Ann(a))\to 0
    .\]
    Ahora sean $I,J\leq R$ dos ideales derechos finitamente
    generados.
    Notemos que $I\cap J$ es el núcleo del morfismo $R\to
    R/I\oplus R/J$, así que tenemos la sucesión exacta
    \[
      0\to I\cap J\to R\xto\beta \frac{R}{I}\oplus \frac{R}{J}
    .\]
    Como la suma directa es exacta y $R/I$, $R/J$ son finitamente
    presentados, se sigue que $R/I\oplus R/J$ es finitamente
    presentado, así que es coherente, por la hipótesis (c).
    Luego, la imagen $\Im\beta$, siendo un submódulo finitamente
    generado de $R/I\oplus R/J$ (pues es cociente de $R$), es un
    módulo finitamente presentado.
    Así, tenemos una sucesión exacta
    \[
      0\to I\cap J \to R \xto{\beta'}\Im\beta\to 0
    ,\]
    donde $\beta'$ es la (co)restricción de $\beta$ a su imagen.
    Haciendo el mismo argumento que con la sucesión
    $0\to\Ann(a)\to R\to aR\to 0$, concluimos que $I\cap J$ es
    finitamente generado.
    
    \item
    (d)$\implies$(e).
    La sucesión $0\to R\to R\to 0$ muestra que $R$ es finitamente
    presentado.
    Ahora sea $I\leq R$ un ideal derecho finitamente generado,
    digamos $I=a_1R+a_2R+\dots+a_nR$.
    Por inducción sobre $n$, veremos que $I$ es finitamente
    presentado. Para $n=1$, tenemos la sucesión exacta
    \[
      0\to\Ann(a_1)\to R\to a_1R\to 0
    \]
    con $\Ann(a_1)$ finitamente generado por hipótesis (d), lo
    cual muetra que $a_1R=I$ es finitamente presentado.
    
    Ahora supongamos que $I_1=a_1R+a_2R+\dots+a_{n-1}R$ es
    finitamente presentado. Si $I_2=a_nR$, mostremos que
    $I=I_1+I_2$ es finitamente presentado.
    Consideremos la sucesión exacta
    \[
      0\to I_1\cap I_2\to I_1\oplus I_2\to I\to 0
    \]
    donde $I_1\oplus I_2$ es finitamente presentado porque $I_1$
    e $I_2$ lo son, mientras que $I_1\cap I_2$ es finitamente
    generado por la hipótesis (d).
    Luego, tenemos un diagrama con renglones y columas exactas
    \[
      \begin{tikzcd}
        && R^n\ar[d] \\
          & R^k \ar[d] \ar[r,dotted]
          & R^m \ar[d] \\
        0 \ar[r]
          & I_1\cap I_2\ar[r] \ar[d]
          & I_1\oplus I_2 \ar[r] \ar[d]
          & I \ar[r]
          & 0, \\
        & 0 & 0
      \end{tikzcd}
    \]
    donde la existencia del morfismo punteado $R^k\to R^m$ que
    hace conmutar el diagrama se debe a la proyectividad de
    $R^k$.
    Notemos que el morfismo compuesto $R^m\to I$ es epi, pues es
    composición de epis.
    Además, si $x\in R^m$ tiene imagen $0$ en $I$, entonces está
    en el núcleo de $I_1\otimes I_2\to I$, por lo cual tiene un
    levantamiento en $I_1\cap I_2$ y así tiene un levantamiento
    en $R^k$. 
    Esto muestra que la sucesión
    \[
      R^k\to R^m\to I\to 0
    \]
    es exacta y, por lo tanto, es una presentación finita de $I$.

    \item
    (e)$\implies$(a).
    Supongamos que $(L_i\mid i\in S)$ es una familia de
    $R$-módulos izquierdos planos.
    Sea $I\leq R$ es un ideal derecho finitamente generado.
    Como cada $L_i$ es plano, cada funtor $-\otimes L_i$ preserva
    monomorfismos. Así, de la sucesión exacta $0\to I\to R$
    obtenemos la exactitud de
    \[
      0\to I\otimes L_i\to L_i
    \]
    para cada $i\in S$.
    Luego, por la exactitud del producto directo, la sucesión
    \[
      0\to\prod_i(I\otimes L_i)\to\prod_iL_i
    \]
    es exacta.
    Ahora, por la hipótesis (e), $I$ es finitamente presentado,
    así que el morfismo
    \[
      I\otimes\prod_iL_i\to \prod_i(I\otimes L_i)
    \]
    es un iso.
    Como el cuadrado del diagrama
    \[
      \begin{tikzcd}
        0 \ar[r,dotted]
          & I\otimes\prod_iL_i \ar[r]\ar[d,"\simeq"']
          & \prod_iL_i \ar[d,"\id"] \\
        0 \ar[r]
          & \prod_i(I\otimes L_i) \ar[r]
          & \prod_iL_i
      \end{tikzcd}
    \]
    es conmutativo, donde el morfismo superior del cuardado se
    obtiene de $0\to I\to R$ al aplicar $-\otimes\prod_iL_i$,
    se sigue que el primer renglón, incluyendo la flecha
    punteada, también es exacto. Así, $\prod_iL_i$ es plano.
  \end{itemize}
\end{proof}

\begin{remark}
  En el transcurso de la demostración del teorema anterior,
  probamos dos resultados útiles acerca sucesiones exactas,
  así que vale la pena ponerlos por separado.
  Supongamos que
  \[
    0\to A\to B\to C\to 0.
  \]
  es una sucesión exacta.
  \begin{itemize}
    \item Si $C$ es finitamente presentado y $B$ es
    finitamente generado, entonces $A$ es finitamente
    generado. La prueba es como en (c)$\implies$(d).
    \item
    Si $B$ es finitamente presentado y $A$ es finitamente
    generado, entonces $C$ es finitamente presentado.
    La prueba es como en (d)$\implies$(e).
  \end{itemize}
\end{remark}

PENDIENTE SESIÓN 13
\todo{SES 13}

Bla bla bla

\section{Localización}
\begin{proposition}
Sean $R$ y $A$ anillos y sea $T:R\ds\Mod\to A\ds\Mod$ un funtor aditivo exacto. Entonces para cada complejo $C_\bul,d)\in \Com(R)$ y para cada $n\in\mathbb{Z}$ existe un isomorfismo 
$$H_n(TC_\bul,Td)\rightarrow TH_n(C_\bul,d).$$
\end{proposition}
\begin{proof}
Consideremos el siguiente diagrama conmutativo
$$
\begin{tikzcd}
             & C_{n+1} \arrow[r, "d_{n+1}"] \arrow[d, "d'_{n+1}"'] & C_n \arrow[r, "d_n"]        & C_{n-1}          &   \\
{} \arrow[r] & \Im d_{n+1} \arrow[r, "j"]                                & \ker d_n \arrow[r] \arrow[u, "k"] & H_n(C) \arrow[r] & 0
\end{tikzcd}
$$
aplicando $T$ tenemos
$$
\begin{tikzcd}
             & TC_{n+1} \arrow[r, "Td_{n+1}"] \arrow[d, "Td'_{n+1}"'] & TC_n \arrow[r, "Td_n"]        & TC_{n-1}          &   \\
{} \arrow[r] & T(\Im d_{n+1}) \arrow[r, "Tj"']                                & T(\ker d_n) \arrow[r] \arrow[u, "Tk"'] & TH_n(C) \arrow[r] & 0
\end{tikzcd}
$$
y $T(\Im d_{n+1})=\Im T(d_{n+1})$, ya que $T$ es exacto, y el último renglón del último diagrama puede reescribirse de la siguiente manera: 
$$
\begin{tikzcd}
 \Im Td_{n+1} \arrow[r, "Tj"']       & \ker Td_n \arrow[r] & TH_n(C) \arrow[r] & 0
\end{tikzcd}
$$
Luego, 
$$\frac{\ker Td_n}{\Im Td_n}=H_n(TC_\bul) \cong TH_n(C_\bul) $$
\end{proof}

Ahora, tomando un subconjunto multiplicativamente cerrado $S$ de un anillo $R$, veamos que el $\Tor$ conmuta con localización  através del siguiente resultado: 

\begin{proposition}
Si $S\subseteq R$ es un subconjunto multiplicativo, entonces existen isomorfismos 
$$S^{-1}(\Tor_n^R(A,B))\cong \Tor_n^{S^{-1}R}(S^{-1}A,S^{-1}B) \qquad \forall n\geq 0$$
para todos $A,B\in R-\Mod$.
\end{proposition}
\begin{proof}
Procedemos por inducción:
\begin{itemize}
    \item $n=0$. Para $A,B\in R$-mód fijos, existe un isomorfismo natural 
    $$\tau_{A,B}:S^{-1}(A\otimes_R B)\rightarrow S^{-1}A\otimes_{S^{-1}R}S^{-1}B $$ ya que
    \begin{eqnarray*}
    S^{-1}(A\otimes_R B) & \cong & S^{-1}R\otimes_R (A\otimes_R B)\\
    & \cong & (S^{-1}A\otimes_R A )\otimes_{S^{-1}R\otimes_R B}\\
    & \cong & S^{-1}A\otimes_{S^{-1}R}S^{-1}B
    \end{eqnarray*}
    \item $n\geq 0$. Ya que la localización es un funtor aditivo exacto, entonces 
    $$H_n(S^{-1})(A\otimes_R P_B)\cong S^{-1}(H_n(A\otimes P_B))\cong S^{-1}(\Tor_n^R(A,B)). $$
\end{itemize}
\end{proof}

De lo anterior, notemos que si $R$ es conmutativo, $\mathrm{m}$ maximal y si $A_\mathrm{m}$ es plano, entonces $\Tor_n^{R_\mathrm{m}}(A_\mathrm{m},B_\mathrm{m})=0$ para toda $n\leq 1$. Pero $\Tor_n^{R_\mathrm{m}}(A_\mathrm{m},B_\mathrm{m})=\Tor_n^R(A,B)_\mathrm{m}=0$. Así $\Tor_n^R(A,B)=0$, ya   que ser cero es una propiedad local. Y así, obtenemos el siguiente resultado: 
\begin{corollary}(Ser plano es una propiedad local)
Sea $A\in R\ds\Mod$, con $R$ un anillo conmutativo.
Si $A_\m$ es plano como $R_\m$-módulo para todo $\m\in\maxSpec(R)$,
entonces $A$ es plano. 
\end{corollary}
\begin{proof}
    Como cada $A_\m$ es plano, entonces $\Tor^{R_\m}_n(A_\m,B_\m)=0$
    para cualesquiera $n\geq 1$, $B\in R\ds\Mod$, $\m\in\maxSpec(R)$.
    Luego, $\Tor^R_n(A,B)_\m=0$
    para cualesquiera $n\geq 1$, $B\in R\ds\Mod$ y $\m\in\maxSpec(R)$.
    Como ser cero es una propiedad local, se sigue que $\Tor^R_n(A,B)=0$
    para todo $n\geq 1$ y $B\in R\ds\Mod$. Luego, $A$ es plano.
\end{proof}

\begin{lemma}
Si $R$ es un anillo noetheriano izquierdo y $M$ es finitamente generado, entonces existe una resolución proyectiva $P$ de $M$ donde cada $P_n$ es finitamente generado.
\end{lemma}
\begin{proof}
Ya que $M$ es finitamente generado, existe un $R$-mpdulo libre finitamente generado $P_0$ tal que
$$ 
\xymatrix{0\ar[r] &  K\ar[r] & \ar[r] P_0\ar[r] & M\ar[r] & 0 }
$$
como $P_0$ es finitamente generado, entonces $P_0\cong R^{(n)}$ entonces $K$ es finitamente generado, pues $R$ es noetheriano. 
\end{proof}

Finalizamos esta sección con el siguiente resultado
acerca de módulos finitamente
generados sobre un anillo noetheriano.

\begin{theorem}
Si $R$ es un anillo noetheriano conmutativo
y si $A$ y $B$ son $R$-módulos finitamente generados, entonces 
$\Tor_n^R(A,B)$
es finitamente generado, para toda $n\leq 0$.
\end{theorem}
\begin{proof}
Procedemos por inducción. 
\begin{itemize}
    \item $n=0$. Notemos que $A\otimes_R B$ es finitamente generado ya que $A$ y $B$ lo son en el anillo conmutativo $R$.
    \item $n\leq 0$. Sea $P_A$ una resolución proyectiva de $A$, ya que $P_n\otimes B$ es finitamente generado por el resultado anterior, entonces $\ker (d_n\otimes id_B)$ es finitamente generado y $\Tor_n^R(A,B)$ también lo es.  
\end{itemize}
\end{proof}

\section{Propiedades del Ext}
En esta sección estudiamos algunos resultados y propiedades con el $\Ext$. 
\begin{proposition}
Si $(A_k)_{k\in K}$ es una familia de módulos, entonces existe un isomorfismo natural $$\Ext_R^n(\oplus_{k}A_k,B) \cong \prod \Ext_R^n(A_k,B) .$$
\end{proposition}
\begin{proof}
\begin{itemize}
    \item $n=0$.
    Del teorema \ref{teorema:ext0-hom} tenemos $\Ext_R^0(-,B)\cong\Hom_R(-,B)$.
    \item $n=1$. Para $k\in K$ tenemos
    $$ 
    \xymatrix{0 \ar[r] & L_k \ar[r] & P_k \ar[r] & A_k \ar[r] & 0  }
    $$
    con $P_k$ proyectivo. De aquí, obtenemos la siguiente sucesión:
    $$ 
    \xymatrix{0 \ar[r] & \oplus_k L_k \ar[r] &  \oplus_k P_k \ar[r] & \oplus_k A_k \ar[r] & 0  }
    $$
    donde $\oplus P_k$ es proyectivo ya que la suma directa de proyectivos lo es. Por lo tanto, existe el siguiente diagrama conmutativo con renglones exactos
    $$
    \begin{tikzcd}
{\Hom(\oplus_k P_k,B)} \arrow[r] \arrow[d, "\tau"] & {\Hom(\oplus_k L_k,B)} \arrow[r, "\partial"] \arrow[d, "\sigma"] & {\Ext^1(\oplus_k A_k,B)} \arrow[r] \arrow[d] & {\Ext^1(\oplus_k P_k,B)} \arrow[d] \\
{\prod \Hom(P_k,B)} \arrow[r]                      & {\prod \Hom(L_k,B)} \arrow[r, "d"']                              & {\prod \Ext^1(A_k,B)} \arrow[r]              & {\prod \Ext^1(P_k,B)}             
\end{tikzcd}
    $$
    pero, 
    $$\Ext^1(\oplus P_k,B)=0=\prod \Ext^1(P_k,B) $$
    puesto que $P_k$ y $\oplus P_k$ son proyectivos. Así, $\partial$ y $d$ son sobreyectivos. Luego, existe un isomorfismo 
    $$\Ext^1(\oplus A_k,B)\cong \prod \Ext^1(A_k,B).$$
    \item $n\leq 1$. Tomamos el siguiente diagrama
    $$
    \begin{tikzcd}
{\Ext^n(\oplus_k P_k,B)} \arrow[r] & {\Ext^n(\oplus_k L_k,B)us_k L_k,B)} \arrow[r, "\partial"] \arrow[d, "\sigma"] & {\Ext^{n+1}(\oplus_k A_k,B)} \arrow[r] \arrow[d, dashed] & {\Ext^{n+1}(\oplus_k P_k,B)} \\
{\prod \Ext^n(P_k,B)} \arrow[r]    & {\prod \Ext^n(L_k,B)} \arrow[r, "d"']                                         & {\prod \Ext^{n+1}(A_k,B)} \arrow[r]                      & {\prod \Ext^{n+1}(P_k,B)}   
\end{tikzcd}
    $$
    donde $\sigma$ es un isomorfismo por hipótesis de inducción. 
    
    Ya que 
    $$\Ext^n(\oplus P_k,B)=0=\prod \Ext^n(P_k,B) $$
    puesto que $P_k$ y $\oplus P_k$ son proyectivos entonces $\partial$ y $d$ son isomorfismos. 
    Finalmente, $d\circ\partial^{-1}:\Ext^{n+1}(\oplus A_k,B)\rightarrow \prod \Ext^{n+1}(A_k,B)$ es un isomorfismo.  
\end{itemize}
\end{proof}

Tenemos el resultado análogo.
\begin{proposition}
Si $(B_k)_{k\in K}$ es una familia de módulos,
entonces existe un isomorfismo natural
$$\Ext_R^n(A,\prod_{k}B_k) \cong \prod \Ext_R^n(A,B_k) .$$
\end{proposition}

\subsection{Extensiones y Ext}
\begin{definition}
Dados $A,C\in R$-mód, una extensión de $A$ por $C$ es una sucesión exacta corta 
$$ 
\xymatrix{ 0\ar[r] & A \ar[r]^{i}& B\ar[r]^{p} & C \ar[r] &0 }.
$$
Una extensión se escinde si existe un mapeo $s:C\rightarrow R$ con $p s= \id_C$. 
\end{definition}

\begin{proposition}
Si $\Ext^1_R(C,A)=0$, entonces cada extensión de $A$ por $C$ se escinde.
\end{proposition}
\begin{proof}
Tomemos 
$$ 
\xymatrix{ 0\ar[r] & A \ar[r]^{i}& B\ar[r]^{p} & C }.
$$
aplicando $\Hom(C,-)$ tenemos 
$$ 
\xymatrix{ 0\ar[r] & \Hom(C,B) \ar[r]^{p_*}&  \Hom(C,C) \ar[r]^{\partial} & \Ext^1(C,A) }.
$$
Por hipótesis, $\Ext^1(C,A)=0$ y así $p_*$ es sobrectiva. Y existe $s\in \Hom(C,B)$ tal que $\id_C=p_*(s)$, es decir, $ps=\id_C$
\end{proof}

El siguiente corolario se sigue inmediatamente de la proposición anterior. 

\begin{corollary}
\begin{enumerate}
    \item $P\in R$-mód es proyectivo si y solo si $\Ext^1_R(P,B)=0$ para todo $B\in R$-mód.
    \item $E\in R$-mód es inyectivo si y solo si $\Ext^1_R(A,E)=0$ para todo $A\in R$-mód.
\end{enumerate}
\end{corollary}

\begin{definition}
Dados dos módulos $C$ y $A$, dos extensiones de $A$ por $C$ 
$$ 
\xi: \xymatrix{ 0\ar[r] & A \ar[r]& B\ar[r] & C \ar[r] &0 },
\quad
\xi': \xymatrix{ 0\ar[r] & A \ar[r]& B'\ar[r] & C \ar[r] &0 }.
$$
decimos que un isomorfismo $\varphi:B\rightarrow B'$ es
un isomorfismo de extensiones $\varphi:\xi\simeq\xi'$
si el siguiente diagrama es conmutativo.
$$
\begin{tikzcd}
0 \arrow[r] & A \arrow[r] \arrow[d, "\id_A"] & B \arrow[r] \arrow[d, "\varphi"] & C \arrow[r] \arrow[d, "\id_C"] & 0 \\
0 \arrow[r] & A \arrow[r]                    & B' \arrow[r]                     & C \arrow[r]                    & 0
\end{tikzcd}
$$
(de hecho, por el lema del 5, cualquier morfismo $\phi$ que haga conmutar
el diagrama es, necesariamente, un isomorfismo).
Se verifica inmediatamente que la existencia de un isomorfismo
$\xi\simeq\xi'$ es una relación de equivalencia
en la clase de extensiones de $A$ por $C$. Entonces definimos
$$e(C,A) =
\left\{\text{extensiones de $A$ por $C$ }\right\}/\simeq$$
\end{definition}
Nuestro objetivo ahora es construir una biyección
\[
    e(C,A) \simeq \Ext^1(C,A).
\]
El primer paso será construir una función $e(C,A)\to\Ext^1(C,A)$.
Sea $(P_C,\epsilon)$ una resolución proyectiva de $C$, de modo que
$\Ext^n(C,A)$ es la homología del complejo
\[
    \dots \from
    \Hom(P_3,A) \xfrom{d_3^*}
    \Hom(P_2,A) \xfrom{d_2^*}
    \Hom(P_1,A) \xfrom{d_1^*}
    \Hom(P_0,A) \xfrom{d_0^*}
    0.
\]
En particular, $\Ext^1(C,A)=\ker(d_2^*)/\Img(d_1^*)$.

Ahora tomemos una extensión $\xi$ de $A$ por $C$:
\[
    \xi=(0\to A\to B\to C\to 0).
\]
Dado que $\epsilon:P_C\to C$ es una resolución proyectiva y la sucesión
$0\to A\to B\to C$ es exacta, el lema de comparación nos da
un morfismo de complejos $\alpha:(P_C,\epsilon)\to\xi$ sobre $\id_C$:
\[
    \begin{tikzcd}
        \dots \ar[r] &
        P_3 \ar[r,"d_3"] \ar[d,"\alpha_3"] &
        P_2 \ar[r,"d_2"] \ar[d,"\alpha_2"] &
        P_1 \ar[r,"d_1"] \ar[d,"\alpha_1"] &
        P_0 \ar[r,"\epsilon"] \ar[d,"\alpha_0"] &
        C \ar[r] \ar[d,"\id_C"] & 0 \\
        \dots \ar[r] &
        0 \ar[r] &
        0 \ar[r] &
        A \ar[r] &
        B \ar[r] &
        C \ar[r] & 0.
    \end{tikzcd}
\]
Nótese que $\alpha_n=0$ para $n\geq 2$.
Así, $\alpha_1d_2=0$ por la conmutatividad del diagrama.
Esto es, $d_2^*(\alpha_1)=0$, por lo cual $\alpha_1\in\ker(d_2^*)$.
De este modo, obtenemos $[\alpha_1]\in\Ext^1(C,A)$.
Para ver que esto define una función $e(C,A)\to\Ext^1(C,A)$,
probaremos que la asignación no depende del morfismo de complejos
$\alpha$ ni del representante de la clase de $\xi$ en $e(C,A)$.
Primero, si $\alpha':(P_C,\epsilon)\to\xi$ es otro morfismo sobre $\id_C$,
entonces el lema de comparación nos da una homotopía
$s:\alpha'\sim\alpha$.
En grado $1$, tenemos el diagrama (no conmutativo)
\[
    \begin{tikzcd}
        P_2 \ar[r,"d_2"] &
        P_1 \ar[r,"d_1"] \ar[dl,"s_1"']
        \ar[d,"\alpha_1"',shift right]
        \ar[d,"\alpha_1'",shift left] &
        P_0 \ar[dl,"s_0"]
        \\
        0 \ar[r] &
        A \ar[r] &
        B
    \end{tikzcd}
\]
con $\alpha_1'-\alpha_1=0s_1+s_0d_1=s_0d_1=d_1^*(s_0)$.
Es decir, $\alpha_1'-\alpha_1 \in \Img(d_1^*)$, por lo cual
$[\alpha_1']=[\alpha_1]$.

Ahora supongamos que $\xi'=(0\to A\to B'\to C\to 0)$
es una extensión isomorfa a $\xi$.
Tomando de nuevo el morfismo de complejos $\alpha:(P_C,\epsilon)\to\xi$
dado por el lema de comparación, podemos componerlo con
el isomorfismo $\xi\simeq\xi'$ para obtener un morfismo de complejos
$(P_C,d_0)\to\xi'$ sobre $\id_C$.
\[
    \begin{tikzcd}
        \dots \ar[r] &
        P_3 \ar[r,"d_3"] \ar[d,"\alpha_3"] &
        P_2 \ar[r,"d_2"] \ar[d,"\alpha_2"] &
        P_1 \ar[r,"d_1"] \ar[d,"\alpha_1"] &
        P_0 \ar[r,"\epsilon"] \ar[d,"\alpha_0"] &
        C \ar[r] \ar[d,"\id_C"] & 0 \\
        \dots \ar[r] &
        0 \ar[r] \ar[d] &
        0 \ar[r] \ar[d] &
        A \ar[r] \ar[d,"\id_A"] &
        B \ar[r] \ar[d,"\beta"] &
        C \ar[r] \ar[d,"\id_C"] & 0 \\
        \dots \ar[r] &
        0 \ar[r] &
        0 \ar[r] &
        A \ar[r] &
        B' \ar[r] &
        C \ar[r] & 0.
    \end{tikzcd}
\]
Así, a $\xi'$ le corresponde en $\Ext^1(C,A)$ la clase de
$\id_A\alpha_1=\alpha_1$, que es el mismo elemento que le corresponde
a $\xi$. Por lo tanto, la asignación $[\xi]\mapsto[\alpha_1]$ es una función
bien definida
\[
    \psi:e(C,A)\to\Ext^1(C,A)
.\]

\begin{remark}
Notemos que, si $\xi:0\to A\to B\xto p C\to 0$ se escinde,
entonces $\psi([\xi])=0$. En efecto, si $p$ tiene una sección $j:C\to B$
(es decir, $pj=\id_C$), entonces $pj\epsilon=\epsilon$, así que
el siguiente diagrama es un morfismo de complejos
\[
    \begin{tikzcd}
        \dots \ar[r] &
        P_2 \ar[r,"d_2"] \ar[d,"0"] &
        P_1 \ar[r,"d_1"] \ar[d,"0"] &
        P_0 \ar[r,"\epsilon"] \ar[d,"j\epsilon"] &
        C \ar[r] \ar[d,"\id_C"] & 0 \\
        \dots \ar[r] &
        0 \ar[r] &
        A \ar[r] &
        B \ar[r,"p"] &
        C \ar[r] \ar[l,"j",bend left] & 0.
    \end{tikzcd}
\]
con $\alpha_1=0$. Luego, $\psi([\xi])=[\alpha_1]=[0]$.
\end{remark}
Ahora vamos a construir la función $\Ext^1(C,A)\to e(C,A)$ inversa a $\psi$.
Para esto, consideremos primero el siguiente lema.
\begin{lemma}
Dada una extensión $\Xi:0\to X_1\xto j X_0\xto\epsilon C\to 0$
y un morfismo $h:X_1\to A$, existe una única extensión (salvo isomorfismo)
$0\to A\xto i B\xto\eta C\to 0$ tal que el diagrama
\[
    \begin{tikzcd}
        0 \ar[r] &
        X_1 \ar[r,"j"] \ar[d,"h"] &
        X_0 \ar[r,"\epsilon"] \ar[d,"\beta"] &
        C \ar[r] \ar[d,"\id_C"]
        & 0 \\
        0 \ar[r] &
        A \ar[r,"i"] &
        B \ar[r,"\eta"] &
        C \ar[r] & 0.
    \end{tikzcd}
\]
es conmutativo.
\end{lemma}
\begin{proof}
    Sea $B=A\oplus_{X_1}X_0=\Coeq(j\oplus 0,-h\oplus 0)$
    el coproducto fibrado de $h$ y $j$.
    Definiendo $i,\beta$ como los morfismos canónicos,
    \[
        \begin{tikzcd}
            X_1 \ar[r,"j"] \ar[d,"h"']
            \ar[dr,phantom,"\ulcorner" description,very near end]
            & X_0 \ar[d,"\beta"] \\
            A \ar[r,"i"'] & B
        \end{tikzcd}
    \]
    tenemos que $i$ es mono porque $j$ lo es.
    Dado que $\epsilon j=0=0h:X_1\to C$, la propiedad del coproducto fibrado,
    nos da una única flecha $\eta:B\to C$ que hace conmutar el diagrama
    \[
        \begin{tikzcd}
            X_1 \ar[r,"j"] \ar[d,"h"']
            & X_0 \ar[d,"\beta"] \ar[rdd,"\epsilon",bend left] \\
            A \ar[r,"i"'] \ar[drr,"0"',bend right] & B \ar[dr,dotted,"\eta"] \\
            && C
        \end{tikzcd}
    \]
    Notemos que $\eta$ es epi, pues si $f:C\to Z$ cumple $f\eta=0$,
    tenemos $0=f\eta\beta=f\epsilon$, así que $f=0$ porque $\epsilon$ es epi.
    Por lo tanto, el diagrama
    \[
        \begin{tikzcd}
            0 \ar[r] &
            X_1 \ar[r,"j"] \ar[d,"h"] &
            X_0 \ar[r,"\epsilon"] \ar[d,"\beta"] &
            C \ar[r] \ar[d,"\id_C"]
            & 0 \\
            0 \ar[r] &
            A \ar[r,"i"] &
            B \ar[r,"\eta"] &
            C \ar[r] & 0.
        \end{tikzcd}
    \]
    es conmutativo y solo queda revisar la exactitud del segundo renglón en $B$.
    Para esto, veamos que $\eta:B\to C$ cumple la propiedad del conúcleo de $i$.
    Consideremos un morfismo $g:B\to Z$ con $gi=0$.
    Entonces $g\beta j=gih=0$. Como $X_0\xto\epsilon C$ es el conúcleo de $j$,
    la composición $g\beta$ se factoriza de manera única a través de $\epsilon$,
    digamos como $g\beta=u\epsilon$.
    \[
        \begin{tikzcd}
            0 \ar[r] &
            X_1 \ar[r,"j"] \ar[d,"h"] &
            X_0 \ar[r,"\epsilon"] \ar[d,"\beta"] &
            C \ar[r] \ar[d,"\id_C"]
            & 0 \\
            0 \ar[r] &
            A \ar[r,"i"] \ar[dr,"0"'] &
            B \ar[r,"\eta"] &
            C \ar[r] & 0
            \\
            && Z \ar[from=u,"g"']
            \ar[from=uur,"u"',near start,bend right=25,crossing over]
        \end{tikzcd}
    \]
    Notemos que tanto $g$ como $u\eta$ hacen conmutar el diagrama
    \[
        \begin{tikzcd}
            X_1 \ar[r,"j"] \ar[d,"h"']
            & X_0 \ar[d,"\beta"] \ar[rdd,"g\beta",bend left] \\
            A \ar[r,"i"'] \ar[drr,"0"',bend right]
            & B \ar[dr,"g",shift left] \ar[dr,"u\eta"',shift right] \\
            && Z
        \end{tikzcd}
    \]
    Por la propiedad universal del producto fibrado, se sigue que $g=u\eta$.
    Más aún, esta $u$ es la única factorización de $g$ a través de $\eta$,
    ya que $\eta$ es epi.
    Esto muestra que $B\xto\eta C$ es el conúcleo de $A\xto i B$, así que
    el diagrama
    \[
        \begin{tikzcd}
            0 \ar[r] &
            X_1 \ar[r,"j"] \ar[d,"h"] &
            X_0 \ar[r,"\epsilon"] \ar[d,"\beta"] &
            C \ar[r] \ar[d,"\id_C"]
            & 0 \\
            0 \ar[r] &
            A \ar[r,"i"] &
            B \ar[r,"\eta"] &
            C \ar[r] & 0.
        \end{tikzcd}
    \]
    tiene renglones exactos.
    
    Ahora supongamos que $0\to A\to B'\to C\to 0$ es otra extensión que hace
    conmutar el diagrama
    \[
        \begin{tikzcd}
            0 \ar[r] &
            X_1 \ar[r,"j"] \ar[d,"h"] &
            X_0 \ar[r,"\epsilon"] \ar[d,"\beta'"] &
            C \ar[r] \ar[d,"\id_C"]
            & 0 \\
            0 \ar[r] &
            A \ar[r,"i'"] &
            B' \ar[r,"\eta'"] &
            C \ar[r] & 0.
        \end{tikzcd}
    \]
    Como $B$ es el coproducto de $h$ y $j$, tenemos una única flecha
    $\phi:B\to B'$ que hace conmutativo el diagrama
    \[
        \begin{tikzcd}
            X_1 \ar[r,"j"] \ar[d,"h"']
            & X_0 \ar[d,"\beta"] \ar[rdd,"\beta'",bend left] \\
            A \ar[r,"i"'] \ar[drr,"i'"',bend right]
            & B \ar[dr,dotted,"\phi"] \\
            && B'
        \end{tikzcd}
    \]
    Luego, tenemos el diagrama
    \[
        \begin{tikzcd}
            0 \ar[r] &
            X_1 \ar[r,"j"] \ar[d,"h"] &
            X_0 \ar[r,"\epsilon"] \ar[d,"\beta"] &
            C \ar[r] \ar[d,"\id_C"]
            & 0 \\
            0 \ar[r] &
            A \ar[r,"i"] \ar[d,"\id_A"] &
            B \ar[r,"\eta"] \ar[d,"\phi"] &
            C \ar[r] \ar[d,"\id_C"] & 0 \\
            0 \ar[r] &
            A \ar[r,"i'"] &
            B' \ar[r,"\eta'"] &
            C \ar[r] & 0.
        \end{tikzcd}
    \]
    Por el lema del 5, $\phi$ es un isomorfismo y, por lo tanto, las
    extensiones de $B$ y $B'$ son isomorfas.
\end{proof}

PENDIENTE SESIÓN 16
\todo{SES 16}

\chapter{Cohomología de gavillas}
\section{Gavillas}
Existen muchos tipos de categorías de gavillas,en particular, nosotros nos enfocaremos en gavillas abelianas. En todo momento $X$ será un espacio topológico y denotaremos por $\Omega X$ al conjunto de abiertos de $X$. El objetivo de las gavillas es estudiar a $X$ en términos de sus invariantes topológicos, para ello le asociamos un funtor cotravariante que va de $\Omega X$ a $\Ab$, las categorías abelianas, formalmente:  

\begin{definition}
Una \textbf{pregavilla} en $X$ es un funtor contravariante 
$$\mathcal{F}:\Omega X  \rightarrow \Ab$$
 tal que si $U\subseteq V$ en $X$, entonces tenemos las flechas
 $$\F(V)\rightarrow \F(U) $$
 llamadas \textbf{restricciones} y denotadas por $res_U^V$. 
 \end{definition}

 A los elementos de $\F(V)$ se les llama secciones. Así, sea $s\in \F(V)$, entonces $res_U^V(s)=s|_U$. 
 
 \begin{definition}
 Un \textbf{morfismo de pregavillas} es una transformación natural $\eta: \F \rightarrow \G$, es decir, para toda $U\in\Omega X$,
 $$\eta_U:\F(U)\rightarrow \G(U) $$
 es un morfismo de grupos, y de hecho, si $U\subseteq V$, entonces el siguiente diagrama conmuta
 $$
 \begin{tikzcd}
\F(V) \arrow[r, "\eta_V"] \arrow[d, "res_U^V"'] & \G(V) \arrow[d, "res_U^V"] \\
\F(U) \arrow[r, "\eta_U"']                      & \G(U)                     
\end{tikzcd}
 $$
 \end{definition}
Es claro que esto forma una categoría, llamada categoría de pregavillas abelianas del espacio topológico $X$, denotada por $\Psh(X,\Ab)$.

\begin{definition}[Axioma de gavilla]
Una pregavilla $\F\in \Psh(X,\Ab)$ es una gavilla si para todo $U\in\Omega X$ tal que $\mathcal{U}=\{U_i\}$ es una cubierta de $U$ se cumple que: 
\begin{enumerate}
    \item para cualesquiera dos secciones $s,s'\in \F(U)$ tales que
    $$s|_{U_i}=s'|_{u_j} \qquad \forall i,j$$
    entonces $s=s'$.
    \item  si $s_i\in\F(U_i)$ y $s_j\in \F(U_j)$ con $U_i,U_j\in\mathcal{U}$ tal que $$ s_i|_{U_i\cap U_j}=s_j|_{U_i\cap U_j}\qquad \forall i,j$$
    entonces existe una sección $s\in \F(U)$ tal que $s|_{U_i}=s_i$.
 \end{enumerate}
\end{definition}

Algunas observaciones que se siguen directamente de la definición son las siguientes: 
\begin{itemize}
    \item Observemos que la sección global de (ii) es única por (i). 
    \item  $\F(\emptyset)=0$. Ya que si $\emptyset=\cup U_i$ con $j\in\emptyset$ y si $s,s'\in\F(\emptyset)$ entonces por vacuidad 
    $$s|_{U_i}=s'|_{U_j} $$
    para toda $i,j$ y por lo tanto, $s=s'$.
    \item Si hay un elemento en $\F(\emptyset)$ es único. En efecto, si $s_i\in\F(U_i)$ tal que 
    $$ s_i|_{U_i\cap U_j}=s_j|_{U_i\cap U_j}\qquad \forall i,j$$
    entonces por (ii) existe una única sección global tal que $s|_{U_i}=s_i$.
\end{itemize}
\begin{example}
Sea $X$ un espacio topológico. Para cada $U\in\Omega X$,
definimos $\cal C(U)=\{f:U\to\C: f \textup{ es continua}\}$.
Así, $\cal C$ es una gavilla con valores en $\C$-espacios vectoriales.
$$\cal C:\Omega X\to\C\ds\Mod .$$
\end{example}

\begin{example}
Si $S$ es una superficie de Riemann, entonces para todo $U\in\Omega S$ definimos
$$\cal O(U)=\{f:U\to\C: f \textup{ es holomorfa}\}. $$ 
Este es un funtor contravariante
$$\mathcal{O}(-):\Omega S \rightarrow \textup{Rings}. $$
Además,  
$\cal C^\infty(U) $ es una gavilla de funciones diferenciables. En particular, podemos construir el haz tangente como una gavilla. 
\end{example}

\subsection{Fibras}

Dado un espacio topológico $X$ y un punto $p\in X$, denotemos como $\cal U(p)$
al sistema de vecindades abiertas de $p$ en $X$:
$$\cal U(p)=\{U\in\Omega X:p\in U\}.$$
Notemos que $\cal U(p)$, ordenado con la contención usual,
es un conjunto codirigido (dirigido hacia abajo), pues
si $U_\alpha,U_\beta\in\cal U(p)$, entonces $U_\alpha \cap U_\beta\in\cal U(p)$.
Al aplicar $\cal F$, la familia $\cal F(\cal U(p))$ es un conjunto
dirigido (hacia arriba), pues las inclusiones
$$\iota_\alpha^\beta:U_\alpha \rightarrow U_\beta$$
se convierten en morfismos
$$\varphi_\alpha^\beta
=\F(\iota_\alpha^\beta):\F(U_\beta)\rightarrow \F(U_\alpha).$$
Así, tenemos el límite directo $\lim_{U\in\cal U(p)}\cal F(U)$.

\begin{definition}
La \textbf{fibra} de $\F$ en el punto $p$ es el límite directo
$$\F_p:=\colim \cal F(\cal U(p))=\colim_{p\in U} \F(U).$$
\end{definition}
Notemos que, como $\cal F(\cal U(p))$ es un conjunto dirigido,
entonces $\cal F_p$ tiene la descripción
\[
    \cal F_p = (\bigsqcup_{U\in\cal U(p)} \cal F(U))/\sim
\]
donde dos secciones $s\in\cal F(U),s'\in\cal F(U')$, definidas en vecindades
abiertas $U,U'$ de $p$, son equivalentes ($s\sim s'$) si existe
una vecindad abierta $U''$ de $p$ con $U''\subseteq U\cap U'$, tal que
$s|_{U''}=s'|_{U''}$.
La clase de $s\in\cal F(U)$ en $\cal F_p$ se llama germen de $s$ en $p$,
y la podemos denotar como
$$s(p)=[s]_p=\<U,s\>_p.$$
Cuando el punto $p$ es claro por contexto,
podemos omitirlo de la notación, para escribir $[s]$ o $\<U,s\>$.

\begin{example}
Sea $S$ una superficie de Riemann.
Tomamos $\mathcal{O}:\Omega S\rightarrow \textup{Ab}$  tal que $\mathcal{O}_p=\lim_{\rightarrow U}\mathcal{O}(U)$ y tomemos $s\in \mathcal{O}(U_\alpha)$ una función holomorfa definida en $U_\alpha$, $s:U_\alpha\rightarrow \mathbb{C}$. Si elegimos cartas $(V,z)$ al rededor de $p$ 
y $V$ es una carta de $z(p)$ en $\mathbb{C}$, entonces
$$s(z)=\sum_{n=0}^{\infty} c_n(z-z(p))^n $$
Tomando el límite directo en $p$, obtenemos
$$\mathcal{O}_p\mapsto \mathbb{C}\{z-z(p)\} $$
es decir, la fibra en $p$ es isomorfo al anillo en series de Taylor en los complejos al rededor del punto $z(p)$. 

De manera similar, tomando la gavilla $\mathcal{M}:\Omega S\rightarrow \textup{Campos}$, vemos que la fibra es isomorfa al anillo en series de potencias al rededor del punto $z(p)\in\mathbb{C}$: 
$$ \mathcal{M}_p\mapsto \mathbb{C}[[z-z(p)]].$$
\end{example}

\begin{lemma}
Sea $F$ una gavilla en un espacio topológico $S$ y $u\in\Omega S$
cualquier abierto.
Dadas secciones $s,t\in Fu$, si $s_p=t_p$ para todo $p\in u$,
entonces $s=t$.
\end{lemma}
\begin{proof}
    Para cada $p\in u$ tenemos $[s]_p=[t]_p$.
    Por la descripción de $F_p$, esto significa que,
    para cada $p\in u$, existe un abierto
    $u_p\subseteq u$, con $p\in u_p$, donde las restriciones
    de $s$ y $t$ coinciden:
    \[
        s|u_p=t|u_p
    .\]
    Como para cada $p\in u$ tenemos $p\in u_p\subseteq u$,
    la colección $\{u_p\mid p\in u\}$ es una cubierta de $u$.
    Por el axioma de localidad (primer axioma de gavilla),
    se sigue que $s=t$.
\end{proof}

\section{La categoría de gavillas}
Dado un espacio topológico $S$, las pregavillas y las gavillas
sobre $S$ forman categorías, donde los morfismos son las transformaciones
naturales.

\begin{lemma}
    Un morfismo de (pre)gavillas $\eta:F\to G$ es un isomorfismo
    si, y solo si, para todo abierto $u\in\Omega S$, el morfismo
    $\eta_u:Fu\to Gu$ es un isomorfismo.
\end{lemma}

Ahora tomemos un morfismo de gavillas $\eta:F\to G$.
Dado un punto $p\in S$, consideremos el sistema de vecindades abiertas de $p$.
Si $u,v$ son vecindades de $p$ con $v\leq u$, tenemos un diagrama conmutativo
\[
    \begin{tikzcd}[column sep=0.01]
        Fu \ar[rr] \ar[d] && Fv \ar[d] \\
        Gu \ar [rr] \ar[dr] && Gv \ar[dl] \\
        & G_p
    \end{tikzcd}
\]
así, la familia de morfismos $Fu\to G_p$ nos da un único morfismo
$F_p\to G_p$ que hace conmutar todos los diagramas de la forma
\[
    \begin{tikzcd}
        Fu \ar[d] \ar[r] & Gu \ar[d] \\
        F_p \ar[r] & G_p.
    \end{tikzcd}
\]
con $p\in u$.

\begin{proposition}
Un morfismo de gavillas $\eta:F\to G$ es un isomorfismo
si, y solo si, para todo $p\in S$ el morfismo entre las fibras
$\eta_p:F_p\to G_p$ es un isomorfismo.
\end{proposition}
\begin{proof}
    Por un lado si $\eta$ es un isomorfismo, entonces
    cada $\eta_u:Fu\to Gu$ es un isomorfismo, así que
    $\eta_p:F_p\to G_p$ es un isomorfismo para cada $p\in S$.
    
    Por otro lado, supongamos que, para cada $p\in S$,
    el morfismo $\eta_p$ es iso.
    Sea $u\in\Omega S$ y mostremos que $\eta_u:Fu\to Gu$ es un iso.
    Tomando $s,t\in Fu$, con $\eta_us = \eta_ut$,
    para cada $p\in u$ los gérmenes $[\eta_us]_p=[\eta_us]_p$
    son $\eta_p[s]_p=\eta_p[t]_p$.
    Como cada $\eta_p$ es un isomorfismo,
    tenemos $[s]_p=[t]_p$ para todo $p$.
    Como $F$ es gavilla, esto implica que $s=t$.
    Así, $\eta_u$ es inyectiva.
    
    Ahora consideremos una sección $r\in Gu$.
    Como cada $\eta_p$ es suprayectiva, para cada $p\in U$
    existe una sección $t_p\in F(v_p)$ definida en una vecindad $v_p$ de $p$
    tal que $\eta_p[t_p]_p=[r]_p$.
    
    Nos gustaría pegar estas secciones $t_p$ en la cubierta
    $\{v_p\mid p\in u\}$. Lamentablemente, pueden no coincidir en las
    intersecciones $v_p\cap v_q$. Por eso, necesitamos restringirnos más.
    
    Para cada $p\in u$, tenemos $\eta_p[t_p]_p=[r]_p$,
    esto es: $[\eta_{v_p}t_p]_p=[r]_p$,
    lo cual significa que existe un abierto $u_p$
    con $p\in u_p\subseteq v_p\subseteq u$ tal que las restricciones de
    $\eta_{v_p}t_p\in G(v_p)$ y $r\in Gu$ coinciden:
    \[
        r|u_p = (\eta_{v_p}t_p)|u_p = \eta_{u_p}(t_p|u_p)
    \]
    Ahora sí, definamos $s_p=(t_p|u_p)\in F(u_p)$ para cada $p\in u$,
    de modo que
    \[
        r|u_p = \eta_{u_p}s_p
    .\]
    
    Esta familia de secciones $s_p$ sí las podremos pegar.
    Para esto, veamos que coinciden
    en las intersecciones $u_p\cap u_q$ para cualesquiera dos puntos
    $p,q\in u$. Tomando cualquier punto $x\in u_p\cap u_q$, tenemos
    \begin{align*}
        \eta_x[s_p|(u_p\cap u_q)]_x
        &= \eta_x[s_p]_x \\
        &= [\eta_{u_p}s_p]_x \\
        &= [r|u_p]_x \\
        &= [r|u_q]_x \\
        &= [\eta_{u_q}s_q]_x \\
        &= \eta_x[s_q]_x \\
        &= \eta_x[s_q|(u_p\cap u_q)]_x.
    \end{align*}
    Como cada $\eta_x$ es inyectiva, se sigue que
    $[s_p|(u_p\cap u_q)]_x=[s_q|(u_p\cap u_q)]_x$.
    Por el principio de localidad, tenemos
    $s_p|(u_p\cap u_q)=s_q|(u_p\cap u_q)$.
    Notando que $\{u_p\mid p\in u\}$ es una cubierta de $u$,
    el axioma de pegado (segundo axioma de gavilla) nos permite obtener
    una sección $s\in Fu$ que coincida con $s_p\in F(u_p)$
    en cada $u_p$:
    \[
        s|u_p = s_p
    .\]
    Luego, $s\in Fu$ cumple $\eta_us=r$, dado que, para todo $p\in u$,
    tenemos
    \begin{align*}
        [\eta_us]_p
        &= [(\eta_us)|u_p]_p \\
        &= [\eta_{u_p}(s|u_p)]_p \\
        &= [\eta_{u_p}s_p]_p \\
        &= [r|u_p]_p \\
        &= [r]_p.
    \end{align*}
\end{proof}

\section{Núcleos, imágenes y conúcleos}

\begin{definition}
    Sea $\phi:F\to G$ un morfismo de pregavillas sobre $S$.
    Las construcciones en $\Psh(S)$ se hacen abierto por abierto:
    \begin{enumerate}
        \item El núcleo $\ker\phi$ es la pregavilla dada para todo
        $u\in\Omega S$ como $(\ker\phi)(u)=\ker(\phi_u:Fu\to Gu)$.
        \item La imagen $\Im\phi$ es la pregavilla dada para todo
        $u\in\Omega S$ como $(\Im\phi)(u)=\Im(\phi_u:Fu\to Gu)$.
        \item El conúcleo $\Im\phi$ es la pregavilla dada para todo
        $u\in\Omega S$ como $(\Coker\phi)(u)=\Coker(\phi_u:Fu\to Gu)$.
    \end{enumerate}
\end{definition}

\begin{definition}
Si $F,G\in\Psh(S)$, diremos que $F$ es una subpregavilla de $G$,
$F\leq G$, si
\begin{enumerate}
    \item $\forall u\in\Omega S$, $Fu\leq Gu$
    \item Para toda contención $u\leq v$, el morfismo
    $Fv\to Fu$ es la restricción de $Gv\to Gu$.
\end{enumerate}

Si $F\leq G$, entonces la pregavilla cociente, denotada como $G/F$,
es la pregavilla dada por $(G/F)(u)=(Gu)/(Fu)$ para todo $u\in\Omega S$
y, para toda contención $u\leq v$, el morfismo $(G/F)(v)\to(G/F)(u)$
es el único que hace conmutar el siguiente diagrama:
\[
\begin{tikzcd}
    Fv \ar[r] \ar[d] & Fu \ar[d] \\
    Gv \ar[r] \ar[d] & Gu \ar[d] \\
    (G/F)(v) \ar[r] & (G/F)(u).
\end{tikzcd}
\]
\end{definition}

\begin{proposition}
    Si $\phi:F\to G$ es un morfismo de gavillas, entonces $\ker\phi$ es
    una gavilla.
\end{proposition}
\begin{proof}
    Sea $\cal U\subseteq \Omega X$ una familia
    de abiertos y definamos $U=\bigcup\cal U$.
    Supongamos que tenemos secciones $s,t\in(\ker\phi)U$ que coinciden
    en cada $u\in\cal U$:
    \[
        s|u = t|u
    .\]
    Dado que $s,t\in(\ker\phi)(U)=\ker(\phi_U)\subseteq FU$
    y $F$ es una gavilla, esto implica que $s=t$, así que $\ker\phi$
    cumple el axioma de localidad.
    
    Ahora supongamos que tenemos una familia de secciones
    $\{s_u\mid u\in\cal U\}$ que coinciden en las interseciones $u\cap v$
    para cada $u,v\in\cal U$:
    \[
        s_u|(u\cap v) = s_v|(u\cap v)
    .\]
    Dado que $s_u\in(\ker\phi)(u)=\ker(\phi_u)\subseteq Fu$, el axioma
    de pegado nos da una sección $s\in FU$ que coincide
    con $s_u$ en cada $u$
    \[
        s|u = s_u
    ,\]
    pero falta probar que $s\in(\ker\phi)(U)$.
    Para cada $u\in \cal U$, tenemos $s_u\in\ker(\phi_u)$.
    Denotando los elementos neutros como $0_u\in Fu$ y $0_U\in FU$, tenemos
    \begin{align*}
        (\phi_Us)|u
        &= \phi_us_u \\
        &= 0_u \\
        &= 0_U|u
    \end{align*}
    Por el axioma de localidad, se sigue que $\phi_Us=0_U\in FU$.
    Así, $s\in\ker(\phi_U)=(\ker\phi)(U)$, por lo cual $\ker\phi$
    cumple el axioma de pegado.
    
    Se sigue que $\ker\phi$ es una gavilla.
\end{proof}

\begin{example}
    Sea $S$ una superficie de Riemann.
    Entonces tenemos las gavillas $\cal O$, $\cal O^*$,
    dadas para cada $u\in\Omega S$ como
    \begin{align*}
        \cal Ou &= \{s:u\to\C \mid s \text{ holomorfa } \} \\
        \cal O^*u &= \{s:u\to\C^* \mid s \text{ holomorfa } \},
    \end{align*}
    Si $\cal O$ tiene estructura aditiva y $\cal O^*$,
    entonces tenemos un morfismo de gavillas abelianas
    \[
        \exp:\cal O\to\cal O^*
    \]
    dado, para cada $u\in\Omega S$, como
    \begin{align*}
        \exp_u:\cal Ou &\to\cal O^*u \\
        s &\mapsto e^{2\pi i s}
    \end{align*}
    donde $(e^{2\pi is})(p)=e^{2\pi is(p)}$ para todo $p\in S$.
    
    Para cada $u\in\Omega S$, tenemos
    \begin{align*}
        \exp_u(s) = 1
        &\iff e^{2\pi i s}=1 \\
        &\iff \forall p\in u, \quad s(p)\in\Z \\
        &\iff s \in \{ f:u\to\Z \mid f \text{ localmente constante } \} \\
        &\iff s \in \underline\Z u.
    \end{align*}
    donde $\underline\Z$ es la notación para la gavilla de funciones
    localmente constantes. Así,
    \[
        \ker(\exp) = \underline\Z
    .\]
\end{example}
\begin{example}
    Sea $S$ una superficie de Riemann.
    Tenemos las gavillas $\cal C$ y $\cal C^\infty$ de funciones
    continuas y funciones lisas, con estructura aditiva.
    
    Sabemos que una función $f=u+iv:U\to\C$ es holomorfa si, y solo si,
    cumple las ecuaciones de Cauchy-Riemann:
    \begin{align*}
        \frac{\partial u}{\partial x} &= \frac{\partial v}{\partial y} \\
        \frac{\partial u}{\partial y} &= -\frac{\partial v}{\partial x},
    \end{align*}
    lo cual sucede si, y solo si,
    \[
        \left(
        \frac{\partial u}{\partial x} - \frac{\partial v}{\partial y}
        \right)
        + i \left(
        \frac{\partial u}{\partial y} + \frac{\partial v}{\partial x}
        \right) = 0.
    \]
    Esto se puede escribir como
    \[
        \left(
        \frac{\partial}{\partial x} + i\frac{\partial}{\partial y}
        \right)(u+iv)
        = 0.
    \]
    La linealidad de la derivada nos dice que
    \[
        \ol\partial
        = \frac{\partial}{\partial \ol z}
        = \left(
        \frac{\partial}{\partial x} + i\frac{\partial}{\partial y}
        \right)
        :\cal C^\infty\to\cal C^\infty
    \]
    es un morfismo de gavillas abelianas.
    Por lo tanto, tenemos que $f=u+iv\in\cal C^\infty(U)$
    es holomorfa si, y solo si, $\ol\partial f=0$, así que
    \[
        \cal O = \Ker \ol\partial
    .\]
\end{example}

A pesar de que el núcleo de un morfismo de gavillas es una gavilla,
la imagen de un morfismo de gavillas no es, en general, una gavilla,
como lo muestra el siguiente ejemplo.

\begin{example}
Consideremos la superficie de Riemann $S=\C^*$ y el morfismo de gavillas
abelianas
\[
    \exp:\cal O\to\cal O^*
\]
definido como en el ejemplo anterior,
donde vimos que $\ker(\exp)=\underline\Z$.

Veremos que la pregavilla $\Im(\exp)$ no satisface el axioma de pegado.
Para esto, consideremos la cubierta de $S$ por los dos abiertos
\begin{align*}
    U_1 &= \C^*\setminus\R^-
    &
    U_2 &= \C^*\setminus\R^+.
\end{align*}
Notemos que $U_1$ y $U_2$ son simplemente conexos, por lo cual
tenemos ramas del logaritmo $\log_j\in\cal O(U_j)$ tales que
Por si acaso, esto no denota el logaritmo base $1$ y base $2$,
sino funciones holomorfas $\log_j:U_j\to\C$ que cumplen
\[
    e^{\log_j(z)}=z
\]
para todo $z\in U_j$.

Ahora consideremos las secciones $s_j\in\cal O^*(U_j)$
dadas por $s_j(z)=z$. Estas secciones están en $\Im(\exp)(U_j)$, ya que
\begin{align*}
    s_j(z)
    &= z \\
    &= e^{\log_j(z)} \\
    &= e^{2\pi i\log_j(z)/(2\pi i)} \\
    &= \exp(f_j)(z),
\end{align*}
donde $f_j(z) = \log_j(z)/(2\pi i)$.
En particular, $f_j\in\cal O(U_j)$, así que
$s_j=\exp(f_j)\in\Im(\exp)$.

Además, tenemos $U_1\cap U_2=\C^*\setminus\R$, por lo cual
\[
    s_1|(U_1\cap U_2) = s_2|(U_1\cap U_2)
,\]
sin embargo, el pegado de $s_1$ y $s_2$
es la sección global $s\in\cal O(S)$ dada por $s(z)=z$ para todo $z\in S$,
la cual no está en $\Img(\exp)(S)$,
pues no se puede escribir como $s=\exp(f)$ porque el logaritmo
no se puede definir en todo $S$, ya que éste no es simplemente conexo.

Se sigue que la pregavilla $\Img(\exp)$ no es una gavilla.
\end{example}

PENDIENTE SESIÓN 19
\todo{SES 19}

Tenemos un morfismo natural 
$$\theta: \F \rightarrow \F^+,$$
que definimos $U\in\Omega X$ tal que $\theta_U: \F(U) \rightarrow \F^+(U)$, tomamos una sección $s^+=\theta_U(s):U\rightarrow |F|$ tal que 
$s^+(x)=[s]_x\in\F_x$. 

De la observación REFERENCIA tenemos que si $\F$ es gavilla entonces $\F^+\cong \F$.

\begin{proposition}
Si $\F$ es pregavilla y $\G$ gavilla tomemos dos morfismos
$$ 
\begin{tikzcd}
\F \arrow[r, "\varphi", shift left] \arrow[r, "\psi"', shift right] & \G
\end{tikzcd}
$$
tal que $\varphi_x=\psi_x$ para toda $x\in X$ entonces $\varphi=\psi$
\end{proposition}
\begin{proof}
Basta ver que para toda $U\in\Omega X$, $\varphi_U=\psi_U$. Sea $U\in \Omega X$ y $s\in\F(U)$, tomemos $[s]_x\in\F_x$ para $x\in U$. 
Notemos ahora que 
\begin{eqnarray*}
[\varphi s]_x & = & \varphi_x[s]_x\\
& = & \psi_x[s]_x\\
& = & [\psi s]_x
\end{eqnarray*}
como coinciden, entonces existe $W$ una vecindad de $x$ tal que 
$\varphi s|_W=\psi s|_W$, como esto es para toda $x\in U$, entonces $\{W_x\}$ forman una cubierta de $U$. Como $\G$ es gavilla, existe una única sección que las representa en $\G(U)$.
\end{proof}
Si $\F$ es gavilla, $\theta:\F\rightarrow \F^+$ es biyectiva. 

Para ver la inyectividad tomemos dos secciones $s,s'$ tal que $\theta(s)=\theta(s')$, es decir, $s^+(x)=s'^+(x)$ para toda $x\in U\in\Omega X$, o bien, $[s]_x=[s']_x$ y así $s|_W=s'|_W$ para una vecindad $W$ de $x$,
como $\F$ es gavilla $s=s'$
Probemos la sobreyectividad, sea $s'\in\F^+(U)$ y $x\in U$, tomemos $U_x$ una vecindad de $x$ y $s(x)\in\rangle U_x,s'\langle$, como $U_x$ forman una cubierta para $U$, entonces existe una sección $s\in\F(U)$ tal que $s|_{U_x}=s$ y así $\theta(s)=s(x)=s'(x)$.

Por lo tanto, $\theta$ es isomorfismo.

Ahora, supongamos que si $\F$ es una pregavilla en $X$ y $\G$ es una gavilla en $X$ y además tenemos 
$$
\begin{tikzcd}
\F \arrow[r, "\varphi"] \arrow[d] & \G \arrow[d] \\
\F^+ \arrow[r, "\varphi^+"']      & \G^+        
\end{tikzcd}
$$
que es un diagrama conmutativo.
Entonces como $\theta_G$ es un isomorfismo, existe un flecha de $\varphi^*:=\theta_G^{-1}\varphi^+:\F^+\rightarrow \G$ que factoriza a $\varphi$, es decir, $\varphi= \theta_G^{-1}\varphi^+\theta_F$ ya que 
$\theta_G\varphi= \varphi^+\theta_F$. 
Y además, $\varphi^*$ es único. 

\section{Cohomología de Cech}
\todo{Exposición Alejandro}
\subsection{Sucesión exacta larga en cohomología}
Notemos que si $\phi:\mathcal{F}\rightarrow \mathcal{G}$ es un morfismo de gavillas, entonces $\phi$ induce un morfismo en los grupos de cohomología de $\check{C}$ech.
$$\phi_*:\check{H}^n(X,\mathcal{F})\rightarrow \check{H}^n(X,\mathcal{G}) $$
tal que:
\begin{itemize}
    \item $id_*=id$
    \item $(\phi\circ \psi)_*=\phi_*\circ \psi_*$
\end{itemize}
\subsection{El homomorfismo de conexión}
Tomemos $\phi:\mathcal{F}\rightarrow \mathcal{G}$ un homomorfismo de gavillas sobreyectivo entonces 
$$
\xymatrix{0\ar[r]&\mathcal{K}\ar[r]^{i}& \mathcal{F}\ar[r]^\phi & \mathcal{G} \ar[r]& 0}
$$
es una sucesi\'on exacta de gavillas en $X$, donde $\mathcal{K}$, denota la gavilla kernel de $\phi$.  
Definamos un homomorfismo llamado \textit{homomorfismo de conexi\'on}
$$\Delta:\check{H}^{0}(X,\mathcal{G})\rightarrow \check{H}^1(X,\mathcal{K}),$$
que debido al RESULTADO REFERENCIA se puede escribir como 
$$\Delta: \mathcal{G}(X)\rightarrow \check{H}^1(X,\mathcal{K}),$$
para definirlo tomemos $g\in\mathcal{G}(X)$, ya que $\phi$ es sobreyectivo para cada punto $p$ existe una vecindad $U_p$ de $p$ y un elemento $f_p\in\mathcal{F}(U_p)$ tal que $g|_{U_p}=\phi(f_p)$. 
Note que $\mathcal{U}=\{U_p\}$ es una cubierta abierta de $X$; sea $h_{pq}=f_q-f_p \in \mathcal{U}_{pq}$, es claro que $(h_{pq})$ es un $1$-cociclo para la gavilla $\mathcal{F}$;
adem\'as $\phi(h_{pq})=0$, ya que la diferencia en la intersección $U_p\cap U_q$ es $g-g$. 
Por tanto, $(h_{pq})$ es un $1$-cociclo para la gavilla kernel $\mathcal{K}$,
y representa una clase de cohomolog\'ia en $\check{H}^1(\mathcal{U,K})$. Esta imagen en $\check{H}^1(X,\mathcal{K})$ es denotada por $\Delta(g)$ (\cite{Kato}), además se prueba que no depende de las preimágenes ni de la cubierta elegida para $X$ como podemos apreciar en el siguiente resultado (ver \cite{Mir}). 

\begin{lemma}
La construcci\'on de $\Delta(g)$ es independiente de $\mathcal{U}$ y de las preim\'agenes $f_p$
\end{lemma}
\begin{proof}
Primero demostremos la independencia de las preim\'agenes. Fijemos $\mathcal{U}$. 
Supongamos que existen, para cada $p$, $f_p$ y $f'_p\in\mathcal{F}(U_p)$ tales que 
$$\phi(f_p)=\phi(f'_p)=g|_{U_p}.$$
Sean $h_{pq}=f_q-f_p$ y $h'_{pq}=f'_q-f'_p$.
Definamos $k_i=f_i-f'-i\in\mathcal{F}(U_i)$; 
note que en efecto $k_i\in\mathcal{K}(U_i)$ para cada $i$, as\'i que $(k_i)$ es una $0$-cocadena para $\mathcal{K}$. 
Adem\'as, $d(k_i)=(l_{pq})$, donde 
$$l_{pq}=k_q-k_p=(f_q-f'_q)-(f_p-f'_p)=h_{pq}-h'_{pq}.$$
Por lo tanto, la diferencia $(h_{pq})-(h'_{pq})=(l_{pq})$ es una cofrontera, y ambos representan el mismo elemento en el grupo de cohomolog\'ia $\check{H}^1(\mathcal{U,K})$.

Para demostrar la independencia de la cubierta abierta podemos suponer, sin p\'erdida de generalidad, que $\mathcal{V}\prec \mathcal{U}$, 
de lo contrario, consideramos un refinamiento com\'un, y sean $f_p\in\mathcal{F}(U_p)$ preim\'agenes de $g|_{U_p}$ y $r$ una aplicaci\'on de refinamiento. 
En $V_q$ definimos $f'_q=f_{r(q)}|_{V_q}$ y obtenemos preim\'agenes de $g$ en los $V_q$. Acabaos de demostrar que podemos usar los $f'_q$ para calcular $\Delta(g)$ a trav\'es de $\mathcal{V}$, tal como podr\'amos usar cualquier otro conjunto de preim\'agenes de $g$ en $\mathcal{V}$.
Como $\tilde{r}((f_p))=(f'_p)$ tenemos que su imagen por el homomorfismo can\'onico en $\check{H}^1(X,\mathcal{F})$ ser\'a la misma.
\end{proof}

Este homomorfismo de conexi\'on nos ayuda para saber cu\'ando una secci\'on global $g\in\mathcal{G}(X)$ es la imagen de una secci\'on global de $\mathcal{F}(X)$ mediante $\phi$.
\begin{theorem}
Sea $\phi:\mathcal{F}\rightarrow \mathcal{G}$ un epimorfismo de gavillas y $g\in\mathcal{G}(X)$ una secci\'on global de $\mathcal{G}$. 
Existe una secci\'on global $s\in\mathcal{F}$ tal que $\phi(s)=g$ si y solo si $\Delta(g)=0$.
\label{lemma315}
\end{theorem}
\begin{proof}
$(\Rightarrow) $ Supongamos que $\phi(s)=g$. Entonces en la construcci\'on del homomorfismo de conexi\'on podemos tomar para todo $p\in X$, 
una vecindad $U_p=X$ y una secci\'on $f_p=s$, en este caso tendr\'iamos para todo $p,q$, usando la notaci\'on anterior, $h_{pq}=0$ que inducen el elemento cero en la cohomolog\'ia. 

$(\Leftarrow)$ Supongamos que $\Delta(g)=0$ en $\check{H}^1(X,\mathcal{F})$, entonces para la cubierta $\mathcal{U}$, tenemos que $(h_{pq})$ es una cofrontera, es decir, $h_{pq}=k_q-k_p$ para cierta $(k_p)$ $1$-cocadena de $\mathcal{K}$. Sea $s_p=f_p-k_p$, siendo $f_p$ la preimagen de $g$ en $U_p$. 
En $U_{pq}$ tenemos 
$$ s_p-s_q=(f_p-k_p)-(f_q-k_q)=(k_q-k_p)-(f_q-f_p)=k_q-k_p-h_{pq}=0, $$
as\'i por el axioma de gavilla, las secciones $\{s_p\}$ juntas son las secci\'on $s\in\mathcal{F}(X)$ tal que $s|_{U_p}=s_p$ y como $$g|_{U_p}=\phi(f_p)=\phi(s_p+k_p)=\phi(s_p)=\phi(s|_{U_p}),$$
y nuevamente por el axioma de gavilla se obtiene lo deseado. 
\end{proof}

Un resultado del Teorema anterior es el siguiente:
\begin{corollary}
Sea $\phi:\mathcal{F}\rightarrow \mathcal{G}$ un homomorfismo de gavillas con gavilla kernel $\mathcal{K}$. Si $\check{H}^1(X,\mathcal{K})=0$ la correspondencia entre secciones globales $\phi_X:\mathcal{F}(X)\rightarrow\mathcal{G}(X)$ es sobreyectiva.
\end{corollary}

Observemos que el teorema anterior puede expresarse diciendo que la sucesi\'on 
\begin{equation}
\label{exactacorta}
\xymatrix{ \mathcal{F}(X) \ar[r]^{\phi_X} & \mathcal{G}(X)\ar[r]^{\Delta} & \check{H}^1(X,\mathcal{K})  }.
\end{equation}
es exacta. Además, podemos notar que es una pequeña parte de una sucesión exacta más grande como se muestra a continuación:
\begin{proposition}
Sea $\phi:\mathcal{F}\rightarrow \mathcal{G}$ un epimorfimos de gavillas con kernel $\mathcal{K}$. 
Entonces la sucesi\'on 
\begin{equation}
\xymatrix{0 \ar[r] &  \mathcal{K}(X) \ar[r]^i &  \mathcal{F}(X) \ar[r]^{\phi_X} &  \mathcal{G}(X) \ar[r]^{\Delta} &  \check{H}^1(X,\mathcal{K}) \ar[r]^{i_*} & \check{H}^1(X,\mathcal{F}) \ar[r]^{\phi_*} & \check{H}^1(X,\mathcal{G}) }
\end{equation}
es exacta en cada paso. 
\end{proposition}
\begin{proof}
La exactitud de $\mathcal{K}$ y $\mathcal{F}$ se da por definici\'on; mientras que la exactitud de $\mathcal{G}$ es justamente el Teorema \ref{lemma315}. 

Ahora comprobemos la exactitud en el resto. Veamos primero que $ \Im(\Delta)\subset\ker(i_*)$, supongamos que $g\in\mathcal{G}(X)$. Por construcci\'on de $\Delta(g)$ tomamos una cubierta abierta ${\mathcal{U_i}}_{i \in I}$ y buscamos elementos $f_i\in\mathcal{F}(U_i)$ con $\phi_{U_i}(f_i)=g|_{U_i}$; 
entonces $\Delta(g)$ es definido por un $1$-cociclo $k_{ij}=f_j-f_i$ para la gavilla $\mathcal{K}$. 
Pero este $1$-cociclo es una cofrontera en la gavilla $\mathcal{F}$.

Para finalizar la exactitud en $\check{H}^1(X,\mathcal{K})$, veamos que $\ker(i_*)\subset \Im(\Delta)$. 
Suponga que $(k_{ij})$ es un $1$-cociclo para la gavilla $\mathcal{K}$ que representa una clase en el kernel de $i_*$. 
Entonces $(k_{ij})$ es una cofrontera, considerada como un $1$-cociclo para la gavilla $\mathcal{F}$, as\'i existe una $0$-cocadena $(f_i)$ tal que $k_{ij}=f_j-f_i $ en $U_{ij}$ para toda $i,j$. 
Consideremos la $0$-cocadena $(g_i)$ para $\mathcal{G}$,
donde $g_i=\phi(f_i)$. 
Note que 
$$g_i-g_j=\phi(f_i-f_j)=\phi(k_{ij})=0 \qquad\textup{ en } U_{ij},$$ 
as\'i por el axioma de gavilla para $\mathcal{G}$ existe una secci\'on global $g\in\mathcal{G}(X)$ tal que $g|_{U_i}=g_i$ para toda $i$. 
Adem\'as de la construcci\'on de $\Delta$ tenemos que $\Delta(g)$ es una clase de $(k_{ij})$. 

Finalmente, mostramos la exactitud en $\check{H}^1(X,\mathcal{F})$.
Es claro que $i_*\circ\phi_*=0$, por lo que solamente necesitamos probar que $\ker(\phi_*)\subset\Im(i_*)$. 
Sea $c$ una clase en $\ker(\phi_*)$  representada por un $1$-cociclo $(f_{ij})$ con respecto a la cubierta $\mathcal{U}$ tal que $\phi(f_{ij})=g_j-g_i$ para todo $i,j\in J$.
Desp\'ues de refinar $\mathcal{U}$ (si fuera necesario) ya que $\phi$ es un epimorfismo de gavillas, que cada $g_i=\phi(f_i)$ para cada elemento $f_i\in\mathcal{F}(U_i)$.

Sea $h_{ij}=f_{ij}-f_j+f_i\in\mathcal{F}(U_{ij})$; 
es claramente un $1$-cociclo ya que $(f_{ij})$ lo es. 
Aplicando $\phi$, vemos que 
$\phi(h_{ij})=\phi(f_{ij})-g_i+g_i=0,$
as\'i que $(h_{ij})$ es actualmente un $1$-cociclo para la gavilla kernel $\mathcal{K}$. 
Ya que difiere del cociclo $(f_{ij})$ por la cofrontera de la $0$-cocadena $(f_i)$, esto tambi\'en da la clase orginal $c$ en cohomolog\'ia. 
As\'i $c$ est\'a en la imag\'en de $i_*$.
\end{proof}
 Así, notamos que una sucesi\'on exacta corta de gavillas nos brinda una sucesi\'on exacta larga en cohomolog\'ia. 

\begin{theorem}
Sea $X$ un espacio paracompacto y sea 
$$ 
\xymatrix{ 0 \ar[r] & \mathcal{K} \ar[r] & \mathcal{F} \ar[r] & \mathcal{G} \ar[r] & 0 }
$$
una sucesi\'on exacta corta de gavillas en $X$. Entonces existe un homomorfismo de conexi\'on $\Delta:\check{H}^{n}(X,\mathcal{G})\rightarrow\check{H}^{n+1}(X,\mathcal{G})$ para cada $n\geq 0$ tal que la sucesi\'on de grupos de cohomolog\'a
$$ 
\xymatrix{0 \ar[r] & \check{H}^0(X,\mathcal{K}) \ar[r]^{i_*} & \check{H}^0(X,\mathcal{F}) \ar[r]^{\phi_*} & \check{H}^0(X,\mathcal{G}) \ar[r]^{\Delta} & \\
  \ar[r] & \check{H}^1(X,\mathcal{K}) \ar[r]^{i_*} & \check{H}^1(X,\mathcal{F}) \ar[r]^{\phi_*} & \check{H}^1(X,\mathcal{G}) \ar[r]^{\Delta} & \\
  \ar[r] & \check{H}^2(X,\mathcal{K}) \ar[r]^{i_*} & \check{H}^2(X,\mathcal{F}) \ar[r]^{\phi_*} & \check{H}^2(X,\mathcal{G}) \ar[r]^{\Delta} &}
$$
es exacta.
\end{theorem}

\subsection{Axiomas de cohomología}
\begin{definition}
Una gavilla $\F$ sobre $X$ es fina si para cada cubierta abierta localmente finita $\mathcal{U}=\{U_i\}$ de $X$ existe para cada $i$ un endomorfismo $f_i$ de $\F$ tal que:
\begin{enumerate}
    \item $\Supp(f_i)\in U_i$
    \item $\sum_i f_i =\id$
\end{enumerate}
\end{definition} 
o bien, decimos que $\F$ es fina si para dos conjuntos disjuntos $A\cap B\neq 0$ cerrados de $X$, existe un endomorfismo $\F\rightarrow \F$ cuya restricción a la identidad está en una vecindad de $A$ y el endomorfismo cero en una vecindad de $B$.

\begin{example}
La gavilla de funciones continuas $\mathcal{C}_X$ es una gavilla fina sobre un espacio topológico Hausdorff paracompacto. Ya que para cualquier cubierta abierta finita $\mathcal{U}=\{U_i\}$ de $X$ existe una partición de la unidad $\{\varphi_i\}$ subordinada a dicha cubierta. Usando estas $\varphi_i$ definimos $\eta_i:\mathcal{C}\rightarrow \mathcal{C}$ tal que 
$\eta_i(U):\mathcal{C}(U)\rightarrow \mathcal{C}(U)$ está dado por 
$\varphi_i\cdot s$, para todo $U\in\mathcal{U}$. 
Claramente es un morfismo de gavillas para cada $\eta_i$ es un morfismo de gavillas para cada $i$. Además, 
\begin{enumerate}
    \item Si $x\in X-U_i$, entonces $\eta_i(\mathcal{C}_x)=0$, ya que para cualquier $[s]_x\in\mathcal{C}_x$, representado por $\langle s,U\rangle$, con $x\in U$ podemos elegir $U$ tal que $U\subseteq X-U_i$ y así, 
    $$\eta_i[s]_x=[\eta_i(U)(s)]_x=[\varphi_i\cdot s]_x=0 $$
    \item Sea $s\in\mathcal{C}(X)$ una sección global, entonces para cada $x\in X$ se tiene
    $$\sum_i \eta_i (s)(x)=\sum_i (\varphi_i\cdot s )(x)=\sum_i (\varphi_i(x))\cdot(s(x))=s(x)\sum_i\varphi_i(x)=s(x)$$
\end{enumerate}
\end{example}

Con esta nueva definición en mente y lo que hemos visto hasta el momento de teoría de cohomología de $\check{C}$ech, veamos que en efecto es una teoría de cohomología.

Una teoría de cohomología de gavillas $\mathcal{M}$ para un espacio topológico $X$ con gavillas en $R$-módulos sobre $X$ consta de: 
\begin{enumerate}
    \item Un $R$-módulo $H^q(X,\F)$ para cada gavillas $\F$ y para cada entero $q$. 
    \item Un homomorfismo $H^q(X,\F)\rightarrow H^q(X,G)$ para cada homomorfismo $\F\rightarrow \G$ y para cada entero $q$. 
    \item Un homomorfismo $H^q(X,\F'')\rightarrow H^{q+1}(X,\F)$ para cada sucesión exacta corta
    $$\xymatrix{
    0\ar[r] & \F' \ar[r] & \F \ar[r] & F''\ar[r] & 0
    } $$
    y para cada entero $q$. Tal que las siguientes propiedades se cumplen:
    \begin{enumerate}
        \item $H^q(X,\F)=0$ para $q<0$, y existe un isomorfismo $$H^0(X,\F)\cong \F(X)$$
        tal que para cada homorfismo  $\F\rightarrow \F'$ el siguiente diagrama conmuta
$$
\begin{tikzcd}
{H^0(X,\F)} \arrow[r, "\cong"] \arrow[d] & \F(X) \arrow[d] \\
{H^0(X,\F')} \arrow[r, "\cong"']         & \F''(X)        
\end{tikzcd}
$$
\item $H^q(X,\F)=0$ para toda $q>0$ si $\F$ es una gavilla fina.
\item Si 
$$\xymatrix{
    0\ar[r] & \F' \ar[r] & \F \ar[r] & F''\ar[r] & 0
    } $$
    es exacta, entonces la siguiente sucesión es exacta 
    $$
    \begin{tikzcd}
\cdots  \arrow[r] & {H^q(X,\F)} \arrow[r] & {H^q(X,\F'')} \arrow[r] & {H^{q+1}(X,\F')} \arrow[r] & {H^{q+1}(X,\F')} \arrow[r] & \cdots
\end{tikzcd}
    $$
     \item El homorfismo identidad $\id:\F\rightarrow \F$ induce el morfismo identidad $\id_*: H^q(X,\F)\rightarrow H^q(X,\F)$.
    \item Si el diagrama 
    $$ 
    \begin{tikzcd}
\F \arrow[r] \arrow[rd] & \F' \arrow[d] \\
                        & \F''         
\end{tikzcd}
    $$
    conmuta, entonces para cada $q$ también lo hace el siguiente diagrama
    $$
    \begin{tikzcd}
H^q(X,\F) \arrow[r] \arrow[rd] & H^q(X,\F')  \arrow[d] \\
                        & H^q(X,\F'')        
\end{tikzcd}
    $$
    \item Para cada homomorfismo de sucesiones exactas de gavillas 
    $$
    \begin{tikzcd}
0 \arrow[r] & \F' \arrow[r] \arrow[d] & \F \arrow[r] \arrow[d] & \F'' \arrow[r] \arrow[d] & 0 \\
0 \arrow[r] & \G' \arrow[r]           & \G \arrow[r]           & \G'' \arrow[r]           & 0
\end{tikzcd}
    $$
    el siguiente diagrama conmuta
    $$
    \begin{tikzcd}
{H^q(X,\F'')} \arrow[r] \arrow[d] & {H^{q+1}(X,\F')} \arrow[d] \\
{H^q(X,\G'')} \arrow[r]           & {H^{q+1}(X,\G'')}         
\end{tikzcd}
    $$
    \end{enumerate}
\end{enumerate}
El módulo $H^q(X,\F)$ es llamado $q$-ésimo módulo de la cohomología de $X$ relativo a la cohomología $\mathcal{M}$.

En todo momento tomemos $\mathcal{U}=\{U_i\}$ una cubierta abierta de $X$. 
una colección $(U_0,\ldots, U_q)$ de elementos de $\mathcal{U}$ tal que $U_0\cap\cdots\cap U_q\neq 0$ es llamado $q$-simplejo y denotado por $|\sigma|$. Denotamos, además, por $\sigma^i=(U_0,\ldots U_{i-1},U_{i+1},\ldots , U_q)$. 
Tomemos $C^q(\mathcal{U},\F)$ para $q\geq 0$, el $R$-módulo que consiste en funciones que asignan a cada $q$-simplejo un elemento de $\F(|\sigma|)$ y $C^q(\mathcal{U},\F)=0$ para cada $q<0$. Los elementos de $C^q(\mathcal{U}, \F)$ son llamados $q$-cocadenas. Además, tenemos un homomorfismo cofrontera 
$$d:C^q(\mathcal{U},\F)\rightarrow C^{q+1}(\mathcal{U},\F), $$
con el cual obtenemos un  complejo de cadenas cuyo $q$-ésimo módulo de cohomología de $\check{C}$ech es denotado por $\check{H}^q(X,\F)$.

Hemos visto por el RESULTADO REFERENCIA que $C^q(\mathcal{U},\F)=0$ para todo $q<0$ y por lo tanto $\check{H}^q(X,\F)=0$, lo cual prueba el inciso (a). 

Además, el homomorfismo $\check{H}^q(\mathcal{U},\F)\rightarrow \check{H}^q(\mathcal{U},\F')$ para toda $q$ inducido por $\F\rightarrow \F'$ conmuta con los homorfismos de refinamiento 
$$\check{H}^q(\mathcal{U},\F)\rightarrow \check{H}^q(\mathcal{V},\F)$$
y así induce homomorfismos en 
$$\check{H}^q(X,\F)\rightarrow \check{H}^q(X,\F') $$
los cuales satisfacen (d) y (e).

Para probar (b), tomemos $\eta_i:\F \rightarrow \F$ en la cubierta $\mathcal{U}$, es suficiente probar que $\check{H}^q(\U,\F)=0$. Sea $f\in C^p(\U,\F)$, tomemos $\sigma=(U_0,...,U_{p-1})$ un $(p-1)$- simplejo, entonces 
$$f_i=\eta_i(f(U_i,U_0,\ldots,U_{p-1})) =\eta_i f(U_i,U_0,\ldots, U_{p-1})\in \F (U_i\cap U_0\cap\cdots \cap U_{p-1}).$$
Como $\eta_i=0$ en $X-U_i$, entonces $\eta_i=0$ en $U-(U\cap U_i)$
con soporte $\Supp(f_i)\subset U_i\cap U$. Así, podemos extender por cero a  $f_i$ a una sección continua de $\F$ sobre $U_0\cap\cdots \cap U_{p-1}$ de la siguiente forma: 
\begin{itemize}
    \item El abierto $U=U_0\cap\cdots \cap U_{p-1}$ tiene una cubierta abierta dada por $U_1=U-\Supp(f_i)$ y $U_2=U\cap U_i$.
    \item Tenemos secciones $f_i\in \F(U_2)$, $0\in \F(U_1)$ tales que coinciden, es decir, $f_i=0\in\F(U_1\cap U_2)$.
\end{itemize}
entonces como $\F$ es gavilla existe una única sección, denotada $$F_{|\sigma|}(\eta_if)\in \F(U) $$, llamada \textit{extensión por cero} de $f_i$ a $U$, tal que 
$$F_{|\sigma|}(\eta_if)|_{U\cap U_i}=f_i \in\F(U\cap U_i) $$
Ahora, estas secciones, definen una $(q-1)$-cocadena denotada $g_i\in C^{q-1}(\U,\F)$. Tal que para cualquier $q$-simplejo $\sigma=(U_0,...,U_q )$: 
\begin{eqnarray*}
d g_i(\sigma) & = & \sum_{j=0}^q(-1)^jg_i(\sigma^j)|_\sigma\\
& = & \sum_{j=0}^q(-1)^j F_{\sigma^j}(\eta_if)(U_i\cap U_j)|_\sigma\\
& = & \eta_if(\sigma)-F_\sigma d\eta_if(U_i\cap U_0\cap \cdots \cap U_{q}).
\end{eqnarray*}
Así, $dg_i(\sigma)= \eta_if(\sigma)$ ya que $d\eta_if=0$ porque $f$ es un cociclo. Finalmente, concluimos que $dg_i= \eta_if$ para toda $i$. 

Definimos ahora, $g=\sum_i g_i\in C^{q-1}(\U,\F)$, la suma es finita ya que para cada $i$ y cada $(q-1)$-simplejo $\sigma$ se tiene que $g_i(\sigma)=0$ fuera de $U_i$, así para cualquier $x\in X$ existe un número finito de abiertos $U_i$ que lo contienen, ya que $\U$ es localmente finita. Entonces, solo en esos $U_i$, $g_i\neq 0$.

Finalmente, para todo $(q-1)$-simplejo $\sigma$, tenemos 
\begin{eqnarray*}
d g(\sigma) & = &  d(\sum_i g_i(\sigma)) \\
& = & \sum_i dg_i(\sigma) \\
& = & \sum_i \eta_i f(\sigma) \\
& = & f(\sigma).
\end{eqnarray*} 
Así, todo $q$-ciclo $f$ con $q>0$ es una cofrontera y $\check{H}^q(\U,\F)=0$.

\begin{thebibliography}{X}
\bibitem{Kato} Kato, G. (2006).\textit{ The heart of cohomology}. Springer Science \& Business Media.

\bibitem{Mir} Miranda, R.  (2003).  \textit{Riemann Curves and Algebraic curves.} Cambridge studies in advanced mathematics.

\bibitem{Rot} Rotman, J. J. (2008). \textit{An introduction to homological algebra.} Springer Science \& Business Media.

\bibitem{War} Warner, F. W. (2013). \textit{Foundations of differentiable manifolds and Lie groups}. Springer Science \& Business Media.

\bibitem{Wei} Weibel, C., \& Butler, M. C. R. (1996). \textit{An introduction to homological algebra}. Bulletin of the London Mathematical Society.

	\end{thebibliography}
\end{document} 
